\documentclass[a4paper,12pt,oneside,openany]{memoir}
\usepackage{sty/preamble}

\newlength{\alphabet}
\settowidth{\alphabet}{\normalfont abcdefghijklmnopqrstuvwxyz}

%% Page setup
\usepackage{geometry}
\geometry{
    % margin=2cm,
    vmargin=2cm,
    hmargin=3cm,
    % hmargin=1in,
    % textwidth=2.3\alphabet,
    includehead,
    % includefoot,
    heightrounded,
    % showframe,
    % pass,
}
\pagestyle{fancy}
\fancyfoot{}
\fancyfoot[R]{\thepage}

\setcounter{tocdepth}{1}
\setcounter{secnumdepth}{1}

\setlength{\parindent}{2pc}

%% Frame around the text area
% \usepackage{showframe}
% \renewcommand\ShowFrameLinethickness{0.1pt}
% \renewcommand*\ShowFrameColor{\color{lightgray}}

% \usepackage{indentfirst}
% \usepackage{tocbibind}
\usepackage{verse}
\usepackage{paracol}
\usepackage{xlop}

\let\providelength\undefined
\let\providecounter\undefined
\usepackage{moredefs}


\title{ГЁДЕЛЬ, ЭШЕР, БАХ: \\ эта бесконечная гирлянда}
\author{Дуглас Ричард Хофштадтер}


\begin{document}

\frontmatter

\maketitle

\mainmatter

% Праздничное предисловие автора к русскому изданию книги «Гёдель, Эшер Бах»
\subfile{parts/preface}

\begingroup
\hypersetup{hidelinks}
\clearpage
% \begin{KeepFromToc}
\tableofcontents
% \end{KeepFromToc}
\clearpage
\endgroup

% Обзор
\subfile{parts/overview}
% Список иллюстраций
\subfile{parts/illustrations}
% Благодарность
\subfile{parts/acknowledgment}

\part{ЧАСТЬ~I}

% TODO: illustration 1
% \emph{Рис. I. Иоганн Себастиан Бах в 1748. С портрета кисти Элиаса Готтлиба Хауссманна.}

% Интродукция: музыко-логическое приношение
\subfile{parts/introduction}
% Трехголосная инвенция
\subfile{parts/dial01}
% Глава 1. Головоломка MU
\subfile{parts/ch01}
% Двухголосная инвенция
\subfile{parts/dial02}
% Глава 2. Содержание и форма в математике
\subfile{parts/ch02}
% Соната для Ахилла соло
\subfile{parts/dial03}
% Глава 3. Рисунок и фон
\subfile{parts/ch03}
% Акростиконтрапунктус
\subfile{parts/dial04}
% Глава 4. Непротиворечивость, полнота и геометрия
\subfile{parts/ch04}
% Маленький гармонический лабиринт
\subfile{parts/dial05}
% Глава 5. Рекурсивные структуры и процессы
\subfile{parts/ch05}
% Канон с интервальным увеличением
\subfile{parts/dial06}
% Глава 6. Местонахождение значения
\subfile{parts/ch06}
% Хроматическая фантазия и фига
\subfile{parts/dial07}
% Глава 7. Исчисление Высказываний
\subfile{parts/ch07}
% Крабий канон
\subfile{parts/dial08}
% Глава 8. Типографская теория чисел
\subfile{parts/ch08}
% Приношение «МУ»
\subfile{parts/dial09}
% Глава 9. Мумон и Гёдель
% \subfile{parts/ch09}
% %
% \subfile{parts/dial10}
% %
% \subfile{parts/dial11}
% %
% \subfile{parts/dial12}
% %
% \subfile{parts/dial13}
% %
% \subfile{parts/dial14}

% \include{fragments/xx08}
% \subsubsection{Трехголосная инвенция}
% \subsubsection{Трехголосная инвенция}

\emph{Ахилл (греческий воин, самый быстроногий из смертных) и Черепаха стоят рядом на пыльной беговой дорожке; жара, палит солнце. Далеко в конце дорожки на высоком флагштоке висит большой прямоугольный ярко-красный флаг. В центре флага вырезана дыра в форме кольца, сквозь которую видно небо.}

\emph{Ахилл} : Что~это за странный флаг там, на другом конце дорожки? Он чем-то напоминает мне гравюру моего любимого художника, Эшера.

\emph{Черепаха} : Это флаг Зенона.

\emph{Ахилл} : Не кажется ли вам, что дыра в нем похожа на отверстия в листе Мёбиуса на одной из картин Эшера? Могу поспорить, что с этим флагом что-то не в порядке.

\emph{Черепаха} : В нем вырезано кольцо в форме нуля --- любимого числа Зенона.

\emph{Ахилл} : Но ведь в то время нуль еще не был изобретен! Он будет придуман неким индусским математиком только несколько тысяч лет спустя. Это доказывает, дорогая г-жа Ч, что подобный флаг невозможен.

\emph{Черепаха} : Ваши доводы убедительны, Ахилл, и я должна согласиться, что такой флаг, действительно, не может существовать. Но все равно он замечательно красив, не правда ли?

\emph{Ахилл} : В этом я не сомневаюсь.

\emph{Черепаха} : Интересно, не связана ли его красота с его невозможностью? Не знаю, не знаю.. У меня никогда не доходили лапы до анализа Красоты. Это Сущность с Большой Буквы, а у меня никогда не хватало времени на Сущности с Большой Буквы.

\emph{Ахилл} : Кстати, о Сущностях с Большой Буквы --- вы никогда не задавались вопросом о Смысле Жизни?

\emph{Черепаха} : Бог мой, конечно же, нет!

\emph{Ахилл} : Не спрашивали ли вы себя, зачем мы здесь и кто нас изобрел?

\emph{Черепаха} : Ну, это совершенно другое дело. Нас изобрел Зенон (в чем вы сами скоро убедитесь); мы находимся здесь, чтобы бежать наперегонки.

\emph{Ахилл} : Мы --- наперегонки?. Это возмутительно! Я, самый быстроногий из смертных --- и вы медлительная, как\ldots{} как\ldots{} как Черепаха!

\emph{Черепаха} : Вы могли бы дать мне фору.

\emph{Ахилл} : Это была бы огромная фора.

\emph{Черепаха} : Ну что же, я не возражаю.

\emph{Ахилл} : Все равно я вас нагоню, раньше или позже --- скорее всего, раньше.

\emph{Черепаха} : А вот и нет, если верить парадоксу Зенона. Зенон надеялся с помощью нашего маленького соревнования доказать, что движение невозможно. По Зенону, движение происходит только в нашем воображении. Это значит, что Мир Изменяется Исключительно Иллюзорно. Он доказывает этот постулат весьма элегантно.

\emph{Ахилл} : Ах, да, теперь я припоминаю~ знаменитый коан мастера дзен-буддизма Дзенона\ldots{} тьфу!. Зенона, я имею в виду. Действительно, очень просто.

\emph{Черепаха} : Дзен коан? Дзен мастер? О чем вы говорите?

\emph{Ахилл} : Вот, послушайте\ldots{} Два монаха спорили о флаге Один сказал; «Этот флаг движется». Другой возразил: «Нет, это ветер движется». В это время мимо проходил шестой патриарх, Зенон, который сказал монахам: «Не флаг и не ветер --- движется ваша мысль!»

\emph{Рис. 10. М.К. Эшер. «Лист Мёбиуса I» (гравюра на дереве, отпечатанная с четырех блоков, 1961).}

\emph{Черепаха} : Что-то вы все путаете, Ахилл. Зенон вовсе не мастер дзен-буддизма. На самом деле, он греческий философ из города Элей, лежащего на полпути между точками А и Б. Спустя столетия, его все еще будут славить как автора парадоксов движения. В центре одного из них --- наше соревнование по бегу.

\emph{Ахилл} : Вы меня совсем сбили с толку. Я отчетливо помню, как много раз повторял наизусть имена шести патриархов дзена: «Шестой патриарх --- Зенон, шестой патриарх --- Зенон...» (Внезапно поднимается теплый ветер.) Взгляните, госпожа Черепаха, как развевается флаг! Как приятно смотреть на волны, бегущие по его мягкой ткани. И кольцо, вырезанное в нем, развевается вместе с флагом!

\emph{Черепаха} : Не смешите меня. Этот флаг в принципе невозможен, следовательно, он не может развеваться. Это движется ветер.

\emph{(В этот момент мимо идет Зенон.)}

\emph{Зенон} : День добрый! Приветствую вас! Что слышно?

\emph{Ахилл} : Флаг движется!

\emph{Черепаха} : Ветер движется!

\emph{Зенон} : О мои дражайшие друзья! Прекратите ваши словопрения! Оставьте ваши разногласия! Поберегите ваше красноречие! Я разрешу ваш спор, не сходя с места. Эгей, и в такой чудный денек!

\emph{Ахилл} : Этот тип явно дурака валяет.

\emph{Черепаха} : Нет, подождите, Ахилл, давайте-ка его послушаем. О неизвестный господин, будьте так любезны поделиться с нами вашими соображениями по этому поводу.

\emph{Зенон} : С превеликим удовольствием. Не ветер и не флаг --- на самом деле, вообще ничто не движется, что следует из моей великой Теоремы. Она гласит: «Мир Изменяется Исключительно Иллюзорно». А из этой Теоремы вытекает еще более великая Теорема, Теорема Зенона: «Мир Ультранеподвижен».

\emph{Ахилл} : Теорема Зенона? Вы, случаем, уж не Зенон ли из Элей будете?

\emph{Зенон} : Он самый, Ахилл.

\emph{Ахилл (чешет голову в замешательстве)} : Откуда он знает, как меня зовут?

\emph{Зенон} : Возможно ли убедить вас выслушать меня, чтобы вы поняли, почему это так? Я прошел сегодня от точки А до самой Элей, только затем, чтобы найти кого-нибудь, кто согласился бы послушать мои тщательно отточенные доводы. Но все встречные сразу разбегались. Им, видите ли, было некогда. Вы не представляете себе, как это разочаровывает, когда встречаешь отказ за отказом\ldots{} Однако простите меня --- я совсем замучил вас пересказом моих неприятностей. Я прошу вас только об одном: не согласитесь ли вы ублажить старика-философа и уделить несколько минут --- обещаю вам, всего лишь несколько минут --- его экстравагантным теориям?

\emph{Ахилл} : О, без сомнения! Сделайте милость, просветите нас! Я знаю, что говорю за обоих, так как моя приятельница, госпожа Черепаха, только что отзывалась о вас весьма уважительно и упоминала как раз о ваших парадоксах.

\emph{Зенон} : Благодарю вас. Видите ли, мой Мастер, пятый патриарх, учил меня, что реальность всегда одна и та же, единая и неизменная. Все разнообразие, изменение и движение --- не более, чем иллюзии наших органов чувств. Некоторые смеялись над его взглядами, но я могу доказать всю абсурдность их насмешек. Мои доводы весьма просты. Я покажу их на примере двух персонажей моего собственного изобретения: Ахилл (греческий воин, самый быстроногий из смертных) и Черепаха. В моем рассказе, прохожий убеждает их бежать наперегонки к флагу, развевающемуся на ветру в конце беговой дорожки. Предположим, что Черепаха, как гораздо более медленный бегун, получит фору, скажем, в пятьдесят локтей. Соревнование начинается. В несколько прыжков Ахилл добегает до того места, откуда стартовала Черепаха.

\emph{Ахилл} : Ха!

\emph{Зенон} : Теперь Черепаха впереди него лишь на пять метров. Ахилл вмиг достигает того места.

\emph{Ахилл} : Хо-хо!

\emph{Зенон} : Все же за этот миг Черепаха успела немного продвинуться вперед. В мгновение ока Ахилл покрывает и эту дистанцию.

\emph{Ахилл} : Хи-хи-хи!

\emph{Зенон} : Но и в это кратчайшее мгновение Черепаха чуточку продвинулась, и опять Ахилл оказался позади. Теперь вы видите, что если Ахилл хочет нагнать Черепаху, ему придется играть в эти «догонялки» БЕСКОНЕЧНО --- а следовательно, он НИКОГДА ее не догонит!

\emph{Черепаха} : Хе-хе-хе-хе!

\emph{Ахилл} : Хм\ldots{} хм\ldots{} хм\ldots{} хм\ldots{} хм\ldots{} Этот довод кажется мне неверным. Однако я никак не могу понять, в чем здесь ошибка.

\emph{Зенон} : Хороша головоломочка? Это мой любимый парадокс.

\emph{Черепаха} : Прошу прощения, Зенон, но мне кажется, что вы рассказали нам что-то не то. Через несколько веков этот ваш рассказ будет известен как парадокс Зенона «Ахилл и Черепаха»; он показывает --- гм! --- что Ахилл никогда не догонит Черепаху. Доказательство же того, что Мир Изменяется Исключительно Иллюзорно (а следовательно, Мир Ультранеподвижен) содержится в вашем «Дихотомическом Парадоксе», не так ли?

\emph{Зенон} : Ах, какой стыд. Конечно же, вы правы. Это тот парадокс, где объясняется, что идя от А до Б, надо сначала пройти половину пути --- но от этой половины также придется сначала пройти половину\ldots{} и так далее. Оба эти парадокса очень похожи; честно говоря, я просто обыгрывал мою Великую Идею с разных сторон.

\emph{Ахилл} : Могу поклясться, что эти аргументы содержат ошибку. Хотя я не вижу, где в них ошибка, зато прекрасно понимаю, что они не могут быть верными.

\emph{Зенон} : Так вы сомневаетесь в правильности моих парадоксов? Отчего же вам самим не попробовать? Видите тот красный флаг в конце дорожки?

\emph{Ахилл} : Невозможный, сделанный по гравюре Эшера?

\emph{Зенон} : Тот самый. Как насчет того, чтобы вам с Черепахой пробежаться к флагу наперегонки? Конечно, ей надо будет дать приличную фору, скажем\ldots{}

\emph{Черепаха} : Как насчет пятидесяти локтей?

\emph{Зенон} : Отлично --- пусть будут пятьдесят локтей.

\emph{Ахилл} : Я-то всегда готов.

\emph{Зенон} : Вот и чудесно. Все это захватывающе интересно! Сейчас мы проверим мою строго доказанную Теорему на опыте! Госпожа Черепаха, будьте так добры, займите позицию на пятьдесят локтей впереди Ахилла.

\emph{(Черепаха продвигается на пятьдесят локтей ближе к флагу.)}

Ну как, вы оба готовы?

\emph{Черепаха и Ахилл} : Готовы!

\emph{Зенон} : На старт\ldots{} Внимание\ldots{} Марш!


% % \subsubsection{ГЛАВА I: Головоломк MU}
% \include{fragments/xx10}
% % \subsubsection{\texorpdfstring{\emph{Двухголосная инвенция} }{Двухголосная инвенци }}
% \include{fragments/xx11}
% % \subsubsection{ГЛАВА II: Содержание и форма  математике}
% \include{fragments/xx12}
% % \subsubsection{Соната для Ахилла соло}
% \subsubsection{Соната для Ахилла соло}

\emph{Звонит телефон~--- Ахилл берет трубку.}

\emph{Ахилл} : Алло, Ахилл слушает.

\emph{Ахилл} : А, здравствуйте, г-жа Черепаха. Как дела?

\emph{Ахилл} : Кривошея и чихиллит? Что такое чихи\ldots~--- а, теперь понимаю. Будьте здоровы!\ldots{} Что и говорить, неприятная комбинация. Как это вы ухитрились такое подцепить?

\emph{Ахилл} : И долго вы ее так продержали?

\emph{Ахилл} : Еще на самом сквозняке~--- не удивительно, что вам в шею надуло!

\emph{Ахилл} : Что же вас заставило так долго там проторчать?

\emph{Ахилл} : Многие из них удивительные? Какие, например?

\emph{Ахилл:} Фантасмагорические чудища? Что вы имеете в виду?

\emph{Ахилл} : И вам не страшно было в такой компании?

\emph{Ахилл} : Гитара? Вот странно~--- откуда взялась гитара среди этих диковинных созданий. Кстати, вы играете на гитаре?

\emph{Ахилл} : Ах, для меня это одно и то же.

\emph{Ахилл} : Вы правы удивительно, как это я сам до сих пор не заметил, в чем разница между гитарой и скрипкой. Кстати о скрипках: не хотите ли вы~~заглянуть ко мне и послушать сонату для скрипки соло вашего любимого композитора, И. С. Баха? Я только что купил отличную запись.~Поразительно, как это Баху удалось, используя одну-единственную скрипку,~создать такую интересную вещь.

\emph{Ахилл} : Головная боль тоже? Бедняжка\ldots{} Пожалуй, вам лучше лечь в постель и постараться заснуть.

\emph{Ахилл} : Понятно. Овец считать уже пробовали? Где-то у меня была целая~картотека подобных трюков~--- говорят, они здорово помогают от бессоницы.

\emph{Ахилл} : Ах, да. Я отлично понимаю, что вы имеете в виду~--- я это тоже пробовал. Может быть, если уж эта задачка так застряла у вас в голове, вы~поделитесь ею со мной, чтоб и я мог попробовать свои силы?

\emph{Ахилл} : Слово, внутри которого встречаются подряд буквы «Р», «Т», «О», «Т», «Е»\ldots{} Г-м-м\ldots{} Как насчет «ретотра»?

\emph{Ахилл} : Ах, какой стыд\ldots{} Конечно вы правы~--- я опять все перепутал. К тому же в слове «реторта» эти буквы все равно идут задом наперед.

\emph{Ахилл} : Уже несколько часов? Хорошенькую вы мне задали задачку\ldots{} Где вы откопали такую дьявольскую головоломку?

\emph{Ахилл} : Вы имеете в виду, что он только делал вид, что размышляет над~эзотерическими буддистскими проблемами, когда на самом деле он пытался придумать сложные словесные головоломки?

\emph{Ахилл} : Ага! Улитка знала, чем он занимается. Как же вам удалось с ней~переговорить?

\emph{Ахилл} : Вы знаете, я как-то слышал похожую головоломку. Хотите, я вам ее задам? Или это еще хуже вас отвлечет?

\emph{Ахилл} : Согласен~--- хуже уже вряд ли будет. Так вот: какое слово начинается с «КА» и кончается на «КА»?

\emph{Ахилл} : Очень остроумно~--- но это нечестно. Я совершенно не это имел в виду!

\emph{Ахилл} : Согласен, это слово выполняет условие; но все равно это какое-то~дегенеративное решение.

\emph{Ахилл} : Абсолютно верно! Как вам удалось так быстро найти ответ?

\emph{Ахилл} : Это~--- еще один пример того, какой полезной может оказаться картотека трюков от бессоницы. Прекрасно! Но я все еще блуждаю в потемках с вашей задачкой о «PTOTE».

\emph{Ахилл} : Поздравляю~--- теперь вам, может быть, удастся заснуть. Скажите же мне решение!

\emph{Ахилл} : Вообще-то я не люблю подсказок, но на этот раз ладно, валяйте.

\emph{Ахилл} : Не понимаю. Что вы имеете в виду под «рисунком» и «фоном»?

\emph{Ахилл} : Разумеется, я знаком с «Мозаикой II». Я знаю ВСЕ работы Эшера. В конце концов, это мой любимый художник! Кстати, репродукция «Мозаики II» висит прямо у меня перед носом.

\emph{Ахилл} : Всех черных зверей? Конечно, вижу!

\emph{Ахилл} : Верно: их «негативное пространство»~--- то, что остается свободным~--- определяет белых зверей.

\emph{Ахилл} : А, так вот что вы называете «рисунком» и «фоном»! Но какое отношение это имеет к головоломке о «Р-Т-О-Т-Е»?

\emph{Ахилл} : Это для меня слишком сложно\ldots{} Теперь и у меня начинает болеть голова; пойду, пожалуй, поищу мою спасительную картотеку, может быть она мне поможет забыться сном.

\emph{Ахилл} : Вы хотите зайти сейчас? Но я думал\ldots{}

\emph{Ахилл} : Ну что ж, хорошо. А я пока постараюсь решить эту задачку с помощью вашей подсказки о рисунке и фоне и моей головоломки.

\emph{Ахилл} : С удовольствием сыграю их для вас.

\emph{Ахилл} : Вы изобрели о них теорию?

\emph{Ахилл:} В сопровождении какого инструмента?

\emph{Ахилл} : В таком случае, как странно, что он не записал также и партию~клавесина, и не опубликовал их в таком виде.

\emph{Ахилл} : А, понимаю~--- нам предоставляется выбор: слушать ее с~аккомпанементом или без оного. Но откуда мы знаем, как он должен звучать?

\emph{Ахилл} : Да, вы правы~--- наверное, лучше всего оставить эту работу воображению слушателя. Согласен~--- может быть, у Баха в мыслях вообще не было никакого аккомпанемента. Действительно, эти сонаты и так звучат~замечательно.

\emph{Ахилл} : Точно. Ну, до скорого.

\emph{Ахилл:} Пока, г-жа Ч.

\emph{Рис. 14. М К. Эшер. «Мозаика II» (литография, 1957).}


% % \subsubsection{ГЛАВА III: Рисунок и фон}
% \subsubsection{ГЛАВА III: Рисунок и фон}

Простые и составные числа

ТО, ЧТО некоторые понятия можно выразить при помощи простых~манипуляций типографскими символами, кажется довольно странным. До сих пор мы передали таким образом лишь понятие сложения, и это, возможно, не~показалось нам удивительным. Предположим, однако, что мы захотим создать~формальную систему с теоремами вида \textbf{P} \emph{x} , где~\emph{x} было бы строчкой, состоящей из тире. Количество этих тире должно было бы выражаться простым числом. Так,~\textbf{P-\/-~} ---~было бы теоремой, в то время как~\textbf{P-\/-\/-} теоремой бы не являлось. Как это может быть выражено с помощью типографских операций? Сначала~необходимо точно определить, что мы имеем в виду под «типографскими операциями». Полное описание было дано в системах~\textbf{MIU} и \textbf{pr} , так что сейчас мы~ограничимся только списком наших возможностей:

(1) читать и узнавать любое из конечных множеств символов;

(2) записывать любой из символов, принадлежащий такому множеству,

(3) повторять любой из этих символов в другом месте;

(4) стирать любой из этих символов;

(5) проверять, одинаковы ли два символа;

(6) сохранять и использовать список ранее выведенных теорем.

Список получился немного повторяющимся, но это не столь важно.~Главное то, что он позволяет только самые тривиальные операции, намного проще, чем операция отличения простого числа от не простого. Как же, в таком случае, мы сможем совместить несколько операций и создать такую формальную~систему, в которой простые числа отличались бы от составных?

Система~ur

Первым шагом может стать решение более простой, но сходной задачи. Мы можем попытаться придумать систему, похожую на систему \textbf{pr} , но которая вместо сложения представляла бы умножение. Назовем ее системой~\textbf{ur} (u = «умноженное на»). Предположим, что X, Y, Z , соответственно, --- это~количество тире в строчках x, y, z. (Обратите внимание, что я специально делаю упор на различии между строчкой, и количеством тире, которое эта строчка~содержит.) Мы хотим, чтобы строчка~x\textbf{u} y\textbf{rz} была теоремой только в том случае, когда X, умноженное на Y, равняется Z. Например,~\textbf{-\/-u-\/-\/-r-\/-\/-\/-\/-\/-} ~должно быть теоремой, так как 2, умноженное на 3, равняется 6, в то время как~\textbf{-\/-u-\/-r-\/-\/-~} теоремой быть не должно. Систему~\textbf{ur} так же просто описать, как и систему \textbf{pr} . Для этого нужны всего лишь одна аксиома и одно правило вывода

СХЕМА АКСИОМ:~\emph{x} \textbf{u-r} \emph{x} является аксиомой, когда~\emph{x} ~--- строчка, состоящая из тире.

ПРАВИЛО ВЫВОДА: Предположим, что \emph{x} , \emph{у} , и \emph{z} ~--- строчки тире, и что~\emph{x} \textbf{u} \emph{y} \textbf{r} z --- старая теорема. Тогда~\emph{x} \textbf{u} \emph{y\textbf{-}} \textbf{r} \emph{zx} будет новой теоремой.

Ниже приводится вывод теоремы~\textbf{-\/-u-\/-\/-r-\/-\/-\/-\/-\/-}

(1)~\textbf{-\/-u-r-\/-~} (аксиома)

(2)~\textbf{-\/-u-\/-r-\/-\/-\/-} ~(по правилу вывода, используя (1) в качестве старой теоремы)

(3)~\textbf{-\/-u-\/-\/-r-\/-\/-\/-\/-\/-} ~(по правилу вывода, используя (2) в качестве старой теоремы)

Обратите внимание, что количество тире в средней строке возрастает на~единицу каждый раз, когда мы применяем правило вывода, таким образом, мы можем предсказать, что если мы хотим получить теорему с десятью тире в середине, нам придется применить правило вывода девять раз подряд.

Уловление Составности

Умножение (немного более сложное понятие, чем сложение) теперь уловлено нами в сети типографских правил, подобно птицам в Эшеровском «Освобождении». А как же насчет простых чисел? Следующий план кажется неплохим: используя систему \textbf{ur} , определить новое множество теорем вида S\emph{x} , которые характеризуют \emph{составные числа}

ПРАВИЛО: Предположим, что \emph{x} , \emph{у} , \emph{z} --- строчки тире. Если~\emph{x\textbf{-}} \textbf{u} \emph{y\textbf{-}} \textbf{r} \emph{z} является теоремой, то~S\emph{z} также будет теоремой.

Это означает, что Z (число тире в z) является составным, если оно ---~произведение двух чисел, больших единицы (а именно, X+1 --- число тире в \emph{x\textbf{-}} и Y+1 --- число тире в \emph{y\textbf{-}} ). Я объясняю вам это новое правило в «интеллектуальном режиме», поскольку вы, как существо мыслящее, желаете знать, почему такое правило существует. Если бы вы работали исключительно в «механическом режиме», вам бы не понадобились никакие объяснения, так как работающие в режиме~\textbf{M} следуют правилам чисто механически, никогда не задавая вопросов, и при этом совершенно счастливы!

Поскольку вы работаете в режиме \textbf{I} , вы будете склонны забывать о~различии между строчками и их интерпретацией. Ситуация может стать довольно запутанной, как только вы обнаружите смысл в символах, которыми вы~манипулируете. Вам придется бороться с собой, чтобы не решить, что строчка «\textbf{-\/-\/-»} ~--- то же самое, что число 3. Требование формальности, казавшееся совершенно очевидным в главе I, здесь становится весьма каверзным и приобретает~первостепенную важность Именно оно не дает вам спутать режим \textbf{I} с режимом \textbf{M} , иными словами, оно не позволяет вам смешивать арифметические факты с типографскими теоремами.

«Нелегальная» характеристика простых чисел

Весьма соблазнительно от теорем типа S сразу перескочить к теоремам типа P, путем введения следующего правила

ПРЕДЛОЖЕННОЕ ПРАВИЛО: Предположим, что~\emph{x} --- строчка тире. Если S\emph{x} не является теоремой, то~P\emph{x} является теоремой.

Роковая ошибка здесь заключается в том, что проверка «нетеоремности» S\emph{x} --- не типографская операция. Чтобы узнать наверняка, что \textbf{MU} --- не теорема \textbf{MIU} , нам пришлось бы \emph{выйти из системы} ; в такую же ситуацию мы попадаем и с Предложенным Правилом. Оно подрывает сами основы формальных систем тем, что предлагает вам действовать неформально, вне системы. Типографская операция (6) позволяет вам рассматривать предварительно выведенные~теоремы; однако Предложенное Правило отсылает вас к гипотетической «таблице не-теорем». Чтобы получить подобную таблицу, вам придется работать \emph{вне~системы} , показывая, почему некоторые строчки не могут быть получены в~данной системе. Конечно, может оказаться, что существует другая формальная система, в которой «таблица не-теорем» может быть получена чисто~типографскими способами. На самом деле, наша цель --- найти именно такую систему.Однако Предложенное Правило --- не типографское, а посему нам придется от него отказаться.

Это настолько важный момент, что мы остановимся на нем поподробнее. В нашей \emph{системе S} (включающей систему~\textbf{ur} и правила, определяющие теоремы типа S) у нас есть теоремы вида S\emph{x} , где \emph{x} , как обычно, обозначает строчку тире. В ней имеются также не-теоремы вида \emph{S} x. Говоря о не-теоремах, я имею в виду именно эту разновидность, хотя, конечно, существует множество не-теорем в виде неправильно сформированных строчек:~\textbf{u u-S r r} ~и пр. Между теоремами и не-теоремами есть следующая разница: количество тире в первых ---~составное число, во вторых --- простое. К тому же, все теоремы похожи по форме, так как все они выведены при помощи одного и того же набора типографских правил. Можем ли мы сказать, что в этом смысле все не-теоремы также имеют что-то общее в форме? Ниже приводится список теорем типа S, без их вывода. Число в скобках указывает на количество тире в соответствующей теореме.

S\textbf{-\/-\/-\/-} ~(4)

S\textbf{-\/-\/-\/-\/-\/-} ~(6)

S\textbf{-\/-\/-\/-\/-\/-\/-\/-} ~(8)

S\textbf{-\/-\/-\/-\/-\/-\/-\/-\/-} ~(9)

S\textbf{-\/-\/-\/-\/-\/-\/-\/-\/-\/-} ~(10)

S\textbf{-\/-\/-\/-\/-\/-\/-\/-\/-\/-\/-\/-} (12)

S\textbf{-\/-\/-\/-\/-\/-\/-\/-\/-\/-\/-\/-\/-\/-} (14)

S\textbf{-\/-\/-\/-\/-\/-\/-\/-\/-\/-\/-\/-\/-\/-\/-} ~(15)

S\textbf{-\/-\/-\/-\/-\/-\/-\/-\/-\/-\/-\/-\/-\/-\/-\/-} ~(16)

S\textbf{-\/-\/-\/-\/-\/-\/-\/-\/-\/-\/-\/-\/-\/-\/-\/-\/-\/-} ~(18)

.

.

.

«Дырки» в этом списке как раз и являются не-теоремами. Есть ли у них какая-то общая «форма»? Можно ли предположить, что лишь постольку,~поскольку они являются пробелами в неком упорядоченном списке, они обладают какими-то общими чертами? И да, и нет. Нельзя отрицать, что у них есть общие типографские черты; вопрос в том, правомочно ли называть эти черты «формой». Дело в том, что дырки определены только негативно: они представляют из себя то, что осталось от позитивно определенного списка.

~Рисунок и фон

Это напоминает известное разграничение между рисунком и фоном в~живописи. Когда предмет или «положительное пространство» (например, человеческая фигура, буква или натюрморт) рисуется внутри рамки, неизбежным следствием этого является появление на картине дополняющей формы, также~называющейся «фоном», или «негативным пространством». В большинстве картин~отношение между фоном и рисунком почти не играет роли; как правило, художник в основном занят рисунком. Однако иногда его внимание привлекает также и фон.

Существуют замечательные шрифты, обыгрывающие это различие между рисунком и фоном. Послание, написанное таким шрифтом, приводится ниже. На первый взгляд это просто несколько клякс; но если вы посмотрите на них издали, попристальнее, то увидите семь букв на этом РИСУНКЕ (специальным шрифтом, так, что черный фон, создающий белые буквы, похож на кляксы.)

\emph{Рис. 15. Рисунок}

Такой же эффект производит мой рисунок «Знак из дыма» (рис. 139).~Продолжая в том же ключе, попробуйте решить следующую задачку: возможно ли нарисовать такую картину, чтобы слова были как на рисунке, так и в фоне?

Давайте условимся различать между двумя типами рисунков: \emph{курсивно рисуемыми} и \emph{рекурсивными} (эти термины не являются~общеупотребительными~--- их придумал я сам). В \emph{курсивно рисуемом} рисунке фон является лишь побочным продуктом. В \emph{рекурсивном} рисунке, наоборот, фон может~рассматриваться как отдельный самостоятельный рисунок. Обычно художник делает это вполне сознательно. Приставка «ре» здесь выражает тот факт, что как~рисунок, гак и фон могут быть нарисованы курсивно, то есть, такая картина «дву-курсивна». Любой контур на рекурсивном рисунке --- это обоюдоострый меч. М. К. Эшер был мастером подобных картин; взгляните, например, на его~великолепную рекурсивную гравюру «Птицы» (рис. 16).

\emph{Рис. 16. M. K. Эшер. «Деление пространства при помощи птиц» (из блокнота 1942 года).}

Различие здесь не такое строгое, как в математике; кто может с~уверенностью утверждать, что некий фон не является в то же время и рисунком? При достаточно внимательном рассмотрении, любой фон не лишен собственного интереса. В этом смысле любой рисунок можно назвать рекурсивным. Однако, вводя эти термины, я имел в виду нечто другое. Существует естественное,~интуитивное понятие узнаваемых форм. Являются ли и рисунок и фон узнаваемыми формами? Если да, то такой рисунок рекурсивен. Посмотрев на фон~большинства контурных рисунков, вы обнаружите, что в нем трудно признать какую-либо форму. Это доказывает, что:

Существуют узнаваемые формы, чье негативное пространство не является никакой узнаваемой формой. Или, выражаясь более технично:

Существуют курсивно рисуемые рисунки, которые не рекурсивны.

\emph{Рис. 17. Скотт Е. Ким Рисунок «РИСУНОК-РИСУНОК».}

На рис. 17 показано решение предложенной выше головоломки,~принадлежащее Скотту Киму; я называю это решение «рисунок РИСУНОК --- РИСУНОК». На какую бы часть --- белую или черную --- вы не посмотрели, вы увидите только «ФИГУРЕ» (= английское «РИСУНОК»), и никакого «ФОНА». Великолепный образчик рекурсивного рисунка! Черные области этого~хитроумного рисунка можно охарактеризовать двумя способами:

(1) как \emph{негативное пространство} белых областей;

(2) как \emph{видоизмененные копии} белых областей (полученные путем их окраски и сдвига каждой белой области).

(В данном случае обе характеристики эквивалентны; для большинства~черно-белых рисунков это не так.) В главе VIII, создавая Типографскую Теорию Чисел (ТТЧ), мы будем надеяться, что нам удастся охарактеризовать~множество всех ложных утверждений аналогичными способами:

(1) как \emph{негативное пространство} множества всех теорем ТТЧ;

(2) как \emph{модифицированные копии} множества всех теорем ТТЧ (полученные путем отрицания каждой теоремы ТТЧ).

Однако этой надежда окажется напрасной, так как:

(1) среди множества всех не-теорем существуют некоторые истинные~утверждения;

(2) вне множества всех отрицаний теорем, существуют некоторые ложные утверждения.

Отчего так получается, вы увидите в главе XIV; а пока можете поразмыслить над графическим изображением данной ситуации (Рис. 18).

\emph{Рис. 18. Эта диаграмма отношений между различными классами строчек ТТЧ весьма богата зрительным символизмом. Самый большой прямоугольник --- множество всех строчек ТТЧ. Следующий прямоугольник --- все правильно построенные строчки ТТЧ. Внутри него находится множество всех предложений ТТЧ. Именно на этом уровне начинают происходить интересные вещи. Множество теорем изображено в виде~дерева, чей ствол --- множество аксиом. Символ дерева был выбран из-за того, что оно растет «рекурсивно» новые ветви (теоремы) вырастают из старых. Пальцеобразные ветви проникают во все уголки области представляющей множество истинных~высказываний, однако они не могут занять эту область целиком. Граница между областями истинных и ложных высказываний представляет собой изломанную «береговую линию», которая, как бы близко вы ее не рассматривали, всегда имеет еще более мелкие уровни структуры и таким образом, не поддается описанию каким либо конечным методом (См. книгу Мандельбродта «Фракталы» (В. Mandelbrodt Fractals)). Отраженное дерево справа представляет отрицания теорем все они ложны, но вкупе они не в состоянии заполнить всю область ложных высказываний (Рисунок автора)}

Рисунок и фон в музыке

Аналогию с понятием рисунка и фона можно также найти и в музыке. Примером может служить различие между мелодией и аккомпанементом: мелодия всегда на первом плане, тогда как аккомпанемент в каком-то смысле второстепенен. Поэтому нам кажется удивительным, когда мы узнаем мелодии на «низшем» уровне музыкального произведения. Для пост-барочной музыки это редкое~явление --- обычно гармонии там не выходят на первый план. Но в барочной музыке --- и прежде всего, у Баха --- все уровни «работают» в качестве «рисунка». В этом смысле баховские композиции могут быть названы рекурсивными.

В музыке есть еще одно различие между рисунком и фоном --- ударные и безударные такты. Если вы начнете отмечать ритм счетом «раз-и, два-и, три-и\ldots», большинство нот мелодии придутся на числа, а не на «и». Иногда, однако, мелодия бывает нарочно смещена на «и», чем достигается интересный эффект. Это происходит, например, в нескольких фортепианных этюдах Шопена. Тот же прием можно найти у Баха, в особенности, в сонатах и партитурах для скрипки соло и в сюитах для виолончели соло. В этих композициях Баху удается~поместить несколько мелодий одновременно на разных уровнях. Иногда он~достигает этого эффекта, заставляя солирующий инструмент играть дублировки --- две ноты сразу. В других случаях, однако, он помещает один голос на ударные такты, а другой --- на безударные, так что слух различает две разные мелодии, вплетающиеся одну в другую и гармонически сочетающиеся. Нет нужды~говорить, что Бах не останавливается на этом уровне сложности\ldots{}

Рекурсивно счетные и рекурсивные множества

Перенесем понятие рисунка и фона обратно в область формальных систем. В нашем примере роль позитивного пространства играют теоремы типа S, а роль негативного пространства --- строчки, в которых количество тире выражается простым числом. Пока что единственный способ, который нам удалось найти~~~для выражения простых чисел типографским путем, это негативное пространство. Существует ли какой нибудь способ выразить простые числа в виде позитивного пространства, то есть в виде множества теорем некой системы?

Интуиция подсказывает разным людям разные ответы. Я отчетливо помню, как был озадачен и заинтригован, заметив разницу между негативной и позитивной характеристиками. Я был совершенно уверен в том, что не только простые числа, но и вообще любое негативно определяемое множество чисел может быть определено позитивно. Интуитивное обоснование моей уверенности заключалось в следующем вопросе: «\emph{Как это возможно, чтобы рисунок и фон не содержали совершенно одинаковой информации} ?» Мне казалось, что они представляют собой одну и ту же информацию, закодированную двумя разными способами. А что думаете по этому поводу вы, читатель?

Выяснилось, что я был прав насчет простых чисел, но ошибался в остальном. Тогда это меня поразило и продолжает поражать и по сей день. Оказывается, что:

\emph{существуют такие формальные системы, чье негативное пространство (множество не-теорем) не является позитивным пространством никакой другой формальной системы} .

Как выяснилось, этот результат сравним по глубине с Теоремой Гёделя --- так что неудивительно, что моя интуиция не могла принять его сразу. Подобно математикам начала двадцатого века, я считал мир формальных систем и натуральных чисел более предсказуемым, чем он оказался в действительности. Выраженное более техническим языком, это утверждение звучит так:

\emph{Существуют рекурсивно счетные множества, не являющиеся рекурсивными} .

Выражение «рекурсивно счетные» (часто сокращаемое как р.с.) --- математическое соответствие нашему художественному понятию «курсивно рисуемые», а \emph{рекурсивный} --- соответствие «рекурсивным». Множество строчек является р. с., когда все они могут быть выведены путем применения типографских правил --- например, множество теорем типа S или множество теорем системы \textbf{MIU} ; на самом деле, это определение приложимо ко множеству теорем любой формальной системы. Оно сравнимо с понятием о «рисунке» как о «множестве линий, которые могут быть произведены в соответствии с художественными правилами» (что бы это последнее не означало!). А «рекурсивное множество» подобно рисунку, чей фон, в свою очередь, также является рисунком --- в таком случае не только рисунок, но и его дополнение будут р. с. Из этого вытекает следующий результат:

\emph{Существуют такие формальные системы, у которых нет типографского алгоритма разрешения} .

Из чего это следует? Очень просто. Типографский алгоритм разрешения --- это метод, отличающий теоремы от не-теорем. Он позволяет нам выводить не-теоремы систематически, идя по списку \emph{всех} строчек и отбрасывая те, что не являются теоремами. Эту процедуру можно назвать типографским методом вывода множества не-теорем. Однако из предыдущего утверждения (которое мы пока принимаем на веру) следует, что для \emph{некоторых} формальных систем это невозможно.

Предположим, что мы нашли множество R («R» --- рисунок) натуральных чисел, которое мы можем вывести каким-либо формальным путем --- вроде множества составных чисел. Предположим, что его дополнением является множество F («F» --- фон) --- простые числа. Вместе взятые, R и F дают все натуральные числа. Мы знаем правило, позволяющее вывести все числа множества R, для чисел множества F такого правила не существует. Важно, что если числа R выводятся исключительно \emph{в возрастающем порядке} , то мы всегда можем охарактеризовать F. Трудность заключается в том, что многие р. с. множества производятся при помощи таких методов, которые выводят элементы в произвольном порядке, так что не известно, появится ли какое-либо число, до сих пор пропускаемое, если подождать еще чуть-чуть.

На вопрос «Все ли рисунки рекурсивны?» мы ответили отрицательно. Теперь мы видим что придется ответить отрицательно и на аналогичный вопрос математиков «Все ли множества рекурсивны?» Имея это в виду, давайте вернемся к этому расплывчатому понятию «формы». Обратимся снова к нашим множествам R --- рисунки и F --- фон. Легко согласиться с тем, что все числа во множестве R имеют какую-то общую «форму» --- но можно ли сказать то же самое о числах множества F? Странный вопрос. С самого начала имея дело с бесконечным множеством всех натуральных чисел, весьма сложно прямо и четко определить «дырки», остающиеся в списке после изъятия оттуда неких чисел. Таким образом, возможно что на самом деле у этих дырок нет никаких общих характеристик «формы». Неясно, стоит ли вообще использовать здесь такое соблазнительное словечко как «форма». Может быть лучше не определять этого понятия оставив ему некую интуитивную гибкость.

Вот вам еще одна головоломка, над которой вы можете поразмыслить в связи с изложенным выше Можете ли вы охарактеризовать следующее множество чисел (или его негативное пространство)?

1 3 7 12 18 26 35 45 56 69

Чем данная последовательность напоминает рисунок РИСУНОК-РИСУНОК?

Простые числа в качестве рисунка, а не фона

Как же насчет формальной системы для вывода простых чисел? Как это~сделать? Способ состоит в том чтобы, не останавливаясь на умножении,~обратиться прямо к неделимости, представив ее позитивно. Ниже дана схема аксиом и правило вывода теорем, представляющих понятие числа, не являющегося~делителем других чисел (\textbf{ND} = не делитель).

СХЕМА АКСИОМ:~~\emph{xy} \textbf{ND} \emph{x} , где~\emph{x} и \emph{у} --- строчки тире

Например,~\textbf{-\/-\/-\/-\/-ND-\/-} , где~\emph{x} заменен на «\textbf{-\/-»} и \emph{y} --- на~«\textbf{-\/-\/-»}

ПРАВИЛО: Если~\emph{x} \textbf{ND} \emph{y} является теорема, то~\emph{x} \textbf{ND} \emph{xу} также будет теоремой

Приложив это правило дважды, вы можете вывести теорему

\textbf{-\/-\/-\/-\/-ND-\/-\/-\/-\/-\/-\/-\/-\/-\/-\/-\/-}

которая интерпретируется как~«5 не делитель 12». Однако~\textbf{-\/-\/-ND-\/-\/-\/-\/-\/-} ~не является теоремой. В чем будет ошибка, если вы попытаетесь вывести эту строчку?

Чтобы определить, что данное число простое, у нас должны быть какие-то сведения о его свойствах неделимости. В частности, мы хотим знать, что это число не делится на 2, 3, 4, и т. д., до числа, меньшего его на единицу. Однако в формальных системах мы не можем позволить себе таких расплывчатых~формулировок как «и так далее». Здесь нужна исчерпывающая точность. Нам бы хотелось иметь возможность сказать на языке системы: «число Z \emph{свободно от делителей} до X» (\textbf{SOD} ~= свободно от делителей), имея в виду, что не одно число между 2 и X не является делителем Z. Это можно сделать, но здесь есть небольшой трюк. Если хотите, можете попытаться найти его.

Решение заключается в следующем:

ПРАВИЛО: Если~\textbf{-\/-ND} \emph{z} является теоремой, то \emph{z} \textbf{SOD-\/-} ~также будет~теоремой.

ПРАВИЛО: Если \emph{z} \textbf{SOD} \emph{x} и \emph{x\textbf{-}} \textbf{ND} \emph{z} являются теоремами, то \emph{z} \textbf{SOD} \emph{x} также будет теоремой.

Эти два правила, в совокупности, характеризуют понятие свободы от~делителей. Все что нам нужно, это указать, что простые числа --- это числа, \emph{свободные от делителей} , включая число на единицу меньшее их самих:

ПРАВИЛО: Если \emph{z\textbf{-}} \textbf{SOD} \emph{z} является теоремой, то \textbf{P} \emph{z\textbf{-}} также будет теоремой.

Не будем забывать, что 2 --- тоже простое число!

АКСИОМА: \textbf{P-\/-}

Вот и все, что нам необходимо. Принцип формального выражения «просто-численности» заключается в том, что существует метод проверки, не~требующий никакого отступления назад. Вы всегда двигаетесь вперед, проверяя~данное число на делимость --- сначала на 2, потом на 3, и так далее. Именно эта «монотонность» или однонаправленность --- отсутствие игры между~укорачивающими и удлиняющими правилами --- позволила нам уловить суть простых чисел. И именно этой потенциальной сложностью формальных систем,~могущих вместить сколько угодно взаимодействий между движением вперед и~назад, объясняются такие ограничивающие результаты как Теорема Гёделя и~Проблема Остановки Тюринга, как и тот факт, что не все рекурсивно счетные множества рекурсивны.


% % \subsubsection{Акростиконтрапунктус}
% \include{fragments/xx15}
% % \subsubsection{ГЛАВА IV: Непротиворечивость, полнота  геометрия}
% \include{fragments/xx16}
% % \subsubsection{Маленький гармонический лабиринт}
% \subsubsection{Маленький гармонический лабиринт}

\emph{Черепаха и Ахилл проводят день в Кони Айленде, огромном парке аттракционов. Купив себе по палочке «сахарной ваты», они решают прокатиться на колесе обозрения.}

Черепаха: Это мой любимый аттракцион. Кажется, что едешь так далеко --- а на самом деле никуда не попадаешь!

Ахилл: Понятно, почему это вам так нравится. Вы уже пристегнулись?

Черепаха: Да, все ремни на месте. Поехали! Ур-ра!

Ахилл: Я вижу, вы сегодня предовольны.

Черепаха: И не без основания: моя тетушка-гадалка предсказала мне на сегодня необыкновенную удачу. Так что я вся трепещу в предвкушении.

Ахилл: Неужели вы верите в предсказания судьбы?

Черепаха: Вообще-то нет\ldots{} но говорят, что они действуют, даже когда в них не веришь.

Ахилл: Ну, в таком случае, вам действительно повезло.

Черепаха: Ах, какой вид! Пляж, толпа, океан, город\ldots{}

Ахилл: И правда, великолепно. Взгляните-ка на вертолет --- вон там. Кажется, он летит в нашем направлении. На самом деле, он уже почти над нами.

Черепаха: Странно, оттуда свисает какая-то веревка\ldots{} и она совсем близко к нам --- можно ухватиться\ldots{}

Ахилл: Смотрите-ка: на конце веревки огромный крюк и на нем --- записка.

\emph{(Он протягивает руку и срывает записку. Колесо начинает опускаться.)}

Черепаха: Ну как, что там написано? Можете разобрать?

Ахилл: Да\ldots{} Здесь написано: «Приветик, друзья. Будете снова наверху --- хватайтесь за крюк, и получите Сюрприз!»

Черепаха: Записка грубовата\ldots{} но кто знает, к чему это может привести. Может, это начинается обещанное везенье. Давайте попробуем!

Ахилл: Давайте!

\emph{(Когда колесо снова начинает подниматься, они расстегивают свои ремни и на самой высокой точке хватаются за гигантский крюк. Внезапно веревка взлетает вверх, унося их к зависшему над их головами вертолету. Большая сильная рука втаскивает их внутрь.)}

Голос: Добро пожаловать на борт, лопухи!

Ахилл: К-кто\ldots{} кто вы такой?

Голос: Позвольте представиться: Гексахлорофен Ж. Удача, Знаменитый Похититель Детишек и Пожиратель Черепах --- к вашим услугам!

Черепаха: Ой!

Ахилл (шепотом Черепахе): Вот так «Удача»! Не совсем то, на что мы надеялись\ldots{} (Удаче): Гм-м-м\ldots{} если я могу позволить себе смелость спросить куда вы нас везете?

Удача: Хо-хо! На мою небесную кухню с полным электрическим оборудованием, где я собираюсь приготовить вот этот лакомый кусочек (бросая плотоядный взгляд на Черепаху) --- райский супчик получится, пальчики оближешь! И не сомневайтесь --- я проделываю все это исключительно в усладу моему чревоугодию! Хо-хо-хо!

Ахилл: На это я могу сказать лишь то, что смех у вас довольно злодейский.

Удача (злодейски смеясь): Хо-хо-хо! За эти слова, мои дорогой друг, ты мне дорого заплатишь! Хо-хо!

Ахилл: Ах, господи! Интересно, что он имеет в виду?

Удача: Все очень просто: у меня для вас обоих уготовлена Ужасная Судьба! Погодите --- вы у меня попляшете! Хо-хо- хо! Хо-хо-хо!

Ахилл: Ой, мамочка!\ldots{}

Удача: Вот мы и приехали. Высаживайтесь, друзья, прямо в мою электрическую небесную кухню. (Они заходят внутрь.) Располагайтесь и чувствуйте себя как дома, пока я буду решать вашу судьбу. Вот моя спальня. Вот мой кабинет. Присаживайтесь и подождите меня --- я ненадолго, только ножи наточу. Можете пока попробовать мои вина. Мое последнее приобретение --- «Витаскин»; там что-то еще на этикетке понаписано, да только я языка не понимаю, так что я называю эту штуку просто: «Вытаскин». Вон та бутылочка, на лосьон смахивает\ldots{} Я его еще сам не пробовал. Ну, я пошел. Хо-хо-хо! Черепаший супчик! Черепаший супчик! Мое любимое блюдо! (Уходит.)

Ахилл: Вытаскин! Давайте напьемся с горя!

Черепаха: Ахилл! Вы же уже выпили две кружки пива в парке! Да и как вы можете думать об этом в такой момент, именно когда нам необходима ясная голова?

Ахилл: А мне до лампочки\ldots{} (Поет.) Шуме-ел камы-ы-ыш\ldots{} о, миль пардон, я не должен петь подобных песен в присутствии дамы, да еще в такую ужасную минуту.

Черепаха: Боюсь, что наша песенка так и так спета.

Ахилл: Это еще бабушка надвое сказала. Давайте пока от нечего делать посмотрим, что за книги у нашего хозяина на полках. Ну и коллекция, только для посвященных: «Садовые головы, с которыми я был знаком», «Шахматы и верчение зонтиков --- без труда», «Концерт для чечеточника и оркестра»\ldots{} Гм-м-м.

Черепаха: Что это за открытая книжица лежит там на столе, рядом с додекаэдром и альбомом для рисования?

Ахилл: Эта? Она называется: «Занимательные приключения Ахилла и Черепахи или Вокруг света от кочки до кочки.»

Черепаха: Довольно занимательное название.

Ахилл: Действительно --- и приключение, на котором книга открыта, выглядит занимательно. Оно называется «Джинн и Настойка».

Черепаха: Гм-м-м\ldots{} Интересно, почему. Может, попробуем почитать? Я буду читать за Черепаху, а вы --- за Ахилла.

Ахилл: Согласен. Терять нам все равно нечего\ldots{}

\emph{(Они начинают читать «Джинна и настойку».)}

\emph{(Ахилл пригласил Черепаху в гости, посмотреть коллекцию гравюр его любимого художника, Эшера.)}

\emph{Черепаха} : Чудесные гравюры, Ахилл.

\emph{Ахилл} : Я так и знал, что вам понравится. Какая ваша любимая гравюра?

\emph{Черепаха} : Одна из моих любимых --- «Выпуклое и вогнутое», где совмещаются два внутренне непротиворечивых мира. В результате получается составной, абсолютно невозможный мир. Противоречивые миры всегда забавно посетить, но жить там мне бы не хотелось.

\emph{Ахилл} : «Забавно посетить?» Что вы имеете в виду? Как можно посетить противоречивые миры, если их вообще НЕ СУЩЕСТВУЕТ?

\emph{Черепаха} : Прошу прощения --- но разве мы только что не согласились, что на этой картине Эшера изображен противоречивый мир?

\emph{Ахилл} : Да, но это же двухмерный мир, фикция, картинка. Этот мир посетить не удастся.

\emph{Черепаха} : У меня есть свои способы\ldots{}

\emph{Ахилл} : Как же вам удается затолкать себя в плоский мир картины?

\emph{Черепаха} : Для этого надо выпить стаканчик ПРОТАЛКИВАЮЩЕГО ЗЕЛЬЯ.

\emph{Ахилл} : Что это за штука такая --- проталкивающее зелье?

\emph{Черепаха} : Это жидкость, обычно содержащаяся в маленьких керамических пузырьках; когда вы, глядя на картину, выпиваете немного, жидкость эта проталкивает вас прямо в мир картины. Люди, которые ничего не знают о свойствах проталкивающего зелья, часто бывают поражены тем, в какие ситуации они попадают.

\emph{Ахилл} : А как насчет противоядия? Когда человек таким образом оказывается протолкнутым в картину, он что, так и остается там на всю жизнь?

\emph{Черепаха} : Иногда это не такое уж большое несчастье\ldots{} Но, разумеется, имеется другое зелье --- на самом деле, это скорее что-то вроде бальзама\ldots{} или эликсира\ldots{}

Черепаха: Она, кажется, имеет в виду «настойку».

\emph{Ахилл} : Настойка?

\emph{Черепаха} : Точно, именно это я и имела в виду! ВЫТАЛКИВАЮЩАЯ НАСТОЙКА, так она и называется. Если вы держите ее в правой руке, когда глотаете проталкивающее зелье, то она тоже оказывается протолкнутой в картину вместе с вами. Как только вы возжаждете быть вытолкнутым обратно в реальный мир, отхлебните немного выталкивающей настойки и --- але-оп! --- вы в реальном мире, точно на том же месте, где вы были, когда отведали проталкивающего зелья.

\emph{Ахилл} : Все это звучит захватывающе интересно. А что получится, если принять выталкивающую настойку, не протолкнувшись предварительно в картину?

\emph{Черепаха} : Я точно не знаю, Ахилл, но я бы не стала играть с этими странными жидкостями. Когда-то у меня был друг Фома, который мне не поверил и решил сделать именно это --- и с тех пор никто о нем ничего не слыхал.

\emph{Ахилл} : Жаль. А можно ли взять с собой бутылочку проталкивающего зелья?

\emph{Черепаха} : О, конечно. Надо зажать ее в левой руке и она тоже оказывается протолкнутой в картину вместе с вами.

Ахилл: А если внутри этой картины окажется еще одна, и вы снова примете глоточек проталкивающего зелья?

Черепаха: Случится именно то, чего вы ожидаете: вы очутитесь внутри картины-в-картине.

Ахилл: И, наверное, тогда придется выталкиваться дважды, чтобы вытащить себя из вписанных друг в друга картин и вновь вернуться в реальную жизнь.

Черепаха: Совершенно верно. На каждое проталкивание приходится одно выталкивание, так как первое вводит вас в картину, а второе это действие отменяет.

Ахилл: Знаете, все это звучит подозрительно. Вы уверены, что вы говорите это не только с целью испытать пределы моей доверчивости?

Черепаха: Клянусь! Поглядите: вот тут, в кармане, у меня два пузырька. (Засовывает руку в жилетный карман и вытаскивает два довольно больших пузырька без этикетки; слышно, как в них булькает жидкость, в одном красная, в другом --- голубая.) Ежели желаете, можем попробовать!

Ахилл: Э-э-э\ldots{} ну ладно\ldots{} может быть\ldots{}

Черепаха: Ну и славно! Я так и думала, что вам захочется попробовать. Хотите протолкнуться в мир Эшеровского «Выпуклого и вогнутого?»

Ахилл: Ну, как вам сказать\ldots{}

Черепаха: Значит, решено. Не забыть захватить с собой бутылочку настойки, чтобы мы смогли вытолкнуться обратно. Возьмете на себя эту ответственность, Ахилл?

Ахилл: Знаете, я немного нервничаю, и, если вы не возражаете, я предпочел бы, чтобы вы, с вашим опытом, управляли бы этой операцией.

Черепаха: Отлично. Итак\ldots{}

\emph{(С этими словами Черепаха наливает две маленькие порции проталкивающего зелья, протягивает Ахиллу его стакан и зажимает в правой лапе пузырек с настойкой Оба подносят стаканы к губам.)}

Черепаха: Пей до дна!

\emph{(Они делают по глотку.)}

\emph{Ахилл} : Что за странный привкус!

\emph{Черепаха} : К нему постепенно привыкаешь.

\emph{Ахилл} : А у настойки такой же странный вкус?

\emph{Черепаха} : Что вы, никакого сравнения! После первого же глотка вы чувствуйте такое удовлетворение, будто вы всю жизнь только о ней и мечтали.

\emph{Ахилл} : Прямо не терпится попробовать!

\emph{Черепаха} : Ну, Ахилл, где мы находимся?

\emph{Ахилл (оглядываясь)} : Мы в маленькой гондоле, скользим вниз по каналу! Я хочу сойти на берег. Синьор гондольер, остановите здесь, пожалуйста!

\emph{Рис. 23. М. К. Эшер «Выпуклое и вогнутое» (литография, 1955)}

~\emph{(Гондольер не обращает на эту просьбу ни малейшего внимания)}

~\emph{\textbf{Черепаха}} : Он не понимает по-русски. Придется нам выпрыгивать на берег, пока гондола не вошла в этот ужасный «Туннель любви», прямо перед нами.

~\emph{(Ахилл, слегка побледнев, выпрыгивает из гондолы с быстротой молнии и вытаскивает свою более медлительную спутницу.)}

\emph{\textbf{Ахилл}} : Что-то мне в этом названии определенно не по вкусу. Я очень рад что нам удалось вовремя вылезти. Послушайте, а откуда вы так хорошо знаете эти места? Вы здесь уже бывали раньше?

\emph{\textbf{Черепаха}} : Много раз, но я всегда попадала сюда из других картин Эшера. Знаете ли, позади рам они все соединены. Войдя в одну из картин, можно оттуда попасть в любую другую.

~\emph{\textbf{Ахилл}} : Удивительно! Если бы я не видел всего этого своими глазами, я бы ни за что в это не поверил. (Они выходят наружу сквозь небольшую арку.) Ой, что это там за смешная парочка ящериц?

\emph{\textbf{Черепаха}} : Смешные? Никакие они не смешные --- я вся дрожу при одной мысли о них! Это же злобные стражи волшебной медной лампы. Вон она, висит на потолке. Одно прикосновение языка, и любой смертный превращается в огурчик для закуски!

\emph{\textbf{Ахилл}} : Соленый или маринованный?

\emph{\textbf{Черепаха}} : Маринованный.

\emph{\textbf{Ахилл}} : Какая горькая судьба! Все-таки, если лампа действительно волшебная, я, пожалуй, рискну\ldots{}

\emph{\textbf{Черепаха}} : Это чистое безумие, мой друг. Я бы на вашем месте не стала этого делать.

\emph{\textbf{Ахилл}} : Всего один разочек\ldots{}

\emph{(Крадется к лампе, стараясь не разбудить спящую поблизости ящерицу. Внезапно нога его попадает в странную выемку в форме ракушки --- Ахилл скользит и взлетает в воздух. Судорожно пытаясь за что-то уцепиться, он нащупывает лампу и хватается за нее одной рукой. Лампа раскачивается. Ахилл беспомощно болтается в воздухе, а взбешенные ящерицы шипят и высовывают языки, пытаясь до него достать.)}

\emph{\textbf{Ахилл}} : На по-о-о-мощь!

\emph{(Его крик привлекает внимание стоящей поблизости женщины --- та сбегает с лестницы и будит спящего внизу мальчишку. Оценив ситуацию, он ободряюще улыбается Ахиллу и жестами показывает ему, что все будет в порядке. На странном гортанном наречии мальчишка кричит что-то двум трубачам, глядящим из окон. Они тут же начинают играть. Чудные мелодии сплетаются друг с другом, в необычном ритмическом узоре. Сонный паренек кивает в сторону ящериц, и Ахилл видит, что музыка действует усыпляюще и на них. Вскоре они вновь замирают. Тогда услужливый мальчишка зовет двух товарищей, взбирающихся по лестницам. Они составляют из лестниц что-то вроде моста. Повинуясь их настойчивым приглашающим жестам, Ахилл хватается за перекладины --- но прежде он осторожно разгибает верхнее звено цепи, на которой висит лампа, и снимает ее. Потом он взбирается на лестничный мост и мальчики вытаскивают его на безопасное место. Благодарный воин поочередно обнимает каждого из них.)}

\emph{\textbf{Ахилл}} : Г-жа Черепаха, как мне их отблагодарить?

\emph{\textbf{Черепаха}} : Я слыхала, что эти смельчаки неравнодушны к кофе --- а там внизу, в городе, есть местечко, где подают несравненный кофе-экспресс. Пригласите-ка их на чашечку!

\emph{\textbf{Ахилл}} : Это то что надо!

\emph{(С помощью комической серии жестов, улыбок и слов, Ахиллу удается растолоковать паренькам, что он их приглашает. Компания спускается по крутой лестнице в город. Они подходят к небольшому уютному кафе, усаживаются за один из столиков на улице, и заказывают пять чашечек экспресса. Пока друзья попивают кофе, Ахилл внезапно вспоминает про свою волшебную лампу.)}

\emph{\textbf{Ахилл}} : Чуть не забыл, г-жа Черепаха, лампа-то здесь! А что же в ней такого магического?

\emph{\textbf{Черепаха}} : Да как обычно --- джинн.

\emph{\textbf{Ахилл}} : Что? Вы имеете в виду, что стоит ее потереть, появится джинн и исполнит все ваши желания?

\emph{\textbf{Черепаха}} : Именно. А вы чего ожидали? Манны небесной?

\emph{\textbf{Ахилл}} : Да это же просто фантастика! Любое желание, а? Я всегда мечтал о чем-нибудь подобном\ldots{}

\emph{(Ахилл начинает тихонько тереть большую букву Л, выгравированную на медном боку лампы. Внезапно из лампы вырывается клуб дыма, в котором пятеро друзей различают очертания огромной призрачной фигуры, похожей на башню.)}

\emph{\textbf{Ахилл}} : Джинн!

\emph{\textbf{Черепаха}} : Дух!

\emph{\textbf{Фигура}} : Можно звать просто Гением\ldots{} Приветствую вас, о высокочтимые друзья, и благодарю за спасение моей Лампы от злобной Ящеричной Парочки. (С этими словами Гений подбирает Лампу и сует ее в карман, спрятанный в складках его длинного призрачного одеяния, струящегося из Лампы.) В благодарность за ваш героический поступок, я хотел бы предложить вам, от лица моей Лампы, осуществить три ваших желания.

\emph{\textbf{Ахилл}} : Потрясающе! Как вы думаете, г-жа Ч.?

\emph{\textbf{Черепаха}} : Безусловно. Что ж, друг мой, говорите ваше первое желание.

\emph{\textbf{Ахилл}} : Ух ты!.. Чего же мне пожелать? А, знаю: это пришло мне в голову еще когда я в первый раз читал «Тысячу и одну ночь» --- эти немудреные сказочки, вставлены одна в другую наподобие матрешки. Я хочу иметь не три, а СТО желаний? Здорово, правда, г-жа Ч.? Никогда не~~понимал, почему эти балбесы в сказках не догадываются попросить то же самое?

\emph{\textbf{Черепаха}} : Может быть, сейчас вы поймете.

\emph{\textbf{Гений}} : Мне очень жаль, Ахилл, но я не исполняю мета-желаний.

\emph{\textbf{Ахилл}} : Мне бы хотелось знать, что такое мета-желание\ldots{}

\emph{\textbf{Гений}} : Но это уже мета-мета-желание, Ахилл, а их я тоже не могу исполнить.

\emph{\textbf{Ахилл}} : Что-о? Ничего не понимаю\ldots{}

\emph{\textbf{Черепаха}} : Почему бы вам не выразить вашу просьбу как-нибудь по-другому?

\emph{\textbf{Ахилл}} : Что вы имеете в виду? Почему по-другому?

\emph{\textbf{Черепаха}} : Дело в том, что вы начинаете со слов «Мне бы хотелось\ldots» Но, поскольку вы хотите получить информацию, почему бы вам просто не задать вопрос?

\emph{\textbf{Ахилл}} : Ну хорошо, хотя я не совсем понимаю\ldots{} Скажите, пожалуйста, мистер Гений, что такое мета-желание?

\emph{\textbf{Гений}} : Это всего-навсего желание о желаниях. У меня нет права исполнять мета-желания. В моей власти только самые обыкновенные желания; ящик пива, скатерть-самобранка, готовая на все красотка, миллион долларов\ldots{} Понимаете, что-нибудь простенькое. Но мета-желание --- не могу. БОГ не велит.

\emph{\textbf{Ахилл}} : БОГ? Кто такой БОГ? И почему он не велит вам исполнять мета-желания? Это кажется совсем легко по сравнению с желаниями, о которых вы только что упомянули.

\emph{\textbf{Гений}} : Как вам сказать\ldots{} На самом деле, это довольно сложно. Почему бы вам просто не загадать три желания? Или, для начала, хотя бы одно? Я, знаете ли, не могу сидеть тут у вас до скончания веков.

\emph{\textbf{Ахилл}} : Ах, какое разочарование\ldots{} А я-то так надеялся получить мои сто желаний.

\emph{\textbf{Гений}} : Боже мой, как неприятно разочаровывать людей. К тому же, мета-желания --- мой любимый вид желаний. Пожалуй, я могу постараться вам помочь. Это отнимет только одну минуточку\ldots{}

\emph{(Гений вынимает из легких складок своей, одежды почти такую же Лампу, какую он недавно положил в карман . На этот раз она не медная, а серебряная. На месте буквы «Л» на ней, помельче, выгравировано «МЛ.»)}

\emph{\textbf{Ахилл}} : А это что такое?

\emph{\textbf{Гений}} : Это моя Мета-Лампа.

\emph{(Он начинает тереть Мета-Лампу, из которой вырывается огромный клуб дыма. В дымных водоворотах вырисовывается гигантская призрачная~~фигура, нависшая над ними подобно башне. На этот раз джинн оказывается женщиной.)}

\textbf{Мета-Гений} : Я --- Мета-Гений. Вы звали меня, о, высокочтимый Гений? Каково ваше желание?

\emph{\textbf{Гений}} : Я хочу попросить вас, о Гений, и также БОГа, даровать мне исполнение специального желания: отмены ограничений на типы желаний на время одного Нетипового Желания. Можете ли вы это сделать?

\textbf{Мета-Гений} : Придется, разумеется, направить вашу просьбу по соответствующим каналам\ldots{} Это отнимет только полминутки.

\emph{(Вдвое быстрее чем Гений, она вынимает из легких складок своего платья почти такую же Лампу, какую тот недавно положил в карман. На этот раз она не серебряная, а золотая На месте букв «МЛ» на ней, помельче, выгравировано «ММЛ.»)}

\textbf{Ахилл (его голос теперь звучит на октаву выше)} : Что это такое?

\textbf{Мета-Гений} : Это моя Мета-Мета-Лампа\ldots{}

\emph{(Она начинает тереть Мета-Мета-Лампу и из нее вырывается огромный клуб дыма, в котором они различают смутные очертания фигуры, нависшей над ними, подобно башне.)}

\textbf{Мета-Мета-Гений} : Я Мета-Мета-Гений. Вы звали меня, о Мета-Гений? Чего вы желаете?

\textbf{Мета-Гений} : Я хочу попросить вас, о Гений, и также БОГа, даровать мне исполнение специального желания, отмены ограничений на типы желаний, на время одного Нетипового Желания. Можете ли вы это сделать?

\textbf{Мета-Мета-Гений} : Придется, разумеется, направить вашу просьбу по соответствующим каналам\ldots{} Это отнимет только четверть минутки.

\emph{(И, вдвое быстрее чем Мета-Гений, он достает из складок своего одеяния предмет, напоминающий золотую Мета-Мета-Лампу, с той разницей, что он сделан из\ldots{}}

~\emph{(\ldots втягивается обратно в Мета-Мета-Мета-Лампу, которую Мета-Мета-Гений прячет обратно в складки своего одеяния, вдвое медленнее, чем это делал Мета-Мета-Мета-Гений.)}

Ваше желание исполнено, о Мета-Гений.

\textbf{Мета-Гений} : Благодарю вас, о Гений, и БОГ. (И Мета-Мета-Гений, подобно всем высшим Гениям, исчезает в Мета-Мета-Лампе, которую Мета-Гений затем прячет в складках своего платья, вдвое медленнее, чем Мета-Мета-Гений.) Ваше желание исполнено, Гений.

\emph{\textbf{Гений}} : Благодарю вас, о Гений и БОГ. (И Мета-Гений, подобно всем высшим Гениям, исчезает в Мета-Лампе, которую Гений затем прячет в складках его одеяния, вдвое медленнее, чем Мета-Гений.)Ваше желание исполнено, Ахилл.

\emph{(Ровно минута прошла с тех пор, как он сказал: «Это отнимет только одну минуту».)}

\emph{\textbf{Ахилл}} : Благодарю вас, О Гений и БОГ.

\emph{\textbf{Гений}} : Рад вам сказать, Ахилл, что вам даровано право ровно на одно Нетиповое Желание. Это может быть просто желание, или мета-желание, или мета-мета-желание --- столько «мета», сколько вашей душеньке угодно --- даже бесконечно много, ежели желаете.

\emph{\textbf{Ахилл}} : Я вам бесконечно благодарен, Гений. Но вы задели мое любопытство. Прежде чем я скажу свое желание, не могли бы вы мне ответить, кто такой --- или что такое --- БОГ?

\emph{\textbf{Гений}} : Нет ничего проще. «БОГ» --- это сокращение. Оно расшифровывается так: «БОГ, Одолевающий Гения.»~~Слово «Гений» обозначает Гениев, Мета-Гениев, Мета-Мета-Гениев и т. д. Это Нетиповое слово.

\emph{\textbf{Ахилл}} : Но\ldots{} Но как БОГ может быть словом в своем собственном сокращении? Это совершенная бессмыслица!

\emph{\textbf{Гений}} : Разве вы ничего не слыхали о рекурсивных сокращениях? Я думал, это общеизвестно. Видите ли, БОГ означает «БОГ, Одолевающий Гения», что, в свою очередь, может быть расширено «БОГ, Одолевающий Гения, Одолевающий Гения», что также может быть расширено до «БОГ, Одолевающий Гения, Одолевающий Гения, Одолевающий Гения», что, в свою очередь, может быть расширено\ldots{} и так расширять его можно сколько угодно.

\emph{\textbf{Ахилл}} : Но я так никогда не кончу!

\emph{\textbf{Гений}} : Разумеется, нет. БОГа невозможно познать до конца.

\emph{\textbf{Ахилл}} : Гм-м-м\ldots{} Изрядная путаница. Что вы имели в виду, когда попросили Мета-Гения, а также БОГа, даровать исполнение специального желания?

\emph{\textbf{Гений}} : Я обращался не только к Мета-Гению, но и ко всем Гениям выше нее. С помощью рекурсивного сокращения это делается просто. Услыхав мою просьбу, Мета-Гений передала ее своему БОГу. Так просьба достигла Мета-Мета-Гения, который, в свою очередь, направил ее Мета-Мета-Мета-Гению\ldots{} Поднимаясь таким образом по инстанциям, просьба в конце концов достигает БОГа.

\emph{\textbf{Ахилл}} : Понятно. Значит, БОГ сидит наверху лестницы Гениев?

\emph{\textbf{Гений}} : Да нет же! Наверху ничего нет, так как никакого «верха» не существует Именно поэтому БОГ --- рекурсивное сокращение. БОГ --- не какой-то последний Супер-Гений; это «башня» всех Гениев, находящихся над данным Гением.

\emph{\textbf{Черепаха}} : Мне кажется, что в таком случае каждый Гений имеет свое представление о том, что такое БОГ, так как для каждого Гения БОГ --- это множество высших Гениев, и нет двух таких Гениев, у которых это множество было бы одинаковым.

\emph{\textbf{Гений}} : Вы совершенно правы --- и поскольку я самый «низкий» Гений из всех, мое представление о БОГе самое возвышенное. Бедные высшие Гении --- они воображают, что находятся ближе к БОГу. Какое кощунство!

\emph{\textbf{Ахилл}} : Ух ты! Слишком все это сложно. Поистине, чтобы изобрести БОГа, нужны Гении\ldots{}

\emph{\textbf{Черепаха}} : Вы действительно верите всем этом сказкам о БОГе, Ахилл?

\emph{\textbf{Ахилл}} : Ну конечно, верю. А вы что же, атеистка г-жа Черепаха? Или агностик?

\emph{\textbf{Черепаха}} : Не думаю. Может быть, я --- мета-агностик.

\emph{\textbf{Ахилл}} : Что-о-о? Ничего не понимаю.

\emph{\textbf{Черепаха}} : Понимаете, если бы я была мета-агностиком, я~бы сомневалась в том, агностик ли я --- но я не уверена, что я в этом сомневаюсь. Значит, я, наверное, мета-мета-агностик\ldots{} Ну, ладно. Скажите мне, Гений, а случается ли какому-нибудь Гению ошибиться и перепутать путешествующее вверх или вниз по цепи послание?

\emph{\textbf{Гений}} : Такое иногда случается; это самая распространенная причина того, что Нетиповые Желания не разрешаются. Видите ли, вероятность того, что путаница произойдет на каком-то ОПРЕДЕЛЕННОМ этапе, ничтожно мала --- но когда у вас имеется цепь из бесконечного числа этапов, становится практически неизбежным, что ГДЕ-НИБУДЬ выйдет ошибка. На самом деле, как это ни странно, ошибок бывает бесконечное множество, хотя они и встречаются весьма редко.

\emph{\textbf{Ахилл}} : Тогда это просто чудо, когда какое-нибудь Нетиповое Желание вообще бывает даровано.

\emph{\textbf{Гений}} : Не совсем так. Большинство ошибок остается без последствий, а некоторые ошибки взаимоуничтожаются. Но иногда, хотя и довольно редко, причиной неисполнения Нетипового Желания может быть ошибка какого-то одного несчастного Гения. Когда такое происходит, виновник прогоняется сквозь бесконечный строй, и БОГ наказывает его шлепками. Это большое развлечение для шлепающих и к тому же совсем не больно для виновника. Вас бы позабавило это зрелище.

\emph{\textbf{Ахилл}} : Было бы интересно посмотреть! Но это бывает только в том случае, когда не исполняется Нетиповое Желание?

\emph{\textbf{Гений}} : Верно.

\emph{\textbf{Ахилл}} : Гм-м-м\ldots{} Кажется, я знаю, чего мне пожелать.

\emph{\textbf{Черепаха}} : Да? Чего же?

\emph{\textbf{Ахилл}} : Я бы хотел, чтобы мое желание не исполнилось!

\emph{(В этот момент происходит такое странное событие --- да можно ли это вообще назвать «событием»? --- что его невозможно описать; а значит, мы и пытаться не будем.)}

Ахилл: Интересно, что означает этот загадочный комментарий?

Черепаха: Он относится к Нетиповому Желанию, исполнения которого попросил Ахилл.

Ахилл: Но он еще ничего не пожелал!

Черепаха: Напротив; он сказал: «Я хотел бы, чтобы мое желание не исполнилось,» и Гений принял ЭТИ СЛОВА за желание.

\emph{(В этот момент в коридоре раздаются шаги; они медленно приближаются.)}

Ахилл: Ой! Какой кошмар!

\emph{(Шаги останавливаются и затем начинают удаляться.)}

Черепаха: Уф-ф!\ldots{}

Ахилл: История продолжается, или это уже конец? Переверните-ка страницу и давайте проверим.

\emph{(Черепаха переворачивает страницу «Джинна и настойки», и они обнаруживают, что история продолжается.)}

\emph{\textbf{Ахилл}} : Эй! Что стряслось? Где мой Гений? Моя лампа? Моя чашка кофе-экспресса? Что случилось с нашими юными друзьями из Выпуклого и Вогнутого Миров? И что здесь делают все эти ящерицы?

\emph{\textbf{Черепаха}} : Боюсь, что наш контекст был восстановлен неправильно.

\emph{\textbf{Ахилл}} : Интересно, что означает этот загадочный комментарий?

\emph{\textbf{Черепаха}} : Я имею в виду Нетиповое Желание, исполнения которого вы попросили.

\emph{\textbf{Ахилл}} : Но я еще ничего не пожелал!

\emph{\textbf{Черепаха}} : Напротив --- вы сказали: «Я хотел бы, чтобы мое желание не исполнилось», и Гений принял ЭТИ СЛОВА за желание.

\emph{\textbf{Ахилл}} : Ой! Какой кошмар!

\emph{\textbf{Черепаха}} : Это называется ПАРАДОКС. Чтобы исполнить это Нетиповое Желание, надо отказать в его исполнении. В то же время отказать в его исполнении значило бы исполнить его!

\emph{\textbf{Ахилл}} : Так что же произошло? Земля остановилась? Пространство закуклилось?

\emph{\textbf{Черепаха}} : Нет --- просто система отказала.

\emph{\textbf{Ахилл}} : Что это значит?

\emph{\textbf{Черепаха}} : Это значит, что мы оба мгновенно очутились в Лимбедламии.

\textbf{\emph{Ахилл}} : Где?

\emph{\textbf{Черепаха}} : Лимбедламия --- страна прошедшей икоты и перегоревших лампочек. Это что-то вроде зала ожидания, где дремлют программы в ожидании компьютеров. Нельзя сказать, как долго мы пробыли в Лимбедламии --- может быть, несколько минут, часов или дней, а может быть, и несколько лет.

\emph{\textbf{Ахилл}} : Я не знаю, при чем здесь программы или компьютеры. Я знаю только то, что не успел загадать желания! Верните моего Гения обратно!

\emph{\textbf{Черепаха}} : Мне очень жаль, Ахилл, но вы упустили свой шанс. Из-за вас отказала Система. Благодарите Бога, что мы вообще куда-то попали. Все могло быть гораздо хуже. Не имею ни малейшего понятия, где мы очутились\ldots{}

\emph{\textbf{Ахилл}} : Я знаю, это другая картина Эшера. Она называется «Рептилии».

\emph{\textbf{Черепаха}} : Ага! Система попыталась запомнить как можно больше нашего контекста перед тем, как отказать; ей~~удалось сохранить в памяти то, что мы находились в картине Эшера с ящерицами. Весьма похвально!

\emph{Рис. 24. М. К. Эшер. «Рептилии» (литография, 1943).}

\emph{\textbf{Ахилл}} : И взгляните не наш ли это флакончик с Выталкивающей настойкой там на столе, рядом с ящеричным хороводом?

\emph{\textbf{Черепаха}} : Безусловно, это он, Ахилл. Должна сказать, что нам действительно везет. Система обошлась с нами по-божески, вернув нам эту драгоценную жидкость!

\emph{\textbf{Ахилл}} : Это верно. Теперь мы можем вытолкнуться из эшеровского мира и вернуться ко мне домой.

\emph{\textbf{Черепаха}} : Интересно, что это за книги там, рядом с настойкой? (Она берет книгу поменьше, открытую в середине.) Эта книжица выглядит довольно занимательно.

\emph{\textbf{Ахилл}} : Правда? Как она называется?

\emph{\textbf{Черепаха}} : «Занимательные приключения Черепахи и Ахилла или Вокруг света от точки до точки.» Интересно было было бы почитать немного.

\emph{\textbf{Ахилл}} : Вы можете читать, если хотите, а я не собираюсь рисковать, какая-нибудь ящерица может запросто толкнуть флакон и разлить настойку. Я выпью свою порцию немедленно! (Он бросается к столу и протягивает руку к пузырьку с настойкой; при этом он случайно толкает его. Пузырек падает со стола и катится.) Ой! Г-жа Ч, смотрите! Я нечаянно столкнул настойку на пол и она покатилась\ldots~ к лестнице! Быстрее, а то свалится вниз!

\emph{(Но Черепаха погружена в свою книгу.)}

\emph{\textbf{Черепаха (бормочет)}} : А? Эта история выглядит захватывающе.

\emph{\textbf{Ахилл}} : Г-жа Ч, скорей, на помощь! Помогите поймать пузырек!

\emph{\textbf{Черепаха}} : Что за шум?

\emph{\textbf{Ахилл}} : Пузырек с настойкой, я столкнул его со стола, и сейчас он катится, и\ldots{} (В этот момент пузырек достигает первой ступеньки и падает вниз ) Ох! Что теперь делать? Г-жа Черепаха, вас это не волнует? Мы теряем настойку! Она только что свалилась с лестницы. Единственная наша надежда --- перейти на другой этаж!

\emph{\textbf{Черепаха}} : Перейти на другой рассказ? С превеликим удовольствием! Желаете ко мне присоединиться?

\emph{(Она начинает читать вслух, Ахилл застывает в нерешительности, не зная, что предпринять. Наконец он решает остаться и начинает читать за Черепаху.)}

\textbf{Ахилл} : Как здесь темно, г-жа Ч Я ничего не вижу. Ой! Я натолкнулся на стену. Осторожнее!

\textbf{Черепаха} : У меня есть пара тросточек Вот, держи~те одну. Вы можете прощупывать дорогу, чтобы ни с чем не сталкиваться.

\textbf{Ахилл} : Отличная идея. (Он берет трость.) Вам не кажется, что дорога слегка изгибается влево?

\textbf{Черепаха} : Да, пожалуй.

\textbf{Ахилл} : Интересно, где мы находимся. И увидим ли мы когда-нибудь дневной свет опять. Как жаль, что я вас послушался и проглотил эту штуковину «Выпей меня».

\textbf{Черепаха} : Уверяю вас, она совершенно безвредна. Я делала это много раз и никогда еще об этом не пожалела. Лучше расслабьтесь и постарайтесь получить удовольствие от того, что вы так чудесно уменьшились.

\textbf{Ахилл} : Уменьшился? Что вы со мной сделали, г-жа Черепаха?

\textbf{Черепаха} : Пожалуйста, не обвиняйте меня. Вы проделали все по вашему собственному желанию.

\textbf{Ахилл} : Так вы меня уменьшили? А вдруг лабиринт, в котором мы находимся, такой крохотный, что кто-нибудь может на него наступить?

\textbf{Черепаха} : Лабиринт? Лабиринт? Может ли это быть? Неужели мы попали в знаменитый лабиринт ужасного Мажотавра?

\textbf{Ахилл} : Ой, мамочка! Что это такое?

\textbf{Черепаха} : Говорят --- хотя я лично в это никогда не верила --- что злобный Мажотавр создал миниатюрный лабиринт и сидит в углублении в центре, поджидая невинных жертв, затерявшихся в чудовищно запутанных переходах. Когда они, окончательно заблудившись, забредают в центр, он начинает над ними смеяться, да так громко, что засмеивает их до смерти!

\textbf{Ахилл} : О боже, не может быть!

\textbf{Черепаха} : Это только миф. Смелее, Ахилл!

\emph{(И храбрая парочка осторожно двигается вперед.)}

\textbf{Ахилл} : Потрогайте эти стены. Они напоминают сморщенные жестяные листы --- только все морщины разного размера.

\emph{(Чтобы подчеркнуть свои слова, он прикладывает конец трости к стене и идет вперед. Трость подпрыгивает на неровностях стены --- длинный изогнутый коридор, в котором они находятся, наполняется странными звуками.)}

\textbf{Черепаха (встревоженно)} : Что это такое?

\textbf{Ахилл} : Это я веду тросточкой по стене.

\textbf{Черепаха} : Ох --- я было подумала, что это рев кровожадного Мажотавра.

\textbf{Ахилл} : Я думал, вы сказали, что это все выдумки.

\textbf{Черепаха} : Конечно. Бояться совершенно нечего.

\emph{(Ахилл снова прикладывает трость к стене и идет вперед. При этом слышна музыка; звуки исходят из того места, где трость прикасается к стене.)}

\textbf{Черепаха} : Ох, Ахилл, у меня дурное предчувствие --- мне кажется, что этот Лабиринт не такой уж и миф.

\textbf{Ахилл} : Погодите-ка, что это заставило вас так внезапно передумать?

\textbf{Черепаха} : Слышите эту музыку? (Чтобы лучше слышать, Ахилл опускает трость, и мелодия прекращается.) Эй! Поставьте трость обратно! Я хочу послушать конец этой пьесы!

\emph{Рис. 25. Критский лабиринт (Итальянская гравюра; школа Финигерры) Из книги У. Г. Маттьюса «Лабиринты: их история и развитие» (W.H. Mattews, Mazes and Labyrinths. Their History and Development.)}

(Ахилл, сбитый с толку, повинуется и музыка возобновляется.) Благодарю. Теперь я догадалась, где мы находимся.

\textbf{Ахилл} : Правда? Где же?

\textbf{Черепаха} : Мы идем по звуковой дорожке пластинки, лежащей в конверте. Ваша трость, скребущая по морщинам на стене, действует как иголка, бегущая по звуковой дорожке, позволяя нам слушать музыку.

\textbf{Ахилл} : Ох, нет, нет\ldots{}

\textbf{Черепаха} : Что такое? Разве вы не радуетесь? Когда еще вы находились в таком интимном контакте с музыкой?

\textbf{Ахилл} : Как же я смогу выигрывать соревнования по бегу против людей в натуральную величину, если я теперь меньше блохи, г-жа Черепаха?

\textbf{Черепаха} : Ах, так вот что вас волнует? Право, Ахилл, стоит ли из-за этого беспокоиться\ldots{}

\textbf{Ахилл} : Вы говорите так, что у меня создается впечатление, что вы вообще никогда не волнуетесь.

\textbf{Черепаха} : Не знаю, не знаю\ldots{} Я уверена только в~~одном: о чем я не жалею, так это о том, что я уменьшилась. В особенности тогда, когда нам грозит страшная опасность от чудовищного Мажотавра.

\textbf{Ахилл} : О ужас!.. Вы хотите сказать, что\ldots{}

\textbf{Черепаха} : Боюсь, что да, Ахилл. Музыка выдала его с головой.

\textbf{Ахилл} : Каким же это образом?

\textbf{Черепаха} : Очень просто. Когда я услышала мелодию В-А-С-H в верхнем голосе, меня осенило: на звуковых дорожках, по которым мы идем, записано не что иное, как «Маленький гармонический лабиринт», одна из наименее известных органных пьес Баха. Она названа так из-за модуляций, таких частых, что от них начинает кружиться голова.

\textbf{Ахилл} : Ч-что --- что это такое и с чем это едят?

\textbf{Черепаха} : Как вы знаете, большинство музыкальных произведений написано в какой-нибудь тональности --- например, «до мажор», как эта пьеса.

\textbf{Ахилл} : Я уже слышал это название раньше. Не правда ли, это значит, что «до» --- та нота, на которой произведение должно заканчиваться?

\textbf{Черепаха} : Да, «до» --- это что-то вроде ключа от дома, куда вы хотите попасть. Ключ бывает и в музыке.

\textbf{Ахилл} : Значит, сначала мы удаляемся от этого «дома», чтобы потом туда возвратиться?

\textbf{Черепаха} : Правильно. В музыкальных произведениях часто используются мелодии, уводящие в сторону от ключевой тональности. Мало-помалу нарастает напряжение, и слушатель начинает все сильнее скучать по «дому» ---~ему хочется вновь услышать ключевую тональность.

\textbf{Ахилл} : Таким образом, в конце пьесы я всегда буду чувствовать такое удовлетворение, как будто я всю жизнь желал услышать именно эти звуки?

\textbf{Черепаха} : Точно. Композитор использует свои знания о гармонической прогрессии, чтобы таким образом управлять нашими чувствами и пробудить в нас желание услышать ключевую тональность\ldots{}

\textbf{Ахилл} : Понятно, но, кажется, вы собирались рассказать мне о модуляциях\ldots{}

\textbf{Черепаха} : Ах, да. Один из важных приемов, которые композитор может использовать где-то в середине пьесы, называется модуляцией; это означает, что он устанавливает временную~~«цель», отличную от конечного разрешения в ключевую тональность.

\textbf{Ахилл} : А-а-а\ldots{} кажется, я понимаю. Вы имеете в виду, что определенная серия аккордов изменяет гармоническое напряжение таким образом, что я начинаю желать разрешения в новой тональности?

\textbf{Черепаха} : Именно так. Это усложняет ситуацию, поскольку, наряду с этим новым желанием, подсознательно вы все время ощущаете, что ваша конечная цель --- ключевая тональность, в данном случае, «до мажор». И когда временная цель бывает достигнута, то\ldots{}

\textbf{Ахилл (внезапно начиная возбужденно жестикулировать)} : О, послушайте только: какие восхитительные поднимающиеся вверх аккорды! Какой прекрасный конец у «Маленького гармонического лабиринта»!

\textbf{Черепаха} : Нет, Ахилл, это не конец, это просто ---

\textbf{Ахилл} : Да нет, разумеется, это конец! Вот это да! Какой могучий финал! Какое облегчение! Вот разрешение так разрешение! Гениально! (Поет): Ля-ля-ля\ldots{} (И точно, в этот момент музыка прекращается; стен больше нет, и Черепаха с Ахиллом оказываются в открытом пространстве.) Вот видите, музыка действительно кончилась. Ну, что я вам говорил?

\textbf{Черепаха} : Что-то здесь не так. Эта запись позорит музыкальный мир.

\textbf{Ахилл} : Почему это?

\textbf{Черепаха} : Я только что вам объяснил: Бах промодулировал здесь от «до» в «ля», так что временной целью было услышать мелодию в ключе «ля». Это значит, что вы чувствуете сразу два желания: с одной стороны, вы ожидаете разрешения в «ля», а с другой стороны, вы все время помните, что конечная цель --- триумфальное возвращение в «до мажор».

\textbf{Ахилл} : Почему надо все время о чем-то помнить, когда слушаешь музыку? Разве музыка --- только упражнение для ума?

\emph{Черепаха} : Нет, конечно. Некоторые произведения весьма интеллектуальны, но большинство довольно просты. Обычно наше ухо или мозг делают все «расчеты» за нас, в то время как чувства решают, что именно нам хочется услышать. Нам не приходится думать об этом. Но в этой пьесе Бах проделывает разные трюки, в надежде сбить слушателя с толку --- и надо сказать, что в вашем случае, Ахилл, он вполне преуспел!

\textbf{Ахилл} : Вы хотите сказать, что я среагировал на разрешение во «второстепенной» тональности?

\textbf{Черепаха} : Правильно.

\textbf{Ахилл} : Все же я и сейчас уверен, что это был конец!

\textbf{Черепаха} : Именно этого эффекта Бах и добивался. Вы угодили прямиком в его ловушку. Это место написано так, что оно звучит как финал; но если вы внимательно следите за развитием гармонической прогрессии, вы увидите, что оно не в том ключе. Видимо, не только вы, но и та несчастная студия звукозаписи решила, что это конец, и записала только часть пьесы!

\textbf{Ахилл} : Какую недостойную шутку сыграл со мной старик Бах!

\textbf{Черепаха} : Как раз этого он и хотел --- заставить вас заблудиться в его «Лабиринте». Видите ли, злодей Мажотавр --- сообщник Баха. Если вы не остережетесь, он засмеет вас до смерти --- а может быть, и меня вместе с вами!

\textbf{Ахилл} : Надо срочно уносить ноги отсюда! Скорее! Если мы побежим обратно по звуковым дорожкам, то выберемся из пластинки прежде, чем страшный Мажотавр нас обнаружит!

\textbf{Черепаха} : Ну нет, мое ухо слишком чувствительно, чтобы вынести странные аккорды, получающиеся, когда время обращается вспять!

\textbf{Ахилл} : Ах, г-жа Ч, как же мы выберемся отсюда, если мы не можем вернуться по нашим следам?

\textbf{Черепаха} : Хороший вопрос\ldots{} (Почти отчаявшись, Ахилл начинает бегать взад-вперед в темноте. Внезапно раздается сдавленный крик и затем --- БА-БАХ! --- глухой звук падения.) Ахилл? С вами все в порядке?

\textbf{Ахилл} : Ничего особенного, только маленькая встряска: я свалился в какую-то ямину.

\textbf{Черепаха} : Вы угодили прямиком в логово Страшного Мажотавра! Постараюсь вас вытащить --- нам надо удирать побыстрее!

\textbf{Ахилл} : Осторожнее, г-жа Ч --- я совсем не хочу, чтобы и Вы тоже попали в западню\ldots{}

\textbf{Черепаха} : Да не суетитесь вы, Ахилл. Все будет в порядке\ldots{} (Внезапно раздается сдавленный крик и затем --- БА-БАХ! --- глухой звук~~падения.)

\textbf{Ахилл} : Г-жа Ч, вы тоже упали? Не ушиблись?

\textbf{Черепаха} : Кроме моей гордости, ничего не пострадало.

\textbf{Ахилл} : Вот теперь мы действительно попали в переплет!

\emph{(Внезапно, в опасной близости от них, друзья слышат оглушительный хохот.)}

\textbf{Черепаха} : Осторожно, Ахилл --- тут дело нешуточное!

\textbf{Мажотавр} : Ха-ха-ха! Хи-хи-хи! Хо-хо-хо!

\textbf{Ахилл} : Я слабею на глазах, г-жа Ч\ldots{}

\textbf{Черепаха} : Старайтесь не обращать внимания на его смех --- это ваша единственная надежда.

\textbf{Ахилл} : Я сделаю все, что в моих силах --- ах, если бы сейчас пропустить для храбрости рюмочку-другую\ldots{}

\textbf{Черепаха} : Мне кажется, я чувствую знакомый запах\ldots{} Не вытаскин ли это?

\textbf{Ахилл} : И правда\ldots{} откуда этот запах?

\textbf{Черепаха} : По-моему, это здесь\ldots{} О! Я нашла целую бутыль! Это он и есть!

\textbf{Ахилл} : Вытаскин! Давайте напьемся с горя!

Черепаха: Надеюсь, что это не протолкин --- они до того похожи, что их трудно различить.

\textbf{Ахилл} : Что вы сказали про Толкиена?

\textbf{Черепаха} : Я ничего подобного не говорила. У вас уже галлюцинации начинаются\ldots{}

\textbf{Ахилл} : Б-батюшки мои! Надеюсь, что нет\ldots{} Ну что же, поехали!

\emph{(И друзья начинают отхлебывать вытаскин (или протолкин?) --- и вдруг --- ХЛОП! Кажется, это-таки оказался вытаскин\ldots)}

\emph{\textbf{Черепаха}} : Забавная история, ничего не скажешь. Вам понравилось?

\emph{\textbf{Ахилл}} : Так, ничего себе\ldots{} Интересно, выбрались ли они в конце концов из ямы страшного Мажотавра? Бедняга Ахилл, он так хотел опять стать большим.

\emph{\textbf{Черепаха}} : Не беспокойтесь --- они выбрались, и Ахилл снова вырос до своих обычных размеров. Вытаскин оказался весьма кстати\ldots{}

\emph{\textbf{Ахилл}} : Не знаю, не знаю\ldots{} Единственное, в чем я сейчас АБСОЛЮТНО уверен, это в том, что нам не мешало бы найти нашу бутылочку с настойкой --- у меня уже давно горло пересохло. И ничто так не утоляет жажду, как выталкивающая настойка

\emph{\textbf{Черепаха}} : Она к тому же известна своим тонизирующим~~действием. Известны случаи, когда народ просто с ума по ней сходил. Например, когда в начале века продуктовая фабрика Шёнберга перестала производить джин с тоником и начала производство какао, вы не представляете себе, какой из-этого поднялся шум --- настоящая какаофония!

\emph{\textbf{Ахилл}} : Воображаю\ldots{} Но давайте же искать настойку! Погодите --- взгляните-ка на этих ящериц на столе! Не кажется ли вам, что в них есть что-то необычное?

\emph{\textbf{Черепаха}} : Не вижу ничего особенного. А что такое?

\emph{\textbf{Ахилл}} : Посмотрите: они вылезают из плоскости картины без помощи выталкивающей настойки! Как они это делают?

\emph{\textbf{Черепаха}} : Разве я вам не говорила? Вы можете вылезти из картины, двигаясь перпендикулярно ее плоскости. Ящерки научились лезть НАВЕРХ, когда они хотят выбраться из двухмерного мира альбома.

\emph{\textbf{Ахилл}} : Может быть, мы можем так же выбраться из этой картины Эшера наружу?

\emph{\textbf{Черепаха}} : Разумеется --- нужно только подняться уровнем выше. Хотите попытаться?

\emph{\textbf{Ахилл}} : Все что угодно, только бы попасть домой! Я уже сыт по горло этими занимательными приключениями.

\emph{\textbf{Черепаха}} : В таком случае, следуйте за мной наверх.

\emph{(И они поднимаются на один уровень.)}

\emph{Ахилл} : Хорошо быть снова у себя дома\ldots{} Но постойте, здесь что-то не то! Это вовсе не мой дом --- это ВАШ дом, г-жа Черепаха!

\emph{Черепаха} : Вы правы --- и я предовольна, так как перспектива тащиться от вас к себе домой мне совершенно не улыбалась. Я прямо-таки валюсь с лап от усталости.

\emph{Ахилл} : Что ж, я как раз не возражаю против небольшой прогулки; так что, мне кажется, все сложилось довольно удачно.

\emph{Черепаха} : Я думаю! Вот это удача так удача!


% % \subsubsection{ГЛАВА V: Рекурсивные структуры  процессы}
% \include{fragments/xx18}
% % \subsubsection{Канон с интервальны увеличением}
% \subsubsection{Канон с интервальны увеличением}

\emph{Ахилл и Черепаха только что доели превосходный ужин на двоих в лучшем китайском ресторане города.}

\emph{Ахилл} : Здорово вы управляетесь с палочками, г-жа Ч.

\emph{Черепаха} : Приходится --- я с детства питаю слабость к восточной кухню. Как насчет вас, Ахилл --- вам понравилось?

\emph{Ахилл} : Еще как! Я никогда раньше не пробовал китайской еды, и сегодняшний ужин был приятным знакомством с ней. А сейчас, если вы не торопитесь мы можем еще немного посидеть и поболтать.

\emph{Черепаха} : Что ж, с удовольствием побеседую с вами, пока мы пьем чай. Официант! (Подходит официант.) Пожалуйста, принесите наш счет. И еще немного чая! (Официант торопливо уходит.)

\emph{Ахилл} : Вы можете понимать больше меня в китайской кухне, г- жа Ч, но могу поспорить, что о японской поэзии я знаю побольше вас. Читали ли вы когда-нибудь хайку?

\emph{Черепаха} : Боюсь, что нет. Что это такое?

\emph{Ахилл} : Хайку --- это японская поэма, в которой семнадцать слогов. Правильнее сказать, что это мини-поэма, наводящая на размышление так,же, как благоуханный розовый лепесток или покрытые росой кувшинки в пруду. Обычно хайку состоит из группы пяти слогов, затем --- семи, и затем --- снова пяти.

\emph{Черепаха} : Такая краткость --- всего семнадцать слогов --- но где же здесь смысл?

\emph{Ахилл} : Смысл живет также в голове читателя --- не только в хайку.

\emph{Черепаха} : Гм-м-м\ldots{} Это утверждение наводит на размышления.

\emph{(Подходит официант со счетом, чайничком, полным чая, и парой печений «с сюрпризом» --- бумажкой, на которой написана судьба едока.)}

Премного благодарна. Еще чайку не желаете, Ахилл?

\emph{Ахилл} : Пожалуй. Эти печеньица выглядят весьма аппетитно. (Берет печенье, откусывает кусочек и начинает жевать.) Эй --- что эта за штуковина тут внутри? Клочок бумаги?

\emph{Черепаха} : Это ваша судьба, Ахилл. Во многих китайских ресторанах вместе со счетом подают печенья с судьбой-сюрпризом, чтобы смягчить удар. Завсегдатаи китайских ресторанов обычно считают их не за печенья, а за посланцев судьбы. К несчастью, вы, кажется, проглотили кусочек своей судьбы. Что там написано, на оставшемся клочке?

~\emph{Ахилл} : Странно --- все буквы сгрудились в кучу, нет никакого деления на слова. Может быть, это надо расшифровать? О, я понял если расставить промежутки там, где надо, получится: «НИС КЛАДУН ИЛ АДУ». Поистине, адская бессмыслица! Может быть, это что-то вроде хайку, от которого я отъел большинство слогов.

\emph{Черепаха} : В таком случае, ваша судьба теперь всего лишь 6/17 хайку. Веселенькие ассоциации все это вызывает. Колдуны, болота, черти, клады\ldots{} Что и говорить, картинка унилая\ldots{} унылая, я имею в виду. Это звучит как комментарий к новой форме искусства --- 6/17 хайку. Можно мне взглянуть?

\emph{Ахилл (протягивая Черепахе узкий клочок бумаги)} : Конечно.

\emph{Черепаха} : Но, Ахилл, в моей «расшифровке» получается нечто совершенно другое! Это вовсе не 6/17 хайку, а шестисложное послание --- и вот что в нем написано «НИ СКЛАДУ НИ ЛАДУ». Поистине, глубокий комментарий к этой новой форме искусства --- 6/17 хайку!

\emph{Ахилл} : Вы правы. Удивительно, что это послание содержит комментарий о самом себе!

\emph{Черепаха} : Я только передвинула рамку чтения на единицу --- сдвинула все промежутки между словами на один интервал.

\emph{Ахилл} : Посмотрим, какая судьба выпала сегодня вам.

\emph{Черепаха (ловко разламывая печенье, читает)} : «Судьбу едока не печенье содержит, а его рука».

\emph{Ахилл} : Ваша «судьба» тоже хайку, г-жа Черепаха --- по крайней мере, в ней семнадцать слогов. 5-7-5.

\emph{Черепаха} : Потрясающе! Я бы сама этого ни за что не заметила, Ахилл --- такие вещи только вы подмечаете. То, что меня больше всего удивило, это сам текст послания; разумеется, его можно интерпретировать по-разному.

\emph{Ахилл} : Наверное, мы все интерпретируем послания по-своему, когда с ними сталкиваемся\ldots{} (Лениво рассматривает чаинки на дне чашки.)

\emph{Черепаха} : Подлить вам чаю?

\emph{Ахилл} : Да, спасибо. Кстати, как поживает ваш товарищ, старый Краб? Я частенько о нем вспоминаю, с тех пор, как вы рассказали мне о его диковиной патефонной войне.

\emph{Черепаха} : Я ему о вас кое-что рассказала, и ему тоже не терпится с вами встретиться. У него все в порядке;на днях он приобрел новую штуковину из серии проигрывателей,~какой-то странный проигрыватель-автомат.

\emph{Ахилл} : Расскажите-ка мне об этом поподробнее. Обожаю эти автоматы --- кругом разноцветные огоньки, и когда опустишь монетку, машина играет глупые песни, которые так и окунают тебя в старое доброе прошлое\ldots{}

\emph{Черепаха} : Этот проигрыватель слишком велик, чтобы держать его дома, и Краб построил для него во дворе специальный навес.

\emph{Ахилл} : Не представляю себе, почему он такой большой? Может, в нем огромная коллекция пластинок?

\emph{Черепаха} : На самом деле, в нем всего одна запись.

\emph{Ахилл} : Что? Проигрыватель-автомат с одной пластинкой? Это уже само по себе противоречие! Почему же он так велик? Может, его единственная пластинка --- гигант двадцати футов в диаметре?

\emph{Черепаха} : Да нет, пластинка самая обыкновенная.

\emph{Ахилл} : Ах, г-жа Черепаха, не иначе как вы надо мной смеетесь. Ну скажите на милость, что это за автомат с единственной песней?

\emph{Черепаха} : Кто сказал хотя бы слово о единственной песне?

\emph{Ахилл} : Любой проигрыватель-автомат, с которым я когда-либо сталкивался, подчинялся фундаментальной аксиоме этих аппаратов: «одна пластинка, одна песня.»

\emph{Черепаха} : Этот автомат не таков, Ахилл. Единственная пластинка в нем расположена вертикально, и за ней находится небольшая, но сложная система рельсов, на которых подвешены проигрыватели. Когда вы нажимаете на пару кнопок, скажем, В-1, вы выбираете один из проигрывателей. Это пускает в действие механизм, и проигрыватель со скрипом отправляется по ржавым рельсам. Вскоре он прибывает к краю пластинки, и --- щелк! --- устанавливается в нужную позицию.

\emph{Ахилл} : И тогда пластинка начинает вращаться, и раздается музыка, правда?

\emph{Черепаха} : Не совсем. Пластинка остается неподвижной --- вращается сам проигрыватель.

\emph{Ахилл} : Я мог бы догадаться. Но каким же образом, если у вас только одна пластинка, вы можете выудить из этой сумасшедшей конструкции больше одной песни?

\emph{Черепаха} : Я и сама спрашивала Краба об этом. Он посоветовал мне попробовать самой. Я нашла в кармане монетку (ее хватало на три песни), засунула ее в щель и нажала наугад: В-1, С-3, и V-10.

\emph{Ахилл} : Значит, патефон В-1 поехал по рельсам, подкатился к вертикальной пластинке и стал вращаться?

\emph{Черепаха} : Точно. Получилась довольно приятная музыка, основанная на знаменитой старой мелодии В-А-С-H, которую, я полагаю, вы еще помните\ldots{}

\emph{Ахилл} : Могу ли я ее забыть?

\emph{Черепаха} : Это был патефон В-1. Когда мелодия закончилась, он отъехал назад, чтобы дать место патефону С-3.

\emph{Ахилл} : Неужели С-3 заиграл другую мелодию?

\emph{Черепаха} : Именно так.

\emph{Ахилл} : А, понимаю. Он проиграл другую сторону пластинки, или, может быть, другую полосу на этой стороне.

\emph{Черепаха} : Нет, на этой пластинке дорожки только с одной стороны и на ней только одна полоса.

\emph{Ахилл} : Ничего не понимаю. Получить разные песни из одной записи НЕВОЗМОЖНО!

\emph{Черепаха} : Я тоже так думала, пока не увидела проигрыватель м-ра Краба.

\emph{Ахилл} : Как звучала эта вторая песня?

\emph{Черепаха} : Это-то как раз интересно: она была основана на мелодии C-A-G-E.

\emph{Ахилл} : Но это совершенно иная мелодия!

\emph{Черепаха} : Верно.

\emph{Ахилл} : Кажется, Джон Кэйдж --- это композитор, создатель авангардистской музыки? Мне кажется, я читал о нем в одной из моих книг хайку.

\emph{Черепаха} : Точно. Многие его творения довольно известны, например, 4'33'' --- трехчастная пьеса, состоящая из безмолвий разной длины. Она необыкновенно выразительна --- если у вас есть вкус к подобным вещам.

\emph{Ахилл} : Что ж, если бы я находился в шумном ресторане, я с удовольствием поставил бы 4'33" Кэйджа на музыкальном автомате. Это могло бы быть некоторым облегчением!

\emph{Черепаха} : Правильно --- кому хочется слушать звон тарелок и стук ножей? Эта пьеса пришлась бы весьма кстати еще в одном месте, в Павильоне~Гигантских Кошек, во время кормления.

\emph{Ахилл} : Вы намекаете на то, что Кэйджу место в зверинце? Что ж, если учесть, что его фамилия в переводе с английского значит «клетка»\ldots{} Но вернемся к крабьему музыкальному автомату --- я ничего не понимаю. Как могут на одной и той же записи быть сразу В-А-С-H и C-A-G-E?

\emph{Черепаха} : Если вы посмотрите повнимательней, Ахилл, вы можете подметить, что между ними есть некоторая связь. Вот, взгляните: что у вас получится, если вы последовательно запишете интервалы мелодии В-А-С-H?

\emph{Ахилл} : Ну-ка, посмотрим\ldots{} Сначала она понижается на полтона, от В до А (я имею в виду немецкое В); затем поднимается на три полутона до С, и, наконец, опускается на полутон, до H. Получается следующая схема:

-1, +3, -1

\emph{Черепаха} : Совершенно верно. А как насчет C-A-G-E?

\emph{Ахилл} : Здесь мелодия сначала идет на три полутона вниз, потом поднимается на десять полутонов, и снова опускается на три полутона. Получается:

-3, +10, -3

Очень похоже на первую мелодию, правда?

\emph{Черепаха} : Действительно, похоже. В некотором смысле, у этих двух мелодий совершенно одинаковый «скелет». Вы можете получить C-A-G-E из~В-А-С-H, умножив все интервалы на 3,5 и беря ближайшее целое число.

\emph{Ахилл} : Вот это да! Это значит, что на звуковых дорожках записан только некий основной код, который разные проигрыватели интерпретируют по-разному?

\emph{Черепаха} : Я не уверена --- этот уклончивый Краб не посвятил меня во все детали. Но мне удалось услышать третью песню, произведенную на проигрывателе В-10.

\emph{Ахилл} : И как она звучала?

\emph{Черепаха} : Ее мелодия состояла из огромных интервалов: В-С-А-H.

Схема в полутонах была такая:

-10, +33, -10

Эта мелодия получается из C-A-G-E, если снова умножить интервалы на 3,3 и округлить результаты до ближайшего целого числа.

\emph{Ахилл} : Есть ли какое-то название у такого умножения интервалов?

\emph{Черепаха} : Его можно назвать «интервальным увеличением». Оно похоже на прием ритмического увеличения темы канона. При этом длительность всех нот мелодии умножается на какое-либо постоянное число. В результате мелодия замедляется. Здесь же интересным образом расширяется диапазон мелодии.

\emph{Ахилл} : Удивительно. Так что все три мелодии, что вы услышали, были интервальными увеличениями одной и той же схемы звуковых дорожек?

\emph{Черепаха} : Таково мое заключение.

\emph{Ахилл} : Забавно, когда мы увеличиваем В-А-С-H, у нас получается C-A-G-E, a когда мы опять увеличиваем C-A-G-E, то снова получаем В-А-С-H, только теперь он весь перевернут, словно В-А-С-H разнервничался, проходя через промежуточный этап C-A-G-E.

\emph{Черепаха} : Поистине, глубокий комментарий к этой новой форме искусства --- музыке Кэйджа.


% % \subsubsection{ГЛАВА VI: Местонахождени значения}
% \include{fragments/xx20}
% % \subsubsection{Хроматическая фантазия и фига}
% \subsubsection{Хроматическая фантазия и фига}

\emph{Вдоволь наплававшись в пруду, Черепаха вылезает и отряхивается; тут мимо идет Ахилл.}

\emph{Черепаха} : День добрый, Ахилл. Я о вас только что вспоминала, пока купалась.

\emph{Ахилл} : Ну не забавно ли? И вы у меня из головы не выходили, пока я бродил по лугам. Смотрите, я нашел для вас фигу. Правда, она еще зеленая\ldots{}

\emph{Черепаха} : Вы полагаете? Это напоминает мне об одной идейке\ldots{} Хотите послушать?

\emph{Ахилл} : С превеликим удовольствием. Только, пожалуйста, без этих злодейских логических ловушек, г-жа Ч.

\emph{Черепаха} : Злодейских ловушек? Хорошо же вы обо мне думаете! Какая же я злодейка? Я мирная душа, никому не мешаю, живу спокойной травоядной жизнью. Мои мысли текут себе среди странностей и завихрений мироздания (так как я его вижу). Я, скромная наблюдательница явлений, бреду себе потихоньку и бросаю на ветер всякие глупости, которые, боюсь, никого не впечатляют. Но не волнуйтесь, Ахилл, сегодня я собиралась поговорить всего-навсего о своем панцире --- он-то уж не имеет к логике ни малейшего отношения.

\emph{Ахилл} : Вы меня НА САМОМ ДЕЛЕ успокоили, г-жа Ч. И, честно говоря, мое любопытство задето. Охотно вас послушаю, даже если это и не очень впечатляюще.

\emph{Черепаха} : Ну что ж\ldots{} с чего мне начать? Гмм\ldots{} Присмотритесь-ка к моему панцирю --- вас ничего не удивляет?

\emph{Ахилл} : Как будто почище стал?

\emph{Черепаха} : Премного благодарна. Я только что оставила в пруду несколько слоев грязи, накопившихся на мне за последнее столетие. Теперь вы можете увидеть, какой у меня зеленый панцирь!

\emph{Ахилл} : Такой крепкий, зеленый панцирь --- и как ярко он блестит на солнце!

\emph{Черепаха} : Зеленый? Он вовсе не зеленый.

\emph{Ахилл} : Вы же сами только что сказали, что ваш панцирь зеленый!

\emph{Черепаха} : Я так и сказала.

\emph{Ахилл} : В таком случае, мы согласны: он зеленый.

\emph{Черепаха} : Нет, он не зеленый.

\emph{Ахилл} : О, я понимаю: вы намекаете на то, что то, что вы говорите, не обязательно истинно, что Черепахи играют с языком, что ваши утверждения не всегда совпадают с действительностью, что\ldots{}

\emph{Черепаха} : Ничего подобного у меня и в мыслях не было! Слово для Черепах --- святыня; Черепахи преклоняются перед точностью.

\emph{Ахилл} : Хорошо, тогда почему же вы говорите, что ваш панцирь зеленый, и что он не зеленый?

\emph{Черепаха} : Никогда я ничего такого не говорила --- а жаль!

\emph{Ахилл} : Вы хотели бы это сказать?

\emph{Черепаха} : Нисколько. Я сожалею о том, что я это сказала, и совершенно с этим не согласна.

\emph{Ахилл} : Но это противоречит тому, что вы только что сказали!

\emph{Черепаха} : Противоречит? Противоречит? Я никогда себе не противоречу. Это не в черепашьем характере.

\emph{Ахилл} : Ну, на этот раз я вас поймал, хитрюга этакая! Это же самое настоящее противоречие!

\emph{Черепаха} : Вероятно, вы правы.

\emph{Ахилл} : Опять! Теперь вы противоречите себе еще больше! Вы настолько запутались в противоречиях, что с вами невозможно спорить!

\emph{Черепаха} : Вовсе нет. Я спорю сама с собой постоянно, и у меня это прекрасно получается. Может быть, дело в вас самих. Позволю себе предположить, что противоречивы именно вы --- но, поскольку вы сами себя совершенно запутали, вы не в состоянии заметить собственной непоследовательности.

\emph{Ахилл} : Какое оскорбительное предположение! Я вам покажу, что противоречите себе именно вы, и что об этом не может быть двух мнений.

\emph{Черепаха} : Что ж, если это так, Ахилл, то это дело должно быть вам по плечу. Нет ничего легче, чем указать на противоречие. Валяйте, доказывайте, Ахилл!

\emph{Ахилл} : Гмм\ldots{} Даже не знаю, с чего начать\ldots{} А! Теперь вижу. Вы сказали сначала, что (1) ваш панцирь зеленый и тут же, что (2) ваш панцирь не зеленый. Что тут добавишь?

\emph{Черепаха} : Осталось только указать на противоречие. Будьте любезны, перестаньте, наконец, ходить вокруг да около.

\emph{Ахилл} : Но\ldots{} но\ldots{} но\ldots{} О, теперь я понимаю. (Видите ли, иногда я такой тугодум!) Наверное, мы с вами по-разному понимаем противоречие. В этом-то вся загвоздка. Позвольте мне объясниться: противоречие возникает, когда кто-то утверждает одну вещь и одновременно ее отрицает.

\emph{Черепаха} : Вот ловкий трюк! Хотела бы я увидеть, как подобное возможно. Наверное, лучше всего противоречия получались бы у чревовещателей, которые могут говорить одновременно двумя сторонами рта. Но я-то не чревовещатель\ldots{}

\emph{Ахилл} : На самом деле, я имел в виду только то, что кто-то утверждает одну вещь и ее же отрицает в одном и том же предложении. Это не должно быть буквально в один и тот же момент.

\emph{Черепаха} : Однако в вашем примере не ОДНО предложение, а два!

\emph{Ахилл} : Да --- два предложения, противоречащих друг другу.

\emph{Черепаха} : Ну и путаница у вас в голове, бедняга! Сначала вы мне говорите, что противоречие --- это что-то, что должно быть в одном и том же предложении. Тут же вы утверждаете, что вы нашли противоречие в паре моих предложений. Так и есть --- ваш мыслительный процесс настолько запутан, что вы сами не видите, как вы непоследовательны. Со стороны, однако, это ясно как день.

\emph{Ахилл} : Вы меня совсем сбили с толку вашими отвлекающими маневрами. Я уже перестал понимать, идет ли речь о каких-то чепуховых мелочах или же о чем-то важном и глубоком.

\emph{Черепаха} : Уверяю вас, Черепахи не занимаются мелочами. Следовательно, верно второе.

\emph{Ахилл} : Вы меня успокоили, благодарю вас. Теперь, поразмыслив, я вижу логический шаг, необходимый, чтобы уверить вас в том, что вы противоречили себе.

\emph{Черепаха} : Отлично! Надеюсь, что этот шаг столь же легок, сколь бесспорен.

\emph{Ахилл} : Так и есть --- даже вы с ним согласитесь. Идея в том, что если вы считаете истинным предложение 1 («Мой панцирь зеленый») и предложение 2 («Мой панцирь не зеленый»), то вы должны считать истинной комбинацию этих двух предложений. Не так ли?

\emph{Черепаха} : Безусловно. Это только естественно\ldots{} если, конечно, все согласны с тем, КАК эти предложения комбинировать.

\emph{Ахилл} : Разумеется --- и тут-то я вас поймаю! Я предлагаю такую комбинацию ---

\emph{Черепаха} : С комбинированием предложений надо быть осторожнее. Разрешите мне это продемонстрировать. Наверняка, Ахилл, вы согласитесь со следующим предложением, описывающим ваш странный род:

\emph{У людей пять пальцев.}

К тому же, истинность его весьма нетрудно проверить, не так ли?

\emph{Ахилл (неуверенно)} : Соглашусь ли я? То есть, э-э, гмм\ldots{} как это я могу не согласиться с таким скучным и плоским утверждением? Минуточку\ldots{} (Смотрит себе на пальцы и бормочет.) Раз, два, три, четыре\ldots{} (Вслух, Черепахе) Г-жа Черепаха, а мизинец тоже считается за палец?

\emph{Черепаха} : Я думаю, да.

\emph{Ахилл (снова бормочет)} : Ага! Получается пять. Кажется, правильно. Я проверил все необходимые и достаточные условия истинности, так что\ldots{} (Вслух, на этот раз гораздо более уверенно): ЛЮБОЙ знает, что тривиальное суждение «у людей пять пальцев» --- истинно! Что может быть более очевидно?

\emph{Черепаха} : Разумеется. А теперь потрудитесь проверить почти такое же очевидное утверждение, а именно:

\emph{В этом предложении пять слов.}

\emph{Ахилл (бормочет себе под нос)} : Гмм\ldots{} раз\ldots{} два\ldots{} три\ldots{} четыре\ldots{} пять! Да, действительно, я должен согласиться с истинностью и этого утверждения. В ЭТОМ предложении я не вижу никаких проблем.

\emph{Черепаха} : Превосходно! Теперь, когда мои теоретические предположения получили экспериментальное подтверждение в ваших строгих исследованиях, я чувствую себя значительно лучше. Сейчас же, поскольку мы согласны по всем статьям, нам остается только соединить эти два невинных предложения в одно подлиннее, с помощью вашего безопасного слова «и».

\emph{Ахилл} : Именно «безопасного», г-жа Ч. Вам не удастся обвести меня вокруг пальца! Что ж, начнем, пожалуй\ldots{}

\emph{Черепаха} : Прекрасно. Посмотрим\ldots{} у меня получается следующее предложение, которое, безусловно, должно оказаться истинным:

\emph{У людей пять пальцев и в этом предложении пять слов.}

\emph{Ахилл} : Постойте, г-жа Ч! Что-то здесь не то!

\emph{Черепаха (всем своим видом выражая невинное удивление)} : Что? Что вы имеете в виду?

\emph{Ахилл} : Вы соединяете эти предложения неправильно!

\emph{Черепаха} : Я только последовала вашему совету и использовала ваше любимое «и».

\emph{Ахилл} : Не знаю, не знаю\ldots{} То, что у вас получилось, НЕЛОГИЧНО! Где-то здесь должна быть ошибка\ldots{}

\emph{Черепаха} : Ну вот, вы снова заговорили о г-же Логике и ее великих принципах\ldots{} Будьте так любезны, увольте --- хотя бы на сегодня.

\emph{Ахилл} : Г-жа Черепаха, у меня уже черепушка трещит от всего этого. Признайтесь, что вы немного сжульничали\ldots{}

\emph{Черепаха} : Пожалуйста, не обвиняйте меня в собственных грехах, кто из нас хотел соединить два высказывания с помощью «и».Мне кажется, я только следовала вашим пожеланиям --- и какова же ваша благодарность? Ну и молодежь нынче пошла\ldots{}

\emph{Ахилл} : Ну вот, я же и виноват. Ведь я хотел, как лучше\ldots{}

\emph{Черепаха} : Добрыми намерениями, мой юный друг, вымощена дорога в преисподнюю\ldots{}

\emph{Ахилл} : Я чувствую себя ужасно\ldots{}

\emph{Черепаха} : Я отлично понимаю, куда вы клонили: вы хотели заставить меня принять за истинную фразу «Мой панцирь зеленый и мой панцирь не~зеленый». О, создатель!\ldots{} Какая страшная ложь, и как она противна черепашьему духу!

\emph{Ахилл} : Умоляю вас простить меня, дурака\ldots{} Честное слово, у меня и в мыслях~не было вас обидеть.

\emph{Черепаха} : Ничего, мой друг --- мы, черепахи, привыкли к людской бестактности. Я ценю вашу компанию, Ахилл, пусть ваши мысли и не так кристально ясны, как у созданий нашей хладнокровной породы.

\emph{Ахилл (вздыхая)} : Надеюсь, что для меня еще не все потеряно --- хотя я, наверняка, сделаю еще немало ложных шагов на пути к истине\ldots{}

\emph{Черепаха} : Мужайтесь, Ахилл. Может быть, наша сегодняшняя беседа вам поможет\ldots{} Кстати, не забудьте отдать мне ту фигу, что вы мне принесли. Хоть она и зеленая, все равно пригодится!

\emph{Ахилл} : Вот, возьмите.

\emph{Черепаха} : Что ж, до скорого, мой друг.

\emph{Ахилл} : До скорого.


% % \subsubsection{ГЛАВА VII: Исчислени Высказываний}
% \include{fragments/xx22}
% % \subsubsection{Крабий канон}
% \subsubsection{Крабий канон}

\emph{В один прекрасный день, Ахилл и Черепаха, прогуливаясь по парку, наталкиваются друг на друга.}

\emph{Черепаха} : Приветствую, г-н А.!

\emph{Ахилл} : И я вас тоже.

\emph{Черепаха} : Всегда рада вас видеть.

\emph{Ахилл} : Вы читаете мои мысли.

\emph{Черепаха} : В такой денек приятно пройтись; пожалуй, я пойду домой пешочком.

\emph{Ахилл} : Неужели? Гулять, знаете ли, весьма полезно для здоровья.

\emph{Черепаха} : Кстати, в последнее время вы выглядите как огурчик.

\emph{Ахилл} : О, благодарю вас.

\emph{Черепаха} : Не стоит. Не желаете ли угоститься моими сигарами?

\emph{Ахилл} : Да вы, как я погляжу, филистер. По моему мнению, голландский вклад в эту область --- значительно худшего вкуса, и я хочу попытаться вас в этом убедить.

\emph{Черепаха} : Наши мнения по этому вопросу расходятся. Кстати, говоря о вкусах: несколько дней назад я была на выставке, где, наконец, увидела «Крабий канон» вашего любимого художника, М. К. Эшера. Какая красота! Как ловко он переворачивает тему задом наперед! Но боюсь, что для меня Бах всегда останется выше Эшера.

\emph{Ахилл} : Не знаю, не знаю\ldots{} Я уверен только в том, что меня не волнуют споры о вкусах. De gustibus non est disputandum.

\emph{Черепаха} : Поговорим лучше о другом. Знаете ли вы, что я уже давно пытаюсь собрать полную коллекцию редких записей Баха --- хоть это и отнимает много времени, но я считаю, что лучшего хобби не найти.

\emph{Ахилл} : Ну и волокита! Не знаю, как кому-то могут нравиться такие вещи\ldots{}

\emph{(Вдруг, откуда ни возьмись, появляется Краб. Он стремительно подбегает к друзьям, указывая на огромный синяк под глазом.)}

\emph{Краб} : Приветик! Бонжурчик! Я сегодня как огурчик, только вот синяк --- кошмар, не правда ли? Мне его наставил этот поляк, ужасный, скажу вам, пошляк. Хо! Да еще в такой чудесный денек! Я себе по парку гулял, никого не задирал; вдруг слышу --- музыка небесная, полька расчудесная. Гляжу, а на скамье сидит девица, да такая, что нам с вами не пара; а в руках у нее --- гитара. Я и сам, знаете ли, из музыкальной семьи: мой кузен рак --- мужик не дурак! --- всегда зимовал ничуть не ближе, чем в Париже. Он был придворным музыкантом короля --- услаждал его величество художественным свистом, когда тот сидел с придворными за вистом. Любовь к музыке у нас, ракообразных, в крови\ldots{} Понимаете теперь, почему я не удержался, на скамейку взобрался, и говорю на ушко девице: «Щипать струны вы, гляжу, мастерица! Позвольте мне, как музыканту, сделать вам комплимент --- а также предложить свой аккомпанемент. Чтоб польке дать полнее звук, сыграем-ка в двенадцать рук!» Она как вскочит, да как завопит, что есть мочи! Тут откуда ни возьмись, явился этот здоровяк, этот поляк\ldots{} Бах! Трах! Прямо в глаз попал --- вот откуда этот фингал! Не думайте, что я трус --- атаковать я не боюсь. Но по давней семейной традиции, крабы --- мастера защитной диспозиции\ldots{} Ведь мы, когда идем вперед, движемся назад. Это у нас в генах --- переворачивать все задом наперед. Кстати, это мне напоминает\ldots{} Я всегда спрашивал себя: «Что было раньше, Краб или Ген?» То есть, я хочу сказать: «Что было позже, Ген или Краб»? Я всегда переворачиваю все задом наперед, знаете ли --- это у нас в генах. Ведь мы, когда идем вперед, движемся назад\ldots{} Ох, и заболтался же я, друзья! Да еще в такой чудный денек, хо! Поползу себе, пожалуй. Приветик!

\emph{(И он исчезает так же внезапно, как и появился.)}

\emph{Рис. 43. Кусочек одного из Крабьих Генов. Если спирали ДНК развернуть и положить рядом, то получится следующая картина: TTTTTTTCGAAAAAAA ... AAAAAAAGTTTTTTTT... Обратите внимание на то, что спирали одинаковы - разница только в том, что одна из них идет в обратном порядке. Эта черта определяет также музыкальную форму под названием ракоход, или «крабий канон.» Очень похожи на это и палиндромы --- предложения, которые при прочтении задом наперед дают точно тот же результат. В молекулярной биологии подобные сегменты ДНК называются «палиндромами» --- ко самом деле, более точным названием было бы «крабий канон». Этот сегмент ДНК не только «крабо-каноничен» --- в его основной структуре также закодирована структура Диалога. Присмотритесь повнимательней!}

\emph{Черепаха} : Ну и волокита! Не знаю, как кому-то могут нравиться такие вещи\ldots{}

\emph{Ахилл} : Поговорим лучше о другом. Знаете ли вы, что я уже давно пытаюсь собрать полную коллекцию редких гравюр Эшера --- хоть это и отнимает много времени, но я считаю, что лучшего хобби не найти.

\emph{Черепаха} : Не знаю, не знаю\ldots{} Я уверена только в том, что меня не волнуют споры о вкусах. Disputandum non est de gustibus.

\emph{Ахилл} : Наши мнения по этому вопросу расходятся. Кстати, говоря о вкусах: несколько дней назад я был на концерте, где наконец, услышал «Крабий канон» вашего любимого композитора, И. С. Баха. Какая красота! Как ловко он переворачивает тему задом наперед! Но боюсь, что для меня Эшер всегда останется выше Баха.

\emph{Черепаха} : Да вы, как я погляжу, филистер. По моему мнению, голландский вклад в эту область --- значительно худшего вкуса, и я хочу попытаться вас в этом убедить.

\emph{Ахилл} : Не стоит. Не желаете ли угоститься моими сигарами?

\emph{Черепаха} : О, благодарю вас.

\emph{Ахилл} : Кстати, в последнее время вы выглядите как огурчик.

\emph{Черепаха} : Неужели? Гулять, знаете ли, весьма полезно для здоровья.

\emph{Ахилл} : В такой денек приятно пройтись; пожалуй, я пойду домой пешочком.

\emph{Черепаха} : Вы читаете мои мысли.

\emph{Ахилл} : Всегда рад вас видеть.

\emph{Черепаха} : И я вас тоже.

\emph{Ахилл} : Приветствую, г-жа Ч.

\emph{РИС. 44. «Крабий канон»~из «Музыкального приношения» И. С. Баха.}


% % \subsubsection{ГЛАВА VIII: Типографская теория чисел}
% \include{fragments/xx24}
% % \subsubsection[Приношение «МУ»]{\texorpdfstring{Приношение «МУ»\footnote{Все коаны в этом Диалоге подлинны, они взят из следующих двух книг Paul Reps «Zen Flesh Zen Bones» и Gyomay M. Kubose «Zen Koans»}}{Приношение «МУ»}}
% \subsubsection[Приношение «МУ»]{\texorpdfstring{Приношение «МУ»\footnote{Все коаны в этом Диалоге подлинны, они взят из следующих двух книг Paul Reps «Zen Flesh Zen Bones» и Gyomay M. Kubose «Zen Koans»}}{Приношение «МУ»}}

\emph{Черепаха и Ахилл только что вернулись с лекции о происхождении Генетического Кода; они сидят у Ахилла и пьют чай.}

\emph{Ахилл} : Я должен кое в чем признаться, г-жа Ч.

\emph{Черепаха} : Что такое?

\emph{Ахилл} : Несмотря на интереснейшую тему, я пару раз задремал\ldots{} Но даже во сне я кое-что слышал. Вот какая странная мысль всплыла из глубины моего сознания: «А» и «Т» могут обозначать не «аденин» и «тимин», а мое и ваше имена! Ведь вас зовут Тортилла! Кроме того, в моем полусне вдоль остова двойной спирали ДНК были подвешены крохотные Ахиллы и Тортиллы, всегда в парах, как аденин и тимин. Правда, странный образ?

\emph{Черепаха} : Фу! Кто верит в подобные глупости? К тому же, что вы скажете о «С» и «G»?

\emph{Ахилл} : Что ж, цитозин мог бы обозначать г-на Краба --- ведь его имя пишется «Crab». Насчет «G» я не знаю, но уверен, что можно было бы что-нибудь придумать. Так или иначе, было забавно вообразить мою ДНК, полную ваших малюсеньких копий --- и моих, конечно. Только подумайте, к какой бесконечной регрессии это бы привело!

\emph{Черепаха} : Вижу, что вы не очень-то внимательно слушали лекцию.

\emph{Ахилл} : Неправда --- я старался изо всех сил. Просто было очень трудно отделить мои фантазии от фактов. В конце концов, молекулярные биологи изучают такой необыкновенный нижний мир\ldots{}

\emph{Черепаха} : Что вы имеете в виду?

\emph{Ахилл} : Молекулярная биология полна странных спиральных петель, которые я как следует не понимаю. Например, белки, закодированные в ДНК, могут «провернуться назад» и повлиять на саму ДНК --- даже разрушить ее. Подобные странные петли меня всегда запутывают. В них есть что-то пугающее.

\emph{Черепаха} : Я нахожу их весьма привлекательными.

\emph{Ахилл} : Разумеется --- они вполне в вашем вкусе. Но мне иногда хочется прекратить весь этот анализ и просто помедитировать немного, в качестве противоядия. Это очищает голову от путаницы странных петель и всех этих невероятных сложностей, о которых мы сегодня услышали.

\emph{Черепаха} : Удивительно! Никогда бы не подумала, что вы медитируете.

\emph{Ахилл} : Разве я никогда не говорил вам, что изучаю дзен-буддизм?

\emph{Черепаха} : Боже мой, как вы до этого додумались?

\emph{Ахилл} : Мне всегда казалось, что без инь и янь мое дело --- дрянь; знаете, все эти путешествия в восточный мистицизм, И-Чинг, гуру, и тому подобное. В одни прекрасный день я подумал: «Почему бы мне не заняться и дзеном?» Так это все и началось.

\emph{Черепаха} : Превосходно! Может быть и я, наконец, сподоблюсь просветиться.

\emph{Ахилл} : Ну-ну, не так быстро. Просветление --- совсем не первый шаг на пути к буддизму; скорее, это последний шаг. Просветление не для таких новичков, как вы, г-жа Ч!

\emph{Черепаха} : Вы меня не поняли. Я не имела в виду буддистское просветление --- мне просто хотелось узнать, что такое дзен-буддизм.

\emph{Ахилл} : Бог ты мой, что же вы сразу не сказали? Я буду очень рад рассказать вам все, что знаю о дзене. Может быть, вам даже захочется стать учеником буддизма, таким же, как и я.

\emph{Черепаха} : Что ж, нет ничего невозможного.

\emph{Ахилл} : Вы можете изучать буддизм вместе со мной у моего Мастера Оканисамы --- седьмого патриарха.

\emph{Черепаха} : Черт меня побери, если я что-нибудь понимаю!

\emph{Ахилл} : Чтобы это понять, необходимо знать историю дзен-буддизма.

\emph{Черепаха} : В таком случае, не расскажете ли вы мне немного об истории дзена?

\emph{Ахилл} : Отличная мысль. Дзен --- это тип буддизма; он был основан монахом по имени Бодхидхарма, который оставил Индию и поселился в Китае. Это было в шестом веке. Бодхидхарма был первым патриархом. Шестым патриархом был\ldots{} э-э-э\ldots{} проклятый склероз\ldots{} Энон! (Наконец-то вспомнил!)

\emph{Черепаха} : Неужели Зенон? Как странно, что именно он оказался замешанным в таком деле.

\emph{Ахилл} : Осмелюсь заметить, что вы недооцениваете значимость дзена. Послушайте еще немного и, может быть, вы будете относиться к нему с большим уважением. Так вот, как я говорил, примерно пятьсот лет спустя дзен пришел в Японию, где он прекрасно прижился. С того времени он стал одной из основных религий Японии.

\emph{Черепаха} : Кто такой этот Оканисама, «седьмой патриарх»?

\emph{Ахилл} : Он мой Мастер, и его учение прямо следует из учения шестого патриарха. Он научил меня тому, что действительность --- едина и неизменна; вся множественность, изменения и движение --- не более, чем иллюзии наших чувств.

\emph{Черепаха} : Точно --- это за километр пахнет дзеном. Но как же он впутался в дзен, бедняга?

\emph{Ахилл} : Что-о? Если КТО-ТО и запутался, то это\ldots{} Ну ладно, это уже другой разговор. Так или иначе, я не знаю ответа на ваш вопрос. Вместо этого я вам лучше расскажу еще что-нибудь из поучений моего Мастера. Я узнал, что в дзене человек ищет Просветления, или САТОРИ --- состояния «He-разума». В этом состоянии человек не думает о мире --- он просто СУЩЕСТВУЕТ. Я также узнал, что изучающий дзен не должен «привязывать» себя ни к какому объекту, или мысли, или человеку --- то есть, он не должен верить ни в какой абсолют и не должен зависеть от чего-либо, включая и саму эту философию не-привязанности.

\emph{Черепаха} : Г-мм\ldots{} Это уже КОЕ-ЧТО; дзен начинает мне нравиться.

\emph{Ахилл} : У меня было предчувствие, что вы сразу к нему привяжетесь.

\emph{Черепаха} : Но скажите мне: если дзен отрицает интеллектуальную деятельность вообще, то какой смысл размышлять о нем и усердно его изучать?

\emph{Ахилл} : Мне тоже не давала покоя эта мысль. Но думаю, что я, наконец, нашел ответ: к дзену можно подходить по любой дороге, даже если эта дорога кажется ведущей совершенно в другую сторону. По мере того, как вы к нему приближаетесь, вы учитесь отходить от дороги в сторону; и чем больше вы отходите в сторону, тем ближе вы подходите к дзену.

\emph{Черепаха} : Теперь все кажется совсем простым.

\emph{Ахилл} : Моя любимая дорога к дзену проходит через его короткие, интересные и странные притчи, под названием «коаны».

\emph{Черепаха} : Что это такое --- коан?

\emph{Ахилл} : Коан --- это история о Мастерах дзена и их учениках. Иногда он в форме загадки, иногда --- басни, а иногда коан совершенно не похож ни на что, слышанное вами раньше.

\emph{Черепаха} : Звучит интригующе. Вы думаете, что читать коаны и наслаждаться ими значит заниматься дзен-буддизмом?

\emph{Ахилл} : Сомневаюсь. Однако мне кажется, что получать удовольствие от коанов в миллион раз ближе к настоящему дзену, чем читать об этой религии том за томом, написанные на тяжелом философском жаргоне.

\emph{Черепаха} : Хотелось бы услышать какой-нибудь коан.

\emph{Ахилл} : С удовольствием расскажу вам парочку. Я должен, пожалуй, начать с самого знаменитого. Итак, много столетий тому назад жил Мастер дзен-буддизма по имени Джошу, который дожил до 119 лет.

\emph{Черепаха} : Просто юнец!

\emph{Ахилл} : С вашей точки зрения, конечно. Так вот, однажды, когда Джошу и другой монах стояли вместе в монастыре, мимо пробежала собака. Монах спросил Джошу: «У этого дога --- природа Будды?»

\emph{Черепаха} : Непонятно. Так что же ответил монах?

\emph{Ахилл} : МУ.

\emph{Черепаха} : МУ? Что это за «МУ» такое? А как же насчет собаки? И природы Будды? Как же ответ?

\emph{Ахилл} : Но ведь «МУ» и есть ответ Джошу! Говоря «МУ», Джошу дал понять другому монаху, что только воздерживаясь от подобных вопросов, можно получить на них ответ.

\emph{Черепаха} : Джошу «развопросил» этот вопрос.

\emph{Ахилл} : Именно!

\emph{Черепаха} : Это «МУ» --- весьма полезная штучка. Иногда мне тоже хочется развопросить кое-какие вопросы. Кажется, я начинаю ухватывать суть дзена\ldots{} Вы знаете еще какие-нибудь коаны, Ахилл? Мне хотелось бы услышать еще несколько.

\emph{Ахилл} : Охотно. Я знаю парочку коанов, которые всегда рассказываются вместе. Только\ldots{}

\emph{Черепаха} : Что такое?

\emph{Ахилл} : Дело в том, что мой Мастер предупреждал меня, что только один из них настоящий. Хуже того, он не знает, какой из них подлинный, а какой --- фальшивка.

\emph{Черепаха} : С ума сойти! Расскажите-ка их мне, чтобы мы могли наугадываться всласть!

\emph{Ахилл} : Хорошо. Один из коанов таков:

\emph{Один монах спросил Басо: «Что такое Будда?»}

\emph{Басо ответил: «Этот разум --- Будда.»}

\emph{Черепаха} : Гмм\ldots{} «Этот разум --- Будда»? Иногда мне трудно понять, что хотят сказать эти дзен-буддисты.

\emph{Ахилл} : Тогда второй коан может понравиться вам больше.

\emph{Черепаха} : Что это за коан?

\emph{Ахилл} : Вот он:

\emph{Один монах спросил Басо: «Что такое Будда?»}

\emph{Басо ответил: «Этот разум --- не Будда.»}

\emph{Черепаха} : Ну и ну! Как если бы мой панцирь был зеленый и не зеленый! Это мне нравится!

\emph{Ахилл} : Однако, г-жа Т, коаны совсем не предназначены для того, чтобы просто «нравиться».

\emph{Рис. 45. М. К. Эшер «Мечеть» (черные и белые мелки, 1936)}

\emph{Черепаха} : Отлично, в таком случае это мне не нравится.

\emph{Ахилл} : Так-то лучше. Так вот, как я говорил, мой мастер считает, что только один из них --- настоящий.

\emph{Черепаха} : Не могу себе представить, что заставило его так решить. Все равно этот вопрос чисто академический, поскольку невозможно узнать, какой из двух коанов --- оригинал, а какой --- подделка.

\emph{Ахилл} : Вы ошибаетесь: мой Мастер научил нас, как это сделать.

\emph{Черепаха} : Неужели? Разрешающий алгоритм для установления подлинности коанов? Хотелось бы мне услышать об ЭТОМ.

\emph{Ахилл} : Это довольно сложный ритуал: в нем два этапа. На первом этапе вы должны ТРАНСЛИРОВАТЬ данный коан в цепочку, уложенную спиралью в трех измерениях.

\emph{Черепаха} : Забавная штучка. А как насчет второго этапа?

\emph{Ахилл} : Ну, это совсем просто: надо всего-навсего определить, имеет цепочка природу Будды или нет! Если у нее --- природа Будды, то коан --- подлинный, а если нет, то он --- фальшивка.

\emph{Черепаха} : Гмм\ldots{} Это звучит так, словно вы только перенесли нужду в разрешающей процедуре в другую область. ТЕПЕРЬ вам нужна разрешающая процедура для определения природы Будды. Что же дальше? В конце концов, если вы не можете сказать даже того, буддистская ли природа у СОБАКИ,~~как же вы собираетесь определить это для любого кусочка цепочки трехмерной укладки?

\emph{Ахилл} : Мой мастер объяснил мне, что переход из одной области в другую может помочь. Это похоже на перемену точки зрения. Некоторые вещи выглядят сложными под одним углом, но простыми под другим. Он привел в пример сад: глядя на него с одной стороны, вы не видите никакого порядка, только под некоторыми углами перед вами возникает прекрасная упорядоченность. Вы организовали информацию иначе, взглянув на вещи с иной точки зрения.

\emph{Черепаха} : Понятно. В таком случае, может оказаться, что подлинность коана спрятана в нем где-то глубоко, но когда вам удается перевести его в цепочку, она каким-то образом всплывает на поверхность?

\emph{Ахилл} : Именно это и открыл мой Мастер.

\emph{Черепаха} : В таком случае, мне бы хотелось узнать об этой технике побольше. Но сперва скажите мне, как вы можете превратить коан (последовательность слов) в уложенную в пространстве цепочку (трехмерный объект)? Ведь это довольно разные классы предметов.

\emph{Ахилл} : Это как раз одна из наиболее таинственных вещей, которые я узнал, изучая дзен. Есть два шага: «транскрипция» и «трансляция». Сделать транскрипцию коана --- значит записать его фонетическим алфавитом, который содержит только четыре геометрических символа. Эта фонетическая транскрипция коана называется ПОСРЕДНИКОМ.

\emph{Черепаха} : Как выглядят эти геометрические символы?

\emph{Ахилл} : Они состоят из гексагонов и пентагонов; вот так (берет лежащую рядом салфетку и набрасывает следующие четыре фигуры):

\emph{Черепаха} : Выглядит загадочно.

\emph{Ахилл} : Только для непосвященных. Теперь, когда посредник готов, вы натирайте руки рибосом, и\ldots{}

\emph{Черепаха} : Рибосом? Это что, ритуальная мазь?

\emph{Ахилл} : Не совсем. Это специальный клейкий состав, который помогает цепочке сохранять форму, когда она уложена.

\emph{Черепаха} : Из чего он сделан?

\emph{Ахилл} : Точно не знаю, но он клейкий на ощупь и прекрасно работает. Так или иначе, когда вы натерли руки рибосом, вы можете транслировать последовательность символов в посреднике в некий тип укладки цепочки. Как видите, все очень просто.

\emph{Черепаха} : Подождите! Не так быстро! Как вы это делаете?

\emph{Ахилл} : Вы берете прямую цепочку и начинаете укладывать ее с одного конца, в соответствии с геометрическими символами посредника.

\emph{Черепаха} : Значит, каждый из этих символов обозначает особый тип укладки?

\emph{Ахилл} : Сам по себе нет. Они всегда берутся группами по три. Вы начинаете с одного конца цепочки и с одного конца посредника. Первая тройка символов определяет, что делать с первым дюймом цепочки. Следующие три символа говорят вам, как укладывать второй дюйм. Таким образом, вы шаг за шагом продвигаетесь вдоль цепочки и вдоль посредника, укладывая~~каждый крохотный сегмент цепочки, пока посредник не кончится Если вы хорошенько смазали все рибосом, цепочка сохранит свою укладку и у вас получится трансляция коана в цепочку.

\emph{Черепаха} : Эта процедура не лишена элегантности. Наверное, у вас получаются чертовски интересные цепочки.

\emph{Ахилл} : Еще бы! Коаны подлиннее транслируются в весьма причудливые структуры.

\emph{Черепаха} : Могу себе представить. Но чтобы транслировать посредник в цепочку вы должны знать, какой укладке соответствует каждая тройка геометрических символов. Откуда вы это знаете? У вас что, есть словарь?

\emph{Ахилл} : Да --- это замечательная книга, в которой приведен весь Геометрический Код. Если у вас этой книги нет, то, разумеется, вы не можете транслировать коаны в цепочки.

\emph{Черепаха} : Разумеется нет. Каково происхождение Геометрического Кода?

\emph{Ахилл} : Его начало восходит к древнему Мастеру по имени Великий Ментор, мой Мастер говорит, что он единственный, кто когда-либо достиг Архи-просветления.

\emph{Черепаха} : Ах ты батюшки! Словно одного уровня мало Что ж, обжоры бывают всех сортов --- почему бы не обжираться и просветлением?

\emph{Ахилл} : А что, если в слове Архи-просветление закодировано мое имя? А-Х-И-Л.

\emph{Черепаха} : По моему мнению, это маловероятно. Скорее, там можно найти намек на имя скромной ЧерепАХИ.

\emph{Ахилл} : При чем здесь вы? Вы даже не достигли ПЕРВОГО состояния просветления, и уж тем более\ldots{}

\emph{Черепаха} : Почем знать, почем знать. Может быть те, кто изучил всю подноготную просветления возвращаются в первоначальное, допросветленное состояние Я всегда считала, что дважды просветленный --- это снова непросветленный. Но вернемся же к нашему Великому Ментору.

\emph{Ахилл} : О нем известно очень мало --- пожалуй, только то, что он изобрел Искусство Дзен-Цепочек.

\emph{Черепаха} : Что это такое?

\emph{Ахилл} : Это искусство на котором основана разрешающая процедура для определения буддистской природы. Я могу рассказать вам об этом поподробнее.

\emph{Черепаха} : Буду счастлива. Новичкам вроде меня так много приходится выучить!

\emph{Ахилл} : Говорят, что был даже специальный коан, повествующий о том, с чего началось Искусство Дзен-Цепочек. Но, к несчастью, он уже давным-давно уплыл по течению реки времен --- а она, как известно, уносит навечно. Впрочем может быть это и неплохо --- а то нашлись бы имитаторы, которые стали бы всячески копировать Мастера, пользуясь его именем.

\emph{Черепаха} : Разве плохо, если бы все ученики дзен-будцизма стали бы копировать Великого Ментора --- самого просветленного Мастера всех времен?

\emph{Ахилл} : Позвольте вместо ответа рассказать вам коан об имитаторе.

\emph{Мастер дзена по имени Гутей всегда поднимал палец когда его спрашивали о дзене. Молоденький ученик стал его копировать. Когда Гутей услышал об имитаторе, он позвал ученика и спросил правда ли это. «Да» --- признался тот. Тогда Гутей спросил его понимает ли он, что делает. Вместо ответа ученик поднял указательный палец. Гутей быстро отрезал палец, вопя от боли ученик побежал к двери. Когда он достиг выхода Гутей позвал его: «Мальчик!» Ученик обернулся, и Гутей поднял свой указательный палец. В этот момент юноша достиг Просветления.}

\emph{Черепаха} : Кто бы мог подумать! Как раз когда я решил, что дзен --- весь о Джошу и его проказах, оказалось, что и Гутей приглашен на праздник. Кажется, у него порядочное чувство юмора.

\emph{Ахилл} : Этот коан совершенно серьезен; не знаю, откуда у вас появилась мысль, что в нем какой-то юмор.

\emph{Черепаха} : Может быть, дзен так поучителен именно потому, что в нем много юмора. Мне кажется, что если воспринимать эти истории на полном серьезе, то в половине случаев их смысл пройдет мимо вас.

\emph{Ахилл} : Может быть, в этом Черепашьем Дзене и есть какой-то смысл.

\emph{Черепаха} : Можете ли вы ответить мне на один вопрос? Я хочу знать, почему Бодхидхарма приехал из Индии в Китай.

\emph{Ахилл} : Ого! Хотите, я вам скажу, что ответил Джошу на точно такой же вопрос?

\emph{Черепаха} : О, да!

\emph{Ахилл} : Он ответил: «Дуб в саду.»

\emph{Черепаха} : Разумеется; я сказала бы то же самое. С той разницей, что в моем случае это был бы ответ на другой вопрос: «Какое место лучше всего подходит, чтобы укрыться от полуденного солнца?»

\emph{Ахилл} : Вы, сами того не подозревая, затронули сейчас один из основных вопросов дзена. Вопрос звучит безобидно: «Каков основной принцип дзена?»

\emph{Черепаха} : Удивительно! Я и понятия не имела, что основная цель дзен-буддизма --- в том, чтобы найти место в тенёчке.

\emph{Ахилл} : Да нет же, вы меня совершенно не поняли. Я не имел в виду ЭТОТ вопрос. Я думал о первом вашем вопросе --- почему Бодхидхарма приехал из Индии в Китай.

\emph{Черепаха} : Понятно. Я и не знала, что ныряю на такую глубину\ldots{} Но вернемся к этим странным отображениям. Значит, любой коан может быть превращен в уложенную цепочку, следуя этому методу. А как насчет обратного процесса? Можно ли прочитать любую цепочку так, чтобы получился коан?

\emph{Ахилл} : В некотором роде. Однако\ldots{}

\emph{Черепаха} : Что такое?

\emph{Ахилл} : Вы просто не должны читать ее таким образом. Это нарушило бы Центральную Догму Дзен-цепочек, которую можно нарисовать следующим образом (рисует на салфетке):

коан~~~~ ~=\textgreater~~~~~~ ~посредник~~~ =\textgreater~~~ ~ уложенная цепочка

.~~~~ транскрипция~~~~~~~~~ ~трансляция

Идти против стрелок нельзя --- особенно против второй стрелки.

\emph{Черепаха} : Скажите мне: у этой догмы --- природа Будды, или нет? Впрочем, если подумать, то я, пожалуй, могу развопросить этот вопрос. Если вы, конечно, не возражаете\ldots{}

\emph{Ахилл} : Буду только рад. Я хочу открыть вам один секрет --- поклянитесь, что никому не скажете!

\emph{Черепаха} : Слово Черепахи.

\emph{Ахилл} : Иногда я все-таки двигаюсь против стрелок. Запретный плод сладок, знаете ли\ldots{}

\emph{Черепаха} : Ай да Ахилл! Понятия не имела, что вы способны на такие непочтительные действия!

\emph{Ахилл} : Я никому в этом не признавался --- даже Оканисаме.

\emph{Черепаха} : Так скажите мне, что получается, когда вы двигаетесь против стрелок Центральной Догмы? Это значит, что вы начинаете с цепочки и кончаете коаном?

\emph{Ахилл} : Иногда --- но часто случаются всякие странные вещи.

\emph{Черепаха} : Более странные, чем производство коанов?

\emph{Ахилл} : Да\ldots{} Когда вы делаете трансляцию и транскрипцию наоборот, у вас получается НЕЧТО, что не всегда является коаном. Некоторые цепочки, когда их читаешь вслух таким образом, звучат сплошной бессмыслицей.

\emph{Черепаха} : Разве это не синоним коана?

\emph{Ахилл} : Вижу, моя дорогая, что вы еще не прониклись подлинным духом дзена.

\emph{Черепаха} : По крайней мере, у вас хотя бы получаются рассказы?

\emph{Ахилл} : Не всегда; иногда выходят бессмысленные слоги, иногда --- предложения-окрошка. Но иногда выходит что-то, похожее на коан.

\emph{Черепаха} : Только ПОХОЖЕЕ?

\emph{Ахилл} : Видите ли, это может оказаться подделкой.

\emph{Черепаха} : Ах, разумеется.

\emph{Ахилл} : Я называю такие цепочки, которые производят коаны, «правильно сформированными.»

\emph{Черепаха} : А как вы отличаете поддельные коаны от подлинных?

\emph{Ахилл} : К этому я и веду. Имея коан (или не-коан, как иногда случается), первое, что надо сделать, --- это транслировать его в трехмерную цепочку. Потом остается только выяснить, буддистская ли природа у этой цепочки.

\emph{Черепаха} : Как же можно ухитриться проделать подобное?

\emph{Ахилл} : Мой Мастер говорит, что Великий Ментор мог узнать это, просто взглянув на цепочку.

\emph{Черепаха} : А если вы еще не достигли Архи-просветления? Есть ли иной способ узнать, буддистская ли природа у данной цепочки?

\emph{Ахилл} : Да, есть. Здесь как раз вступает в игру Искусство Дзен-цепочек. Этот способ --- создание бесконечного множества цепочек с буддистской природой.

\emph{Черепаха} : Да что вы говорите! А есть ли способ произвести цепочки БЕЗ буддистской природы?

\emph{Ахилл} : Зачем это вам?

\emph{Черепаха} : Я просто думала --- а вдруг это может пригодиться\ldots{}

\emph{Ахилл} : У вас весьма странный вкус. Надо же! Ей интереснее вещи не-буддистской природы, чем вещи с природой Будды!

\emph{Черепаха} : Можете приписать это моему непросветленному состоянию.

\emph{Ахилл} : Итак, сначала вы вешаете петлю цепочек на руки в одной из пяти дозволенных начальных позиций; например, вот так\ldots{} \emph{(Снимает длинную цепочку, висящую у него на шее, и надевает ее на руки, набрасывая петли между пальцами.)}

\emph{Черепаха} : Что представляют собой дозволенные позиции?

\emph{Ахилл} : Каждая из них --- это позиция, считающаяся самоочевидным способом брать цепочку. Даже новички часто берут цепочки именно так. И все эти пять цепочек имеют природу Будды.

\emph{Черепаха} : Разумеется.

\emph{Ахилл} : Кроме того, имеются некоторые Правила Обращения с Цепочками, следуя которым, можно произвести из цепочек более сложные фигуры. В частности, позволено изменять форму вашей цепочки при помощи простейших движений рук. Например, вы можете взяться за эту цепочку здесь и потянуть вот так --- а теперь так перекрутить. Каждая операция меняет конфигурацию цепочки, надетой на ваши руки.

\emph{Черепаха} : Это выглядит, как игра в веревочку --- «колыбель для кошки» и прочие занимательные фигуры, которые можно сплести из веревки, надетой на пальцы.

\emph{Ахилл} : Верно. Смотрите, некоторые из этих правил усложняют цепочку, а некоторые упрощают. Но неважно, в каком порядке вы это делаете; пока вы следуете Правилам Обращения с Цепочками, любая ваша цепочка будет иметь природу Будды.

\emph{Черепаха} : Это чудесно. А как насчет коана, спрятанного в строчке, что вы только что сплели? Будет ли он подлинным?

\emph{Ахилл} : Согласно тому, что я выучил, именно так и будет. Поскольку я придерживался Правил и начал в одной из пяти самоочевидных позиций, цепочка должна иметь природу Будды и, следовательно, соответствовать подлинному коану.

\emph{Черепаха} : Знаете ли вы, какому именно?

\emph{Ахилл} : Вы хотите, чтобы я нарушил Центральную Догму? Ах вы, вредное создание!

\emph{(Ахилл раскрывает книгу Кода и, высунув от усердия язык, дюйм за дюймом продвигается вдоль цепочки, записывая каждый поворот с помощью тройки геометрических символов этого странного фонетического алфавита для коанов, пока салфетка не оказывается исписанной его каракулями)}

Готово!

\emph{Черепаха} : Здорово! Теперь давайте почитаем, что получилось.

\emph{Ахилл} : Хорошо.

\emph{Путешествующий монах спросил у старухи дорогу к Тайзаиу, известному храму, превращающему тех, кто в нем молится, в мудрецов. Старуха ответила: «Идите прямо». Когда тот удалился, старуха пробормотала себе под нос: «Еще один паломник». Кто-то рассказал об этом случае Джошу, и тот заметил: «Подождите, я сам проверю». На следующий день он отправился тем же путем и задал тот же вопрос. Старуха повторила свой ответ, и Джошу сказал: «Я проверил эту старую женщину».}

\emph{Черепаха} : С его страстью к расследованиям, жаль, что Джошу никогда не работал в ФБР. А скажите, я могла бы повторить то, что вы сейчас сделали, если бы следовала Правилам Искусства Дзен-цепочек, не правда ли?

\emph{Ахилл} : Совершенно верно.

\emph{Черепаха} : Я должна буду проделывать все операции в том же ПОРЯДКЕ, как и вы?

\emph{Ахилл} : Да нет, годится любой порядок.

\emph{Черепаха} : Разумеется, тогда я получу другую цепочку и, следовательно, другой коан. Теперь скажите мне, я должна буду повторить то же ЧИСЛО операций?

\emph{Ахилл} : Ни в коем случае. Вы можете делать любое число шагов.

\emph{Черепаха} : В таком случае, есть бесконечное множество цепочек с природой Будды --- а следовательно, бесконечное множество подлинных коанов! Но откуда вы знаете, есть ли какая-либо цепочка, которая НЕ МОЖЕТ быть получена при помощи ваших Правил?

\emph{Ахилл} : Ах, да --- вернемся к вещам, лишенным природы Будды. Получается так, что как только вы научитесь производить цепочки БУДДИСТСКОЙ природы, вы сразу же научитесь производить и HE-БУДДИСТСКИЕ цепочки. Это мой Мастер вдолбил в меня с самого начала.

\emph{Черепаха} : Прекрасно! Как же это получается?

\emph{Ахилл} : Очень просто. Вот, смотрите: сейчас я сделаю цепочку, у которой нет природы Будды\ldots{}

\emph{(Он берет цепочку, из которой был «извлечен» предыдущий коан, и завязывает на одном из концов неточку, затягивая ее большим и указательным пальцами.)}

Готово: в этой цепочке НЕТ никакой буддистской природы.

\emph{Черепаха} : Потрясающе! Я просвещаюсь с каждой минутой. И всего-то понадобилась какая-то ниточка? Откуда вы знаете, что у новой цепочки нет буддистской природы?

\emph{Ахилл} : Не ниточка, а НЕТОЧКА --- именно так указал мастер. Основное свойство природы Будды таково: если две правильно сформированные цепочки отличаются только тем, что одна из них имеет неточку на конце, то только ОДНА из этих цепочек может иметь буддистскую природу.

\emph{Черепаха} : А скажите: есть ли такие цепочки буддистской природы, которые НЕВОЗМОЖНО получить, в каком бы порядке мы не применяли Правила Дзен-цепочек?

\emph{Ахилл} : Стыдно признаться, но этого я сам точно не знаю. Сначала мой мастер говорил, что буддистская природа цепочки ОПРЕДЕЛЕНА тем, что мы начинаем с одной из пяти начальных позиций и затем строго следуем Правилам. Но позже он сказал что-то о какой-то «Теореме», как бишь его\ldots{} Гоголя?., или Де Голля? Боюсь, что я так этого и не понял; а может быть, просто не расслышал. Но так или иначе, у меня появилось сомнение, можно ли получить этим методом ВСЕ цепочки с природой Будды. До сих пор мне это удавалось, но ведь буддистская природа --- штука непростая, знаете ли\ldots{}

\emph{Черепаха} : Я так и думала, судя по «МУ» Джошу. Хотелось бы мне знать\ldots{}

\emph{Ахилл} : Что такое?

\emph{Черепаха} : Я думала о тех двух коанах\ldots{} Я имею в виду, коан и не-коан: «Этот разум --- Будда» и «Этот разум --- не Будда». Как они выглядят, если перевести их в цепочки по Геометрическому Коду?

\emph{Ахилл} : С удовольствием вам покажу.

\emph{(Он записывает фонетическую транскрипцию, достает из кармана пару цепочек и начинает аккуратно, дюйм за дюймом, складывать их, следуя тройкам символов, записанных странным алфавитом. Затем он кладет получившиеся цепочки рядом.)}

Видите, они различаются.

\emph{Черепаха} : На мой взгляд, они весьма схожи. О, теперь я вижу, в чем разница: на конце у одной из них --- неточка!

\emph{Ахилл} : Клянусь Джошу, вы правы.

\emph{Черепаха} : Ага! Я понимаю теперь, почему ваш Мастер не доверял этим коанам.

\emph{Ахилл} : Неужели?

\emph{Черепаха} : Согласно его указаниям, НЕ БОЛЕЕ, ЧЕМ ОДНА цепочка из этой пары может иметь природу Будды; так что сразу можно сказать, что один из коанов --- подделка.

\emph{Ахилл} : Но это еще не говорит нам, какой именно. Мы с моим Мастером давно пытаемся сложить эти цепочки, следуя Правилам; но у нас пока ничего не выходит. Это ужасно неприятно, и можно начать сомневаться\ldots{}

\emph{Черепаха} : В том, что у этих цепочек вообще есть природа Будды? Может быть, ее нет ни у одной цепочки, и оба коана поддельны?

\emph{Ахилл} : Я никогда не заходил так далеко --- но вы правы, в принципе это возможно. Однако вы не должны задавать так много вопросов о природе Будды. Мастер дзена Мумон всегда предупреждал своих учеников, что слишком много спрашивать опасно.

\emph{Черепаха} : Хорошо --- вопросов больше не будет. Но зато мне очень хочется самой уложить цепочку. Интересно посмотреть, получится ли она правильно сформированной.

\emph{Ахилл} : И правда, интересно. Вот, пожалуйста. \emph{(Передает цепочку Черепахе.)}

\emph{Черепаха} : Вы знаете, я понятия не имею, что с ней делать. Что ж, рискнем --- мое неуклюжее произведение, сделанное без Правил, как Бог на душу положит, будет, скорее всего, совершенно невозможно расшифровать. \emph{(Берет цепочку, делает из нее петлю, и несколькими движениями лап укладывает цепочку в сложный узор, который затем молча протягивает Ахиллу. В этот момент лицо воина освещается.)}

\emph{Ахилл} : Вот это да! Я должен попробовать этот метод сам. Никогда не видел подобной цепочки!

\emph{Черепаха} : Надеюсь, что она правильно сформирована.

\emph{Ахилл} : На одном конце у нее завязана неточка.

\emph{Черепаха} : Ох, погодите --- можно мне эту цепочку на минутку? Я хочу еще кое-что сделать.

\emph{Ахилл} : Почему бы~и нет --- пожалуйста.

\emph{(Снова протягивает ее Черепахе, та завязывает еще одну неточку на том же конце. После этого она встряхивает цепочку и внезапно обе неточки исчезают!)}

\emph{Ахилл} : Что случилось?

\emph{Черепаха} : Я просто хотела избавиться от той неточки.

\emph{Ахилл} : Но вместо того, чтобы ее развязать, вы завязали еще одну, и тут их как ножом отрезало, обе исчезли! Куда они подевались?

\emph{Черепаха} : В Лимбедламию, разумеется. Это Закон Двойного Отрезания.

\emph{(Вдруг обе неточки опять появляются ниоткуда --- то бишь, из Лимбедламии.)}

\emph{Ахилл} : Удивительно. К некоторым районам Лимбедламии, видно, существует легкий доступ, если эти неточки могут так запросто проталкиваться и выталкиваться. Или же вся Лимбедламия одинаково недоступна?

\emph{Черепаха} : Не могу вам сказать. Правда, я думаю, что если бы мы эту цепочку расплавили, то неточки вряд ли вернулись бы. В этом случае, мы считали бы, что они попали на более глубокий уровень Лимбедламии. Там, возможно, есть миллионы уровней. Но это для нас неважно. Меня сейчас интересует то, как эта цепочка зазвучит, если мы переведем ее обратно в фонетические символы.

\emph{Ахилл} : Я всегда чувствую себя виноватым, когда нарушаю Центральную Догму.

\emph{(Достает ручку и книгу Кода и аккуратно записывает тройные символы, соответствующие поворотам Черепашьей цепочки; когда все готово, он откашливается.)} Кхе-кхе. Послушаем, что у вас получилось\ldots{}

\emph{Черепаха} : Если вы готовы\ldots{}

\emph{Ахилл} : Отлично. Вот что тут написано:

\emph{Один монах постоянно приставал к Великой Чепупахе (единственной, которая когда-либо достигла Архи-просветлеиия), спрашивая у нее, имеют ли те или иные вещи природу Будды. Чепупаха отвечала на эти вопросы молчанием. Монах уже спросил о бобе, озере, и лунной ночи. Однажды он принес Чепупахе кусочек цепочки и задал тот же вопрос. В ответ Чепупаха взяла цепочку, сделала из нее петлю и несколькими движениями лап ---}

\emph{Черепаха} : Несколькими движениями лап? Как странно!

\emph{Ахилл} : Почему же именно Вы находите это странным?

\emph{Черепаха} : Ах да, конечно, вы правы. Продолжайте, прошу вас!

\emph{Ахилл} : Хорошо.

\emph{Несколькими движениями лап Чепупаха уложила цепочку в сложный узор, который затем молча протянула монаху. В этот момент монах достиг Просветления.}

\emph{Черепаха} : Что до меня, то я бы предпочла Архи-просветление.

\emph{Ахилл} : Далее тут описывается, как сделать цепочку Великой Чепупахи, если начать с петли, наброшенной на лапы. Эти скучные детали я пропущу\ldots{} А вот и конец:

\emph{С тех пор монах больше не приставал к Чепупахе. Вместо этого он укладывал цепочку за цепочкой по ее методу; он передал этот метод своим ученикам, а те --- своим.}

\emph{Черепаха} : Ну и хитросплетение! Трудно поверить, что все это было спрятано в моей цепочке.

\emph{Ахилл} : Так оно и есть. Удивительно, что вы смогли уложить правильно сформированную цепочку --- верно говорят, что новичкам везет!

\emph{Черепаха} : Но как же выглядела цепочка Великой Чепупахи? Мне кажется, в этом самая суть коана.

\emph{Ахилл} : Сомневаюсь. Мы не должны «привязываться» к таким мелочам. Главное не детали, а дух коана как целого. А знаете, что мне только что пришло в голову? Я думаю, что вы, как это ни удивительно, только что наткнулись на давно утерянный коан, описывающий происхождение Искусства Дзен-цепочек!

\emph{Черепаха} : О, это было бы слишком хорошо для того, чтобы иметь буддистскую природу!

\emph{Ахилл} : Но это бы значило, что великий Мастер, единственный, кто достиг мистического состояния Архи-просветления, звался не Ментором, а Чепупахой. Вот уж поистине странное имя!

\emph{Черепаха} : Я не согласна --- по-моему, это очень красивое имя. Но я все же хочу знать, как выглядела эта Чепупашья цепочка. Можете ли вы воссоздать ее по описанию, данному в коане?

\emph{Ахилл} : Я могу попытаться, хотя мне это будет очень трудно --- ведь у меня нет лап, а в коане все описывается с точки зрения движения именно лап. Это очень необычно, но я постараюсь. Попытка --- не пытка\ldots{}

\emph{(Он берет коан и кусочек цепочки и в течение нескольких минут, пыхтя от усердия, сгибает и складывает его самым невероятным образом, пока в его руках не оказывается готовый продукт.)}

Вот, пожалуйста. Странно, но это выглядит очень знакомо.

\emph{Черепаха} : И правда! Интересно, где я это видела?

\emph{Ахилл} : Я знаю! Это же ВАША цепочка, разве не так?

\emph{Черепаха} : Наверняка нет.

\emph{Ахилл} : Ну конечно: это ваша первая цепочка, которую вы мне дали до того, как завязали вторую неточку.

\emph{Черепаха} : Действительно, она самая. Надо же\ldots{} Интересно, что из этого следует?

\emph{Ахилл} : Все это очень странно, чтобы не сказать большего.

\emph{Черепаха} : Вы думаете, мой коан --- подлинный?

\emph{Ахилл} : Подождите-ка минутку\ldots{}

\emph{Черепаха} : А эта цепочка --- есть ли в ней природа Будды?

\emph{Ахилл} : Ваша цепочка кажется мне подозрительной\ldots{}

\emph{Черепаха (с предовольным видом, не обращая на Ахилла никакого внимания)} : А как насчет Чепупашьей цепочки? Есть ли в ней природа Будды? У меня столько вопросов!

\emph{Ахилл} : Я бы поостерегся задавать столько вопросов, г-жа Ч. Что-то здесь творится, и я совсем не уверен, что это мне нравится.

\emph{Черепаха} : Грустно слышать; но я не понимаю, что вас тревожит?

\emph{Ахилл} : Лучше всего это объясняет цитата из другого древнего Мастера дзен-буддизма по имени Киоген. Киоген сказал:

\emph{Дзен подобен человеку, удерживающемуся зубами за ветку растущего над пропастью дерева. Руки и ноги его, не имея опоры, болтаются в воздухе. Под деревом стоит другой человек и спрашивает его. «Почему Бодхидхарма пришел из Индии в Китай?». Если человек на дереве не ответит, он изменит дзену, а если он ответит, то упадет и погибнет. Что ему делать?}

\emph{Черепаха} : Ясно как день: ему надо оставить дзен и заняться молекулярной биологией.


% \subsubsection{ГЛАВА IX: Мумон и Гёдель}
% \include{fragments/xx26}

\part{Часть~II}

\emph{Триплеты «GEB» и «EGB»}

% \subsubsection{Прелюдия и\ldots{}}
% \subsubsection{Прелюдия и\ldots{}}

\emph{Рис. 54. М. К. Эшер. «Лист Мёбиуса II» (гравюра на дереве, 1963).}

\emph{Ахилл и Черепаха пришли в гости к Крабу, чтобы познакомиться с его другом Муравьедом. После того, как новые знакомые представлены друг другу, вся четверка садится за чай.}

\emph{Черепаха} : Мы вам кое-что принесли, мистер Краб.

\emph{Краб} : Очень любезно с вашей стороны, но зачем же было утруждаться?

\emph{Черепаха} : О это так, мелочь --- в знак нашего уважения. Ахилл, отдайте, пожалуйста, подарок м-ру К.

\emph{Ахилл} : С удовольствием. С наилучшими пожеланиями, м-р К. Надеюсь, что вам понравится.

\emph{(Ахилл протягивает Крабу элегантно завернутый пакет, квадратный и плоский Краб начинает его разворачивать).}

\emph{Муравьед} : Интересно, что это такое?

\emph{Краб} : Сейчас узнаем (Кончает разворачивать и вытаскивает подарок). Две пластинки! Прекрасно! Но погодите-ка здесь нет этикетки. Неужели это снова ваши «особые» записи, г-жа Ч?

\emph{Черепаха} : Если вы имеете в виду разбивальную музыку, на этот раз нет. Но эти записи действительно уникальны, так как они сделаны по персональному заказу. На самом деле, их еще никто никогда не слышал --- кроме, конечно Баха, когда тот их играл.

\emph{Краб} : Когда Бах их играл? Что вы имеете в виду?

\emph{Ахилл} : Вы будете вне себя от счастья, м-р Краб, когда г-жа Ч объяснит вам, что это за пластинки.

\emph{Черепаха} : Почему бы вам самому этого не рассказать, Ахилл? Не стеснятесь, говорите!

\emph{Ахилл} : Можно? Вот здорово! Но я лучше загляну сначала в свои записи (Вытаскивает бумажку и откашливается ) Кхе-кхе. Желаете послушать рассказ о замечательных новых результатах в математике --- результатах, которым ваши пластинки обязаны своим существованием?

\emph{Краб} : Мои пластинки восходят к каким-то математическим выкладкам? Как интересно! Что ж, теперь, когда вы задели мое любопытство, я просто обязан об этом узнать.

\emph{Ахилл} : Отлично! (Делает паузу, чтобы отхлебнуть чай, затем продолжает) Кто-нибудь из вас слышал о печально известной «Последней Теореме» Ферма?

\emph{Муравьед} : Не уверен. Звучит знакомо, но не могу припомнить.

\emph{Рис. 55. Пьер Де Ферма}

\emph{Ахилл} : Идея очень проста. Пьер де Ферма, адвокат по профессии и математик по призванию, однажды, читая классический текст Диофанта «Арифметика», наткнулся на следующее уравнение:

a\&\#178; + b\&\#178; = c\&\#178;

Он тут же понял, что это уравнение имеет бесконечно много решений для \emph{а} , \emph{b} , и \emph{с,} и написал на полях свою знаменитую поправку:

Уравнение:

а \textsuperscript{\emph{n}} + b \textsuperscript{\emph{n}} = с \textsuperscript{\emph{n}}

имеет решение в положительных целых числах \emph{а} , \emph{b} , \emph{с} , и~\emph{n} только при~\emph{n} = 2 (и в таком случае имеется бесконечное множество \emph{a} , \emph{b} , и \emph{c} , удовлетворяющих этому уравнению), но для \emph{n} \textgreater2 решений не существует. Я нашел замечательное доказательство этого, которое, к несчастью, не помещается на полях.

С того дня и в течение почти трехсот лет математики безуспешно пытаются сделать одно из двух: либо доказать утверждение Ферма и таким образом очистить его репутацию, в последнее время слегка подпорченную скептиками, не верящими, что он действительно нашел доказательство --- либо опровергнуть его утверждение, найдя контрпример: множество четырех целых чисел \emph{а} , \emph{b} , \emph{с} , и \emph{n} , где~\emph{n} \textgreater{} 2, которое удовлетворяло бы этому уравнению. До недавнего времени все попытки в любом из этих двух направлений проваливались. Точнее, теорема доказана лишь для определенных значений~\emph{n} --- в частности, для всех~\emph{n} до 125 000.

\emph{Ахилл} : Не лучше ли тогда называть это Гипотезой вместо Теоремы, поскольку настоящее доказательство еще не найдено?

\emph{Ахилл} : Строго говоря, вы правы, но по традиции это зовется именно так.

\emph{Краб} : Удалось ли кому-нибудь в конце концов разрешить этот знаменитый вопрос?

\emph{Ахилл} : Представьте себе, да: это сделала г-жа Черепаха, как всегда, в момент гениального озарения. Она не только нашла ДОКАЗАТЕЛЬСТВО Последней Теоремы Ферма (оправдав, таким образом, ее название и очистив репутацию Ферма), но и КОНТРПРИМЕР, показав, что интуиция скептиков их не подвела!

\emph{Краб} : Вот это да! Поистине революционное открытие.

\emph{Муравьед} : Прошу вас, не тяните: что это за магические числа, удовлетворяющие уравнению Ферма? Мне особенно любопытно узнать значение \emph{n} .

\emph{Ахилл} : Ах, какой ужас! Какой стыд! Верите ли, я оставил все выкладки дома на громаднейшем листе бумаги. К несчастью, он был слишком велик, чтобы принести его с собой. Хотел бы я, чтобы он был сейчас здесь и чтобы можно было вам все показать. Но кое-что я все же помню: величина~\emph{n} --- единственное положительное число, которое нигде не встречается в непрерывной дроби числа \&\#960;.

\emph{Краб} : Какая жалость, что у вас нет с собой ваших записей. Так или иначе, у нас нет оснований сомневаться, что все, что вы нам сказали --- чистая правда.

\emph{Муравьед} : Да и кому, в конце концов, нужно видеть~\emph{n} в десятичной записи? Ахилл же объяснил нам, как найти это число. Что ж, г-жа Черепаха, примите мои сердечные поздравления по поводу вашего эпохального открытия!

\emph{Черепаха} : Благодарю вас. Однако практическая польза, которую немедленно принес мой результат, кажется мне еще важнее теоретического открытия.

\emph{Краб} : Смерть как хочется услышать об этом --- ведь я всегда считал, что теория чисел --- Царица Чистой Математики, единственная ветвь математики, не имеющая НИКАКОГО практического приложения.

\emph{Черепаха} : Вы не единственный, кто так думает; однако на деле почти невозможно предсказать, когда и каким образом какая-либо ветвь чистой математики --- или даже какая-либо индивидуальная Теорема --- повлияет на другие науки. Это происходит совершенно неожиданно, и данный случай --- хороший тому пример.

\emph{Ахилл} : Обоюдоострый результат г-жи Черепахи прорубил дверь в область акусто-поиска.

\emph{Муравьед} : Что такое акусто-поиск?

\emph{Ахилл} : Название говорит само за себя: это поиск и извлечение акустической информации из сложных источников. Например, типичная задача акусто-поиска~--- восстановить звук, произведенный упавшим в воду камнем, по форме расходящихся по воде кругов.

\emph{Краб} : Но это невозможно!

\emph{Ахилл} : Почему же? Это весьма похоже на то, что делает наш мозг, когда он восстанавливает звук, произведенный голосовыми связками другого человека, по колебаниям, переданным барабанной перепонкой далее по лабиринту ушной раковины.

\emph{Краб} : Ясно. Но я все еще не вижу связи этого ни с теорией чисел, ни с моими новыми пластинками.

\emph{Ахилл} : Видите ли, в математике акусто-поиска часто возникают вопросы, связанные с числом решений неких Диофантовых уравнений. А г-жа Ч годами занималась тем, что пыталась восстановить звуки игры Баха на клавесине (что происходило более двухсот лет тому назад), основываясь на расчетах движения всех молекул в атмосфере в настоящее время.

\emph{Муравьед} : Но это же совершенно невозможно! Эти звуки утрачены навсегда, утеряны невозратимо!

\emph{Ахилл} : Так думают непосвященные --- но г-жа Ч посвятила много лет этой проблеме и пришла к выводу, что все зависит от количества решений уравнения:

а \textsuperscript{\emph{n}} + b \textsuperscript{\emph{n}} = с \textsuperscript{\emph{n}}

в положительных числах, при~\emph{n} \textgreater{} 2.

\emph{Черепаха} : Я могла бы объяснить, при чем здесь это уравнение, но не хочу наскучить присутствующим.

\emph{Ахилл} : Оказалось, что теория акусто-поиска предсказывает, что звуки Баховского клавесина могут быть восстановлены по движению всех молекул атмосферы при одном из двух условий ЛИБО у этого уравнения есть хотя бы одно решение.

\emph{Краб} : Удивительно!

\emph{Муравьед} : Фантастика да и только!

\emph{Черепаха} : Кто бы мог подумать!

\emph{Ахилл} : Я хотел сказать, «ЛИБО такое решение существует, ЛИБО существует доказательство, что уравнение НЕ имеет решений!» Итак, г-жа Ч начала кропотливую работу с обоих концов проблемы одновременно Оказалось, что нахождение контрпримера было ключом к нахождению доказательства, так что одно прямо вело к другому.

\emph{Краб} : Как же это возможно?

\emph{Черепаха} : Видите ли, мне удалось показать, что структуру любого доказательства Последней Теоремы Ферма --- если таковое существует --- возможно описать с помощью элегантной формулы, которая зависела бы от величин решения некоего уравнения. Когда я нашла это второе уравнение, к моему удивлению оно оказалось не чем иным как уравнением Ферма. Забавное случайное соотношение между формой и содержанием. Так что, когда я нашла контрпример, мне осталось только использовать эти числа как план для построения доказательства того, что это уравнение не имеет решения. Замечательно просто, если подумать. Не знаю, почему никто не нашел этого результата раньше.

\emph{Ахилл} : В результате этого неожиданного блестящего математического успеха, г-же Ч удалось провести акусто-поиск о котором она столько лет мечтала. Подарок полученный м-ром Крабом представляет собой осязаемую реализацию этой абстрактной работы.

\emph{Краб} : Не говорите мне пожалуйста что это запись Баха, играющего на клавесине собственные сочинения!

\emph{Ахилл} : Сожалею, но приходится поскольку это именно она и есть! Это набор из двух записей Себастиана Баха исполняющего весь Хорошо Темперированный Клавир. На каждой пластинке записана одна из двух его частей, это значит что каждая запись состоит из 24 прелюдий и фуг по одной в каждом мажорном и минорном ключе.

\emph{Краб} : В таком случае мы должны немедленно прослушать эти бесценные пластинки! Как я смогу вас отблагодарить?

\emph{Черепаха} : Вы уже нас отблагодарили сполна этим превосходным чаем, который вы для нас приготовили.

(Краб вынимает одну из пластинок из конверта и ставит ее на свой патефон. Комната наполняется звуками потрясающей, мастерской игры на клавесине, при этом качество записи самое высокое, какое можно вообразить. Можно даже разобрать --- или это только воображение слушателя? ---~тихий голос Баха, подпевающего собственной игре)

\emph{Краб} : Хотите следить по нотам? У меня есть уникальное издание Хорошо Темперированною Клавира, проиллюстрированное одним из моим учителей, который также был необыкновенным каллиграфом.

\emph{Черепаха} : Это было бы чудесно.

\emph{(Краб подходит к элегантному книжному шкафу с застекленными дверцами, открывает его и достает два больших тома.)}

\emph{Краб} : Вот, пожалуйста, г-жа Черепаха. Я сам еще не видел всех прекрасных иллюстраций в этом издании, все никак клешни не доходят. Может быть, ваш подарок меня наконец на это подвигнет.

\emph{Черепаха} : Надеюсь.

\emph{Муравьед} : Вы заметили, что во всех этих произведениях прелюдия точно определяет настроение следующей фуги?

\emph{Краб} : О, да. Хотя это трудно объяснить, но между ними всегда есть некая таинственная связь. Даже если у прелюдии и фуги нет общей музыкальной темы, в них всегда присутствует неуловимое абстрактное нечто, которое их прочно связывает.

\emph{Черепаха} : И в кратких моментах тишины, которые отделяют прелюдию от фуги, есть что-то необыкновенно драматическое. Это тот момент, когда тема фуги готова вступить в свои права, сначала в отдельных голосах, которые потом сплетаются, создавая все более сложные уровни странной, изысканной гармонии.

\emph{Ахилл} : Я знаю, что вы имеете в виду. Я слышал еще далеко не все прелюдии и фуги, но меня очень волнует этот момент тишины; в это время я всегда пытаюсь угадать, что старик Бах задумал на этот раз. Например, я всегда спрашиваю себя, в каком темпе будет следующая фуга? Будет ли это аллегро или адажио? Будет ли она на 6/8 или на 4/4? Будет ли в ней три голоса, или пять --- или четыре? И вот звучит первый голос\ldots{} Потрясающий момент!

\emph{Краб} : Да, я помню давно ушедшие дни моей юности, дни, когда я трепетал от счастья, слушая эти прелюдии и фуги, возбужденный их новизной и красотой, и теми сюрпризами, которые они скрывают.

\emph{Ахилл} : А теперь? Неужели это счастливый трепет прошел?

\emph{Краб} : Он перешел в привычку, как всегда и бывает с подобными чувствами. Но в привычке также есть своя глубина, и это приносит определенное удовлетворение. Кроме того, я всегда обнаруживаю какие-нибудь новые сюрпризы, которых раньше не замечал.

\emph{Ахилл} : Повторения темы, которых вы раньше не слышали?

\emph{Краб} : Может быть; в особенности, когда эта тема проходит в обращении, спрятанная среди нескольких других голосов, или когда она поднимается на поверхность, словно возникая из ничего. Есть там также и удивительные модуляции, которые приятно слушать снова и снова, спрашивая себя, как это старик Бах смог создать подобное.

\emph{Ахилл} : Приятно слышать, что все эти радости останутся у меня после того, как пройдет моя первая влюбленность в Хорошо Темперированный Клавир, жаль, однако, что это блаженное состояние не может длиться вечно.

\emph{Краб} : Не бойтесь, влюбленность не пройдет бесследно. Эта юношеская влюбленность хороша тем, что ее всегда можно оживить именно тогда, когда вы считаете, что она уже умерла. Для этого необходим лишь толчок извне в нужном направлении.

\emph{Ахилл} : Правда? Что же это за толчок?

\emph{Краб} : Например, прослушивание этой музыки ушами того, кто слушает ее в первый раз; такой человек здесь вы, Ахилл. Каким-то образом, ваш трепет передается мне, и я снова полон блаженного восторга!

\emph{Ахилл} : Звучит интригующе. Восторг спит где-то внутри вас, но сами вы не в состоянии вытащить его из глубин подсознания.

\emph{Краб} : Именно так. Возможность оживить это чувство «закодирована» каким-то образом в структуре моего мозга, но я не могу осуществить это по желанию; я должен ждать счастливого случая, который запустит этот механизм.

\emph{Ахилл} : У меня вопрос насчет фуг; мне стыдно об этом спрашивать, но, поскольку я новичок в искусстве слушания фуг, не может ли кто-нибудь из вас, матерых слушателей, научить меня кое-чему?

\emph{Черепаха} : Я с удовольствием поделюсь с вами своими скудными познаниями, если это может вам чем-то помочь.

\emph{Ахилл} : О, благодарю вас. Позвольте мне начать издалека. Знакомы ли вы с гравюрой М. К. Эшера под названием «Куб с магическими лентами»?

\emph{Рис. 56. М. К. Эшер «Куб с магическим лентами» (литография, 1957)}

\emph{Черепаха} : На которой изображены изогнутые ленты с искривлениями в виде пузырей, которые кажутся попеременно то выпуклыми, то вогнутыми?

\emph{Ахилл} : Она самая.

\emph{Краб} : Я помню эту картину. Кажется, что пузыри на ней все время перескакивают из одного состояния в другое: они то выпуклые, то вогнутые, в зависимости от того, с какого угла на них посмотреть. Невозможно одновременно увидеть их и выпуклыми, и вогнутыми --- почему-то мозг этого просто не позволяет. У нас просто есть два разных способа воспринять эти пузыри.

\emph{Ахилл} : Вы совершенно правы Знаете, мне кажется, что я открыл два способа слушать фугу, в чем-то аналогичных этому Вот они: либо следить лишь за одним отдельным голосом в каждый момент, либо слушать общее звучание, не пытаясь распутать голоса. Я пробовал оба эти способа и, к моему разочарованию, оказалось, что каждый из них исключает другой. Это просто не в моей власти слушать каждый индивидуальный голос и в то же время слышать общий эффект. Я все время перескакиваю с одного способа на другой, более или менее спонтанно и непроизвольно.

\emph{Муравьед} : Так же, как когда вы смотрите на магические ленты?

\emph{Ахилл} : Да. Но скажите\ldots{} мое описание двух способов слушания фуги безошибочно указывает на меня, как на наивного, неопытного слушателя, не способного уловить более глубокие уровни восприятия?

\emph{Черепаха} : Вовсе нет, Ахилл Я могу говорить только за себя, но я тоже постоянно перепрыгиваю с одного способа на другой, не контролируя этот процесс и не пытаясь сознательно решить, какой из двух способов должен господствовать. Не знаю, испытывали ли остальные наши друзья что-нибудь подобное.

\emph{Краб} : Безусловно. Это весьма мучительное состояние, поскольку вы чувствуете, что дух фуги витает где-то близко --- но вы не можете охватить его полностью, так как не в состоянии слушать сразу двумя способами.

\emph{Муравьед} : У фуг есть интересная особенность: каждый из голосов является музыкальной пьесой сам по себе, так что фугу можно рассматривать как набор нескольких различных музыкальных произведений, основанных на одной и той же теме и исполняемых одновременно. И слушатель (или его подсознание) должен сам решать, воспринимать ли фугу как целое или как набор отдельных частей, гармонирующих друг с другом.

\emph{Ахилл} : Вы говорите, что эти части «независимы», однако это не может быть совершенно верным. Между ними должна существовать какая-то координация, иначе, когда они исполняются вместе, мы слышали бы беспорядочное столкновение звуков --- а это далеко не так!

\emph{Муравьед} :~Наверное, лучше сказать так: если бы вы слушали каждый голос в отдельности, вы обнаружили бы, что он имеет смысл сам по себе. Он может быть исполнен в одиночку, и именно это я имел в виду, говоря, что голоса независимы. Но вы совершенно правы, указывая, что каждая из этих индивидуальных мелодий соединяется с остальными совсем не случайным образом, сливаясь в изящное целое. Искусство создания прекрасных фуг заключается именно в умении соединять несколько линий, каждая из которых кажется написанной ради своей собственной красоты --- но когда они взяты все~вместе, целое звучит вполне естественно. Между прочим, двойственность между слушанием фуги как целого и слушанием составляющих её голосов --- это частный пример более общей двойственности, приложимой к разным структурам, построенным, начиная с нижних уровней.

\emph{Ахилл} : Правда?~Вы хотите сказать, что мои два «способа»~приложимы не только к ситуации со слушанием фуг?

\emph{Муравьед} : Совершенно верно.

\emph{Ахилл} : Интересно, как это может быть? Наверное, это связано с попеременным восприятием чего-либо как целого, или как собрания его частей. Но я сталкивался с этой дихотомией только слушая фуги

\emph{Черепаха} : Вот это да! Посмотрите-ка! Я только что перевернула страницу следя за музыкой, и нашла великолепную иллюстрацию на странице перед титульным листом.

\emph{Краб} : Я раньше никогда не видел этой иллюстрации. Будьте добры, передайте книгу по кругу.

\emph{(Черепаха передает книгу. Каждый из четырех приятелей рассматривает книгу по-своему --- кто издалека, кто поднося прямо к глазам, при этом каждый из них качает головой в удивлении. Наконец, книга обходит всех и возвращается к Черепахе, которая смотрит в нее очень внимательно)}

\emph{Ахилл} : Мне кажется, прелюдия почти кончилась. Хотелось бы знать, удастся ли мне, слушая фугу, найти ответ на этот опрос «как нужно слушать фугу --- как целое или как сумму частей?»

\emph{Черепаха} : Слушайте внимательно и вы поймете!

\emph{(Прелюдия заканчивается. Следует пауза, и затем\ldots{} )}


% \subsubsection{ГЛАВА X: Уровни описания и компьютерны системы}
% \include{fragments/xx29}



\end{document}
