\documentclass[a4paper,12pt,oneside,openany]{memoir}
\usepackage{sty/preamble}

%% Frame around the text area
% \usepackage{showframe}
% \renewcommand\ShowFrameLinethickness{0.1pt}
% \renewcommand*\ShowFrameColor{\color{lightgray}}

\setcounter{tocdepth}{1}
\setcounter{secnumdepth}{1}


\title{ГЁДЕЛЬ, ЭШЕР, БАХ: \\ эта бесконечная гирлянда}
\author{Дуглас Ричард Хофштадтер}


\begin{document}

\frontmatter

\maketitle

\mainmatter

% Праздничное предисловие автора к русскому изданию книги «Гёдель, Эшер Бах»
\subfile{parts/preface}

\begingroup
\hypersetup{hidelinks}
\clearpage
% \begin{KeepFromToc}
\tableofcontents
% \end{KeepFromToc}
\clearpage
\endgroup

% Обзор
\subfile{parts/overview}
% Список иллюстраций
\subfile{parts/illustrations}
% Благодарность
\subfile{parts/acknowledgment}

\part{ЧАСТЬ~I}

% TODO: illustration 1
% \emph{Рис. I. Иоганн Себастиан Бах в 1748. С портрета кисти Элиаса Готтлиба Хауссманна.}

% Интродукция: музыко-логическое приношение
\subfile{parts/introduction}
% Трехголосная инвенция
\clearpage\subfile{parts/dial01}
% Глава 1. Головоломка MU
\subfile{parts/ch01}
% Двухголосная инвенция
\clearpage\subfile{parts/dial02}
% Глава 2. Содержание и форма в математике
\subfile{parts/ch02}
% Соната для Ахилла соло
\clearpage\subfile{parts/dial03}
% Глава 3. Рисунок и фон
\subfile{parts/ch03}
% Акростиконтрапунктус
\clearpage\subfile{parts/dial04}
% Глава 4. Непротиворечивость, полнота и геометрия
\subfile{parts/ch04}
% Маленький гармонический лабиринт
\clearpage\subfile{parts/dial05}
% Глава 5. Рекурсивные структуры и процессы
\subfile{parts/ch05}
% Канон с интервальным увеличением
\clearpage\subfile{parts/dial06}
% Глава 6. Местонахождение значения
\subfile{parts/ch06}
% Хроматическая фантазия и фига
\clearpage\subfile{parts/dial07}
% Глава 7. Исчисление Высказываний
\subfile{parts/ch07}
% Крабий канон
\clearpage\subfile{parts/dial08}
% Глава 8. Типографская теория чисел
\subfile{parts/ch08}
% Приношение «МУ»
\clearpage\subfile{parts/dial09}
% Глава 9. Мумон и Гёдель
\subfile{parts/ch09}

\part{Часть~II}

% TODO: illustration GEB/EGB
% \emph{Триплеты «GEB» и «EGB»}

% Прелюдия и...
\clearpage\subfile{parts/dial10}
% Глава 10. Уровни описания и компьютерные системы
\subfile{parts/ch10}
% ...и Муравьиная фуга
% \clearpage\subfile{parts/dial11}

\end{document}
