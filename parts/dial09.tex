\documentclass[../main.tex]{subfiles}
\begin{document}
\begingroup

\newtcolorbox{koan}[1][]{% [<style>]
    boxrule=0.3mm,
    fontupper=\small,
    right skip=2pc,
    enlarge top initially by=4pt,
    enlarge bottom finally by=4pt,
    colback=black!3,
    #1
}

\DialogueChapterOpt[Приношение «МУ»]{Приношение «МУ»\protect\footnotemark}
\footnotetext{Все коаны в этом Диалоге подлинны, они взяты из следующих двух книг: Paul Reps «Zen Flesh Zen Bones» и Gyomay M. Kubose «Zen Koans»}

\centerblock{
    \emph{Черепаха и Ахилл только что вернулись с лекции о происхождении Генетического Кода; они сидят у Ахилла и пьют чай.}
}

\begin{dialogue}

\speak{Ахилл} Я должен кое в чем признаться, г-жа Ч\@.

\speak{Черепаха} Что такое?

\speak{Ахилл} Несмотря на интереснейшую тему, я пару раз задремал\ldots{} Но даже во сне я кое-что слышал. Вот какая странная мысль всплыла из глубины моего сознания: «А» и «Т» могут обозначать не «аденин» и «тимин», а мое и ваше имена! Ведь вас зовут Тортилла! Кроме того, в моем полусне вдоль остова двойной спирали ДНК были подвешены крохотные Ахиллы и Тортиллы, всегда в парах, как аденин и тимин. Правда, странный образ?

\speak{Черепаха} Фу! Кто верит в подобные глупости? К тому же, что вы скажете о «С» и~«G»?

\speak{Ахилл} Что ж, цитозин мог бы обозначать г-на Краба \--- ведь его имя пишется «Crab». Насчёт~«G» я не знаю, но уверен, что можно было бы что-нибудь придумать. Так или иначе, было забавно вообразить мою ДНК, полную ваших малюсеньких копий \--- и моих, конечно. Только подумайте, к какой бесконечной регрессии это бы привело!

\speak{Черепаха} Вижу, что вы не очень-то внимательно слушали лекцию.

\speak{Ахилл} Неправда \--- я старался изо всех сил. Просто было очень трудно отделить мои фантазии от фактов. В конце концов, молекулярные биологи изучают такой необыкновенный нижний мир\ldots{}

\speak{Черепаха} Что вы имеете в виду?

\speak{Ахилл} Молекулярная биология полна странных спиральных петель, которые я как следует не понимаю. Например, белки, закодированные в ДНК, могут «провернуться назад» и повлиять на саму ДНК \--- даже разрушить её. Подобные странные петли меня всегда запутывают. В них есть что-то пугающее.

\speak{Черепаха} Я нахожу их весьма привлекательными.

\speak{Ахилл} Разумеется \--- они вполне в вашем вкусе. Но мне иногда хочется прекратить весь этот анализ и просто помедитировать немного, в качестве противоядия. Это очищает голову от путаницы странных петель и всех этих невероятных сложностей, о которых мы сегодня услышали.

\speak{Черепаха} Удивительно! Никогда бы не подумала, что вы медитируете.

\speak{Ахилл} Разве я никогда не говорил вам, что изучаю дзен-буддизм?

\speak{Черепаха} Боже мой, как вы до этого додумались?

\speak{Ахилл} Мне всегда казалось, что без инь и янь мое дело \--- дрянь; знаете, все эти путешествия в восточный мистицизм, И-Чинг, гуру, и тому подобное. В одни прекрасный день я подумал: «Почему бы мне не заняться и дзеном?» Так это всё и началось.

\speak{Черепаха} Превосходно! Может быть и я, наконец, сподоблюсь просветиться.

\speak{Ахилл} Ну-ну, не так быстро. Просветление \--- совсем не первый шаг на пути к буддизму; скорее, это последний шаг. Просветление не для таких новичков, как вы, г-жа Ч!

\speak{Черепаха} Вы меня не поняли. Я не имела в виду буддистское просветление \--- мне просто хотелось узнать, что такое дзен-буддизм.

\speak{Ахилл} Бог ты мой, что же вы сразу не сказали? Я буду очень рад рассказать вам все, что знаю о дзене. Может быть, вам даже захочется стать учеником буддизма, таким же, как и я.

\speak{Черепаха} Что ж, нет ничего невозможного.

\speak{Ахилл} Вы можете изучать буддизм вместе со мной у моего Мастера Оканисамы \--- седьмого патриарха.

\speak{Черепаха} Черт меня побери, если я что-нибудь понимаю!

\speak{Ахилл} Чтобы это понять, необходимо знать историю дзен-буддизма.

\speak{Черепаха} В таком случае, не расскажете ли вы мне немного об истории дзена?

\speak{Ахилл} Отличная мысль. Дзен \--- это тип буддизма; он был основан монахом по имени Бодхидхарма, который оставил Индию и поселился в Китае. Это было в шестом веке. Бодхидхарма был первым патриархом. Шестым патриархом был\ldots{} э-э-э\ldots{} проклятый склероз\ldots{} Энон! (Наконец-то вспомнил!)

\speak{Черепаха} Неужели Зенон? Как странно, что именно он оказался замешанным в таком деле.

\speak{Ахилл} Осмелюсь заметить, что вы недооцениваете значимость дзена. Послушайте ещё немного и, может быть, вы будете относиться к нему с большим уважением. Так вот, как я говорил, примерно пятьсот лет спустя дзен пришел в Японию, где он прекрасно прижился. С того времени он стал одной из основных религий Японии.

\speak{Черепаха} Кто такой этот Оканисама, «седьмой патриарх»?

\speak{Ахилл} Он мой Мастер, и его учение прямо следует из учения шестого патриарха. Он научил меня тому, что действительность \--- едина и неизменна; вся множественность, изменения и движение \--- не более, чем иллюзии наших чувств.

\speak{Черепаха} Точно \--- это за километр пахнет дзеном. Но как же он впутался в дзен, бедняга?

\speak{Ахилл} Что-о? Если КТО-ТО и запутался, то это\ldots{} Ну ладно, это уже другой разговор. Так или иначе, я не знаю ответа на ваш вопрос. Вместо этого я вам лучше расскажу ещё что-нибудь из поучений моего Мастера. Я узнал, что в дзене человек ищет Просветления, или САТОРИ \--- состояния «He-разума». В этом состоянии человек не думает о мире \--- он просто СУЩЕСТВУЕТ\@. Я также узнал, что изучающий дзен не должен «привязывать» себя ни к какому объекту, или мысли, или человеку \--- то есть, он не должен верить ни в какой абсолют и не должен зависеть от чего-либо, включая и саму эту философию не-привязанности.

\speak{Черепаха} Г-мм\ldots{} Это уже КОЕ-ЧТО; дзен начинает мне нравиться.

\speak{Ахилл} У меня было предчувствие, что вы сразу к нему привяжетесь.

\speak{Черепаха} Но скажите мне: если дзен отрицает интеллектуальную деятельность вообще, то какой смысл размышлять о нем и усердно его изучать?

\speak{Ахилл} Мне тоже не давала покоя эта мысль. Но думаю, что я, наконец, нашел ответ: к дзену можно подходить по любой дороге, даже если эта дорога кажется ведущей совершенно в другую сторону. По мере того, как вы к нему приближаетесь, вы учитесь отходить от дороги в сторону; и чем больше вы отходите в сторону, тем ближе вы подходите к дзену.

\speak{Черепаха} Теперь все кажется совсем простым.

\speak{Ахилл} Моя любимая дорога к дзену проходит через его короткие, интересные и странные притчи, под названием «коаны».

\speak{Черепаха} Что это такое \--- коан?

\speak{Ахилл} Коан \--- это история о Мастерах дзена и их учениках. Иногда он в форме загадки, иногда \--- басни, а иногда коан совершенно не похож ни на что, слышанное вами раньше.

\speak{Черепаха} Звучит интригующе. Вы думаете, что читать коаны и наслаждаться ими значит заниматься дзен-буддизмом?

\speak{Ахилл} Сомневаюсь. Однако мне кажется, что получать удовольствие от коанов в миллион раз ближе к настоящему дзену, чем читать об этой религии том за томом, написанные на тяжелом философском жаргоне.

\speak{Черепаха} Хотелось бы услышать какой-нибудь коан.

\speak{Ахилл} С удовольствием расскажу вам парочку. Я должен, пожалуй, начать с самого знаменитого. Итак, много столетий тому назад жил Мастер дзен-буддизма по имени Джошу, который дожил до 119 лет.

\speak{Черепаха} Просто юнец!

\speak{Ахилл} С вашей точки зрения, конечно. Так вот, однажды, когда Джошу и другой монах стояли вместе в монастыре, мимо пробежала собака. Монах спросил Джошу: «У этого дога \--- природа Будды?»

\speak{Черепаха} Непонятно. Так что же ответил монах?

\speak{Ахилл} МУ.

\speak{Черепаха} МУ? Что это за «МУ» такое? А как же насчёт собаки? И природы Будды? Как же ответ?

\speak{Ахилл} Но ведь «МУ» и есть ответ Джошу! Говоря «МУ», Джошу дал понять другому монаху, что только воздерживаясь от подобных вопросов, можно получить на них ответ.

\speak{Черепаха} Джошу «развопросил» этот вопрос.

\speak{Ахилл} Именно!

\speak{Черепаха} Это «МУ» \--- весьма полезная штучка. Иногда мне тоже хочется развопросить кое-какие вопросы. Кажется, я начинаю ухватывать суть дзена\ldots{} Вы знаете ещё какие-нибудь коаны, Ахилл? Мне хотелось бы услышать ещё несколько.

\speak{Ахилл} Охотно. Я знаю парочку коанов, которые всегда рассказываются вместе. Только\ldots{}

\speak{Черепаха} Что такое?

\speak{Ахилл} Дело в том, что мой Мастер предупреждал меня, что только один из них настоящий. Хуже того, он не знает, какой из них подлинный, а какой \--- фальшивка.

\speak{Черепаха} С ума сойти! Расскажите-ка их мне, чтобы мы могли наугадываться всласть!

\speak{Ахилл} Хорошо. Один из коанов таков:

% TODO
\stage{\small\centering
    Один монах спросил Басо: «Что такое Будда?» \\
    Басо ответил: «Этот разум \--- Будда.»
}

\speak{Черепаха} Гмм\ldots{} «Этот разум \--- Будда»? Иногда мне трудно понять, что хотят сказать эти дзен-буддисты.

\speak{Ахилл} Тогда второй коан может понравиться вам больше.

\speak{Черепаха} Что это за коан?

\speak{Ахилл} Вот он:

% TODO
\stage{\small\centering
    Один монах спросил Басо: «Что такое Будда?» \\
    Басо ответил: «Этот разум \--- не Будда.»
}

\speak{Черепаха} Ну и ну! Как если бы мой панцирь был зеленый и не зеленый! Это мне нравится!

\speak{Ахилл} Однако, г-жа Т, коаны совсем не предназначены для того, чтобы просто «нравиться».

% TODO: illustration 45
\emph{Рис. 45. М.К.~Эшер «Мечеть» (черные и белые мелки, 1936)}

\speak{Черепаха} Отлично, в таком случае это мне не нравится.

\speak{Ахилл} Так-то лучше. Так вот, как я говорил, мой мастер считает, что только один из них \--- настоящий.

\speak{Черепаха} Не могу себе представить, что заставило его так решить. Все равно этот вопрос чисто академический, поскольку невозможно узнать, какой из двух коанов \--- оригинал, а какой \--- подделка.

\speak{Ахилл} Вы ошибаетесь: мой Мастер научил нас, как это сделать.

\speak{Черепаха} Неужели? Разрешающий алгоритм для установления подлинности коанов? Хотелось бы мне услышать об ЭТОМ.

\speak{Ахилл} Это довольно сложный ритуал: в нем два этапа. На первом этапе вы должны ТРАНСЛИРОВАТЬ данный коан в цепочку, уложенную спиралью в трех измерениях.

\speak{Черепаха} Забавная штучка. А как насчёт второго этапа?

\speak{Ахилл} Ну, это совсем просто: надо всего-навсего определить, имеет цепочка природу Будды или нет! Если у нее \--- природа Будды, то коан \--- подлинный, а если нет, то он \--- фальшивка.

\speak{Черепаха} Гмм\ldots{} Это звучит так, словно вы только перенесли нужду в разрешающей процедуре в другую область. ТЕПЕРЬ вам нужна разрешающая процедура для определения природы Будды. Что же дальше? В конце концов, если вы не можете сказать даже того, буддистская ли природа у СОБАКИ,~~как же вы собираетесь определить это для любого кусочка цепочки трехмерной укладки?

\speak{Ахилл} Мой мастер объяснил мне, что переход из одной области в другую может помочь. Это похоже на перемену точки зрения. Некоторые вещи выглядят сложными под одним углом, но простыми под другим. Он привел в пример сад: глядя на него с одной стороны, вы не видите никакого порядка, только под некоторыми углами перед вами возникает прекрасная упорядоченность. Вы организовали информацию иначе, взглянув на вещи с иной точки зрения.

\speak{Черепаха} Понятно. В таком случае, может оказаться, что подлинность коана спрятана в нем где-то глубоко, но когда вам удается перевести его в цепочку, она каким-то образом всплывает на поверхность?

\speak{Ахилл} Именно это и открыл мой Мастер.

\speak{Черепаха} В таком случае, мне бы хотелось узнать об этой технике побольше. Но сперва скажите мне, как вы можете превратить коан (последовательность слов) в уложенную в пространстве цепочку (трехмерный объект)? Ведь это довольно разные классы предметов.

\speak{Ахилл} Это как раз одна из наиболее таинственных вещей, которые я узнал, изучая дзен. Есть два шага: «транскрипция» и «трансляция». Сделать транскрипцию коана \--- значит записать его фонетическим алфавитом, который содержит только четыре геометрических символа. Эта фонетическая транскрипция коана называется ПОСРЕДНИКОМ.

\speak{Черепаха} Как выглядят эти геометрические символы?

\speak{Ахилл} Они состоят из гексагонов и пентагонов; вот так (берет лежащую рядом салфетку и набрасывает следующие четыре фигуры):

\speak{Черепаха} Выглядит загадочно.

\speak{Ахилл} Только для непосвященных. Теперь, когда посредник готов, вы натирайте руки рибосом, и\ldots{}

\speak{Черепаха} Рибосом? Это что, ритуальная мазь?

\speak{Ахилл} Не совсем. Это специальный клейкий состав, который помогает цепочке сохранять форму, когда она уложена.

\speak{Черепаха} Из чего он сделан?

\speak{Ахилл} Точно не знаю, но он клейкий на ощупь и прекрасно работает. Так или иначе, когда вы натерли руки рибосом, вы можете транслировать последовательность символов в посреднике в некий тип укладки цепочки. Как видите, все очень просто.

\speak{Черепаха} Подождите! Не так быстро! Как вы это делаете?

\speak{Ахилл} Вы берете прямую цепочку и начинаете укладывать её с одного конца, в соответствии с геометрическими символами посредника.

\speak{Черепаха} Значит, каждый из этих символов обозначает особый тип укладки?

\speak{Ахилл} Сам по себе нет. Они всегда берутся группами по три. Вы начинаете с одного конца цепочки и с одного конца посредника. Первая тройка символов определяет, что делать с первым дюймом цепочки. Следующие три символа говорят вам, как укладывать второй дюйм. Таким образом, вы шаг за шагом продвигаетесь вдоль цепочки и вдоль посредника, укладывая~~каждый крохотный сегмент цепочки, пока посредник не кончится Если вы хорошенько смазали все рибосом, цепочка сохранит свою укладку и у вас получится трансляция коана в цепочку.

\speak{Черепаха} Эта процедура не лишена элегантности. Наверное, у вас получаются чертовски интересные цепочки.

\speak{Ахилл} ещё бы! Коаны подлиннее транслируются в весьма причудливые структуры.

\speak{Черепаха} Могу себе представить. Но чтобы транслировать посредник в цепочку вы должны знать, какой укладке соответствует каждая тройка геометрических символов. Откуда вы это знаете? У вас что, есть словарь?

\speak{Ахилл} Да \--- это замечательная книга, в которой приведен весь Геометрический Код. Если у вас этой книги нет, то, разумеется, вы не можете транслировать коаны в цепочки.

\speak{Черепаха} Разумеется нет. Каково происхождение Геометрического Кода?

\speak{Ахилл} Его начало восходит к древнему Мастеру по имени Великий Ментор, мой Мастер говорит, что он единственный, кто когда-либо достиг Архи-просветления.

\speak{Черепаха} Ах ты батюшки! Словно одного уровня мало Что ж, обжоры бывают всех сортов \--- почему бы не обжираться и просветлением?

\speak{Ахилл} А что, если в слове Архи-просветление закодировано мое имя? \mbox{А-Х-И-Л}.

\speak{Черепаха} По моему мнению, это маловероятно. Скорее, там можно найти намек на имя скромной ЧерепАХИ.

\speak{Ахилл} При чем здесь вы? Вы даже не достигли ПЕРВОГО состояния просветления, и уж тем более\ldots{}

\speak{Черепаха} Почем знать, почем знать. Может быть те, кто изучил всю подноготную просветления возвращаются в первоначальное, допросветленное состояние Я всегда считала, что дважды просветленный \--- это снова непросветленный. Но вернемся же к нашему Великому Ментору.

\speak{Ахилл} О нем известно очень мало \--- пожалуй, только то, что он изобрел Искусство Дзен-Цепочек.

\speak{Черепаха} Что это такое?

\speak{Ахилл} Это искусство на котором основана разрешающая процедура для определения буддистской природы. Я могу рассказать вам об этом поподробнее.

\speak{Черепаха} Буду счастлива. Новичкам вроде меня так много приходится выучить!

\speak{Ахилл} Говорят, что был даже специальный коан, повествующий о том, с чего началось Искусство Дзен-Цепочек. Но, к несчастью, он уже давным-давно уплыл по течению реки времен \--- а она, как известно, уносит навечно. Впрочем может быть это и неплохо \--- а то нашлись бы имитаторы, которые стали бы всячески копировать Мастера, пользуясь его именем.

\speak{Черепаха} Разве плохо, если бы все ученики дзен-будцизма стали бы копировать Великого Ментора \--- самого просветленного Мастера всех времен?

\speak{Ахилл} Позвольте вместо ответа рассказать вам коан об имитаторе.

\begin{koan}
    Мастер дзена по имени Гутей всегда поднимал палец когда его спрашивали о дзене. Молоденький ученик стал его копировать. Когда Гутей услышал об имитаторе, он позвал ученика и спросил правда ли это. «Да» \--- признался тот. Тогда Гутей спросил его понимает ли он, что делает. Вместо ответа ученик поднял указательный палец. Гутей быстро отрезал палец, вопя от боли ученик побежал к двери. Когда он достиг выхода Гутей позвал его: «Мальчик!» Ученик обернулся, и Гутей поднял свой указательный палец. В~этот момент юноша достиг Просветления.
\end{koan}

\speak{Черепаха} Кто бы мог подумать! Как раз когда я решил, что дзен \--- весь о Джошу и его проказах, оказалось, что и Гутей приглашен на праздник. Кажется, у него порядочное чувство юмора.

\speak{Ахилл} Этот коан совершенно серьезен; не знаю, откуда у вас появилась мысль, что в нем какой-то юмор.

\speak{Черепаха} Может быть, дзен так поучителен именно потому, что в нем много юмора. Мне кажется, что если воспринимать эти истории на полном серьезе, то в половине случаев их смысл пройдет мимо вас.

\speak{Ахилл} Может быть, в этом Черепашьем Дзене и есть какой-то смысл.

\speak{Черепаха} Можете ли вы ответить мне на один вопрос? Я хочу знать, почему Бодхидхарма приехал из Индии в Китай.

\speak{Ахилл} Ого! Хотите, я вам скажу, что ответил Джошу на точно такой же вопрос?

\speak{Черепаха} О, да!

\speak{Ахилл} Он ответил: «Дуб в саду.»

\speak{Черепаха} Разумеется; я сказала бы то же самое. С той разницей, что в моем случае это был бы ответ на другой вопрос: «Какое место лучше всего подходит, чтобы укрыться от полуденного солнца?»

\speak{Ахилл} Вы, сами того не подозревая, затронули сейчас один из основных вопросов дзена. Вопрос звучит безобидно: «Каков основной принцип дзена?»

\speak{Черепаха} Удивительно! Я и понятия не имела, что основная цель дзен-буддизма \--- в том, чтобы найти место в тенёчке.

\speak{Ахилл} Да нет же, вы меня совершенно не поняли. Я не имел в виду ЭТОТ вопрос. Я думал о первом вашем вопросе \--- почему Бодхидхарма приехал из Индии в Китай.

\speak{Черепаха} Понятно. Я и не знала, что ныряю на такую глубину\ldots{} Но вернемся к этим странным отображениям. Значит, любой коан может быть превращен в уложенную цепочку, следуя этому методу. А как насчёт обратного процесса? Можно ли прочитать любую цепочку так, чтобы получился коан?

\speak{Ахилл} В некотором роде. Однако\ldots{}

\speak{Черепаха} Что такое?

\speak{Ахилл} Вы просто не должны читать её таким образом. Это нарушило бы Центральную Догму Дзен-цепочек, которую можно нарисовать следующим образом \emph{(рисует на салфетке)}:
\[
    \tcbset{size=small,on line,before upper=\strut}
    \tcbox{коан}
    \underset{\text{~~транскрипция~~}}{\Longrightarrow}
    \tcbox{посредник}
    \underset{\text{~~трасляция~~}}{\Longrightarrow}
    \tcbox{уложенная цепочка}
\]

Идти против стрелок нельзя \--- особенно против второй стрелки.

\speak{Черепаха} Скажите мне: у этой догмы \--- природа Будды, или нет? Впрочем, если подумать, то я, пожалуй, могу развопросить этот вопрос. Если вы, конечно, не возражаете\ldots{}

\speak{Ахилл} Буду только рад. Я хочу открыть вам один секрет \--- поклянитесь, что никому не скажете!

\speak{Черепаха} Слово Черепахи.

\speak{Ахилл} Иногда я все-таки двигаюсь против стрелок. Запретный плод сладок, знаете~ли\ldots{}

\speak{Черепаха} Ай да Ахилл! Понятия не имела, что вы способны на такие непочтительные действия!

\speak{Ахилл} Я никому в этом не признавался \--- даже Оканисаме.

\speak{Черепаха} Так скажите мне, что получается, когда вы двигаетесь против стрелок Центральной Догмы? Это значит, что вы начинаете с цепочки и кончаете коаном?

\speak{Ахилл} Иногда \--- но часто случаются всякие странные вещи.

\speak{Черепаха} Более странные, чем производство коанов?

\speak{Ахилл} Да\ldots{} Когда вы делаете трансляцию и транскрипцию наоборот, у вас получается НЕЧТО, что не всегда является коаном. Некоторые цепочки, когда их читаешь вслух таким образом, звучат сплошной бессмыслицей.

\speak{Черепаха} Разве это не синоним коана?

\speak{Ахилл} Вижу, моя дорогая, что вы ещё не прониклись подлинным духом дзена.

\speak{Черепаха} По крайней мере, у вас хотя бы получаются рассказы?

\speak{Ахилл} Не всегда; иногда выходят бессмысленные слоги, иногда \--- предложения-окрошка. Но иногда выходит что-то, похожее на коан.

\speak{Черепаха} Только ПОХОЖЕЕ?

\speak{Ахилл} Видите ли, это может оказаться подделкой.

\speak{Черепаха} Ах, разумеется.

\speak{Ахилл} Я называю такие цепочки, которые производят коаны, «правильно сформированными.»

\speak{Черепаха} А как вы отличаете поддельные коаны от подлинных?

\speak{Ахилл} К этому я и веду. Имея коан (или не-коан, как иногда случается), первое, что надо сделать, \--- это транслировать его в трехмерную цепочку. Потом остается только выяснить, буддистская ли природа у этой цепочки.

\speak{Черепаха} Как же можно ухитриться проделать подобное?

\speak{Ахилл} Мой Мастер говорит, что Великий Ментор мог узнать это, просто взглянув на цепочку.

\speak{Черепаха} А если вы ещё не достигли Архи-просветления? Есть ли иной способ узнать, буддистская ли природа у данной цепочки?

\speak{Ахилл} Да, есть. Здесь как раз вступает в игру Искусство Дзен-цепочек. Этот способ \--- создание бесконечного множества цепочек с буддистской природой.

\speak{Черепаха} Да что вы говорите! А есть ли способ произвести цепочки БЕЗ буддистской природы?

\speak{Ахилл} Зачем это вам?

\speak{Черепаха} Я просто думала \--- а вдруг это может пригодиться\ldots{}

\speak{Ахилл} У вас весьма странный вкус. Надо же! Ей интереснее вещи не-буддистской природы, чем вещи с природой Будды!

\speak{Черепаха} Можете приписать это моему непросветленному состоянию.

\speak{Ахилл} Итак, сначала вы вешаете петлю цепочек на руки в одной из пяти дозволенных начальных позиций; например, вот так\ldots{} \emph{(Снимает длинную цепочку, висящую у него на шее, и надевает её на руки, набрасывая петли между пальцами.)}

\speak{Черепаха} Что представляют собой дозволенные позиции?

\speak{Ахилл} Каждая из них \--- это позиция, считающаяся самоочевидным способом брать цепочку. Даже новички часто берут цепочки именно так. И все эти пять цепочек имеют природу Будды.

\speak{Черепаха} Разумеется.

\speak{Ахилл} Кроме того, имеются некоторые Правила Обращения с Цепочками, следуя которым, можно произвести из цепочек более сложные фигуры. В частности, позволено изменять форму вашей цепочки при помощи простейших движений рук. Например, вы можете взяться за эту цепочку здесь и потянуть вот так \--- а теперь так перекрутить. Каждая операция меняет конфигурацию цепочки, надетой на ваши руки.

\speak{Черепаха} Это выглядит, как игра в веревочку \--- «колыбель для кошки» и прочие занимательные фигуры, которые можно сплести из веревки, надетой на пальцы.

\speak{Ахилл} Верно. Смотрите, некоторые из этих правил усложняют цепочку, а некоторые упрощают. Но неважно, в каком порядке вы это делаете; пока вы следуете Правилам Обращения с Цепочками, любая ваша цепочка будет иметь природу Будды.

\speak{Черепаха} Это чудесно. А как насчёт коана, спрятанного в строчке, что вы только что сплели? Будет ли он подлинным?

\speak{Ахилл} Согласно тому, что я выучил, именно так и будет. Поскольку я придерживался Правил и начал в одной из пяти самоочевидных позиций, цепочка должна иметь природу Будды и, следовательно, соответствовать подлинному коану.

\speak{Черепаха} Знаете ли вы, какому именно?

\speak{Ахилл} Вы хотите, чтобы я нарушил Центральную Догму? Ах вы, вредное создание!

\stage{\emph{(Ахилл раскрывает книгу Кода и, высунув от усердия язык, дюйм за дюймом продвигается вдоль цепочки, записывая каждый поворот с помощью тройки геометрических символов этого странного фонетического алфавита для коанов, пока салфетка не оказывается исписанной его каракулями)}}

Готово!

\speak{Черепаха} Здорово! Теперь давайте почитаем, что получилось.

\speak{Ахилл} Хорошо.

% TODO
\begin{koan}
    Путешествующий монах спросил у старухи дорогу к Тайзаиу, известному храму, превращающему тех, кто в нем молится, в мудрецов. Старуха ответила: «Идите прямо». Когда тот удалился, старуха пробормотала себе под нос: «Ещё один паломник». Кто-то рассказал об этом случае Джошу, и тот заметил: «Подождите, я сам проверю». На следующий день он отправился тем же путем и задал тот же вопрос. Старуха повторила свой ответ, и Джошу сказал: «Я проверил эту старую женщину».
\end{koan}

\speak{Черепаха} С его страстью к расследованиям, жаль, что Джошу никогда не работал в ФБР\@. А скажите, я могла бы повторить то, что вы сейчас сделали, если бы следовала Правилам Искусства Дзен-цепочек, не правда ли?

\speak{Ахилл} Совершенно верно.

\speak{Черепаха} Я должна буду проделывать все операции в том же ПОРЯДКЕ, как и вы?

\speak{Ахилл} Да нет, годится любой порядок.

\speak{Черепаха} Разумеется, тогда я получу другую цепочку и, следовательно, другой коан. Теперь скажите мне, я должна буду повторить то же ЧИСЛО операций?

\speak{Ахилл} Ни в коем случае. Вы можете делать любое число шагов.

\speak{Черепаха} В таком случае, есть бесконечное множество цепочек с природой Будды \--- а следовательно, бесконечное множество подлинных коанов! Но откуда вы знаете, есть ли какая-либо цепочка, которая НЕ МОЖЕТ быть получена при помощи ваших Правил?

\speak{Ахилл} Ах, да \--- вернемся к вещам, лишенным природы Будды. Получается так, что как только вы научитесь производить цепочки БУДДИСТСКОЙ природы, вы сразу же научитесь производить и HE-БУДДИСТСКИЕ цепочки. Это мой Мастер вдолбил в меня с самого начала.

\speak{Черепаха} Прекрасно! Как же это получается?

\speak{Ахилл} Очень просто. Вот, смотрите: сейчас я сделаю цепочку, у которой нет природы Будды\ldots{}

\stage{\emph{(Он берет цепочку, из которой был «извлечен» предыдущий коан, и завязывает на одном из концов неточку, затягивая её большим и указательным пальцами.)}}

Готово: в этой цепочке НЕТ никакой буддистской природы.

\speak{Черепаха} Потрясающе! Я просвещаюсь с каждой минутой. И всего-то понадобилась какая-то ниточка? Откуда вы знаете, что у новой цепочки нет буддистской природы?

\speak{Ахилл} Не ниточка, а НЕТОЧКА \--- именно так указал мастер. Основное свойство природы Будды таково: если две правильно сформированные цепочки отличаются только тем, что одна из них имеет неточку на конце, то только ОДНА из этих цепочек может иметь буддистскую природу.

\speak{Черепаха} А скажите: есть ли такие цепочки буддистской природы, которые НЕВОЗМОЖНО получить, в каком бы порядке мы не применяли Правила Дзен-цепочек?

\speak{Ахилл} Стыдно признаться, но этого я сам точно не знаю. Сначала мой мастер говорил, что буддистская природа цепочки ОПРЕДЕЛЕНА тем, что мы начинаем с одной из пяти начальных позиций и затем строго следуем Правилам. Но позже он сказал что-то о какой-то «Теореме», как бишь его\ldots{} Гоголя?., или Де Голля? Боюсь, что я так этого и не понял; а может быть, просто не расслышал. Но так или иначе, у меня появилось сомнение, можно ли получить этим методом ВСЕ цепочки с природой Будды. До сих пор мне это удавалось, но ведь буддистская природа \--- штука непростая, знаете ли\ldots{}

\speak{Черепаха} Я так и думала, судя по «МУ» Джошу. Хотелось бы мне знать\ldots{}

\speak{Ахилл} Что такое?

\speak{Черепаха} Я думала о тех двух коанах\ldots{} Я имею в виду, коан и не-коан: «Этот разум \--- Будда» и «Этот разум \--- не Будда». Как они выглядят, если перевести их в цепочки по Геометрическому Коду?

\speak{Ахилл} С удовольствием вам покажу.

\stage{\emph{(Он записывает фонетическую транскрипцию, достает из кармана пару цепочек и начинает аккуратно, дюйм за дюймом, складывать их, следуя тройкам символов, записанных странным алфавитом. Затем он кладет получившиеся цепочки рядом.)}}

Видите, они различаются.

\speak{Черепаха} На мой взгляд, они весьма схожи. О, теперь я вижу, в чем разница: на конце у одной из них \--- неточка!

\speak{Ахилл} Клянусь Джошу, вы правы.

\speak{Черепаха} Ага! Я понимаю теперь, почему ваш Мастер не доверял этим коанам.

\speak{Ахилл} Неужели?

\speak{Черепаха} Согласно его указаниям, НЕ БОЛЕЕ, ЧЕМ ОДНА цепочка из этой пары может иметь природу Будды; так что сразу можно сказать, что один из коанов \--- подделка.

\speak{Ахилл} Но это ещё не говорит нам, какой именно. Мы с моим Мастером давно пытаемся сложить эти цепочки, следуя Правилам; но у нас пока ничего не выходит. Это ужасно неприятно, и можно начать сомневаться\ldots{}

\speak{Черепаха} В том, что у этих цепочек вообще есть природа Будды? Может быть, её нет ни у одной цепочки, и оба коана поддельны?

\speak{Ахилл} Я никогда не заходил так далеко \--- но вы правы, в принципе это возможно. Однако вы не должны задавать так много вопросов о природе Будды. Мастер дзена Мумон всегда предупреждал своих учеников, что слишком много спрашивать опасно.

\speak{Черепаха} Хорошо \--- вопросов больше не будет. Но зато мне очень хочется самой уложить цепочку. Интересно посмотреть, получится ли она правильно сформированной.

\speak{Ахилл} И правда, интересно. Вот, пожалуйста. \emph{(Передает цепочку Черепахе.)}

\speak{Черепаха} Вы знаете, я понятия не имею, что с ней делать. Что ж, рискнем \--- мое неуклюжее произведение, сделанное без Правил, как Бог на душу положит, будет, скорее всего, совершенно невозможно расшифровать. \emph{(Берет цепочку, делает из нее петлю, и несколькими движениями лап укладывает цепочку в сложный узор, который затем молча протягивает Ахиллу. В этот момент лицо воина освещается.)}

\speak{Ахилл} Вот это да! Я должен попробовать этот метод сам. Никогда не видел подобной цепочки!

\speak{Черепаха} Надеюсь, что она правильно сформирована.

\speak{Ахилл} На одном конце у нее завязана неточка.

\speak{Черепаха} Ох, погодите \--- можно мне эту цепочку на минутку? Я хочу ещё кое-что сделать.

\speak{Ахилл} Почему бы~и нет \--- пожалуйста.

\stage{\emph{(Снова протягивает её Черепахе, та завязывает ещё одну неточку на том же конце. После этого она встряхивает цепочку и внезапно обе неточки исчезают!)}}

\speak{Ахилл} Что случилось?

\speak{Черепаха} Я просто хотела избавиться от той неточки.

\speak{Ахилл} Но вместо того, чтобы её развязать, вы завязали ещё одну, и тут их как ножом отрезало, обе исчезли! Куда они подевались?

\speak{Черепаха} В Лимбедламию, разумеется. Это Закон Двойного Отрезания.

\stage{\emph{(Вдруг обе неточки опять появляются ниоткуда \--- то бишь, из Лимбедламии.)}}

\speak{Ахилл} Удивительно. К некоторым районам Лимбедламии, видно, существует легкий доступ, если эти неточки могут так запросто проталкиваться и выталкиваться. Или же вся Лимбедламия одинаково недоступна?

\speak{Черепаха} Не могу вам сказать. Правда, я думаю, что если бы мы эту цепочку расплавили, то неточки вряд ли вернулись бы. В этом случае, мы считали бы, что они попали на более глубокий уровень Лимбедламии. Там, возможно, есть миллионы уровней. Но это для нас неважно. Меня сейчас интересует то, как эта цепочка зазвучит, если мы переведем её обратно в фонетические символы.

\speak{Ахилл} Я всегда чувствую себя виноватым, когда нарушаю Центральную Догму.

\stage{\emph{(Достает ручку и книгу Кода и аккуратно записывает тройные символы, соответствующие поворотам Черепашьей цепочки; когда все готово, он откашливается.)}}

Кхе-кхе. Послушаем, что у вас получилось\ldots{}

\speak{Черепаха} Если вы готовы\ldots{}

\speak{Ахилл} Отлично. Вот что тут написано:

\begin{koan}
    Один монах постоянно приставал к Великой Чепупахе (единственной, которая когда-либо достигла Архи-просветлеиия), спрашивая у нее, имеют ли те или иные вещи природу Будды. Чепупаха отвечала на эти вопросы молчанием. Монах уже спросил о бобе, озере, и лунной ночи. Однажды он принес Чепупахе кусочек цепочки и задал тот же вопрос. В~ответ Чепупаха взяла цепочку, сделала из нее петлю и несколькими движениями лап \---
\end{koan}

\speak{Черепаха} Несколькими движениями лап? Как странно!

\speak{Ахилл} Почему же именно Вы находите это странным?

\speak{Черепаха} Ах да, конечно, вы правы. Продолжайте, прошу вас!

\speak{Ахилл} Хорошо.

\begin{koan}
    Несколькими движениями лап Чепупаха уложила цепочку в сложный узор, который затем молча протянула монаху. В этот момент монах достиг Просветления.
\end{koan}

\speak{Черепаха} Что до меня, то я бы предпочла Архи-просветление.

\speak{Ахилл} Далее тут описывается, как сделать цепочку Великой Чепупахи, если начать с петли, наброшенной на лапы. Эти скучные детали я пропущу\ldots{} А вот и конец:

\begin{koan}
    С тех пор монах больше не приставал к Чепупахе. Вместо этого он укладывал цепочку за цепочкой по её методу; он передал этот метод своим ученикам, а те \--- своим.
\end{koan}

\speak{Черепаха} Ну и хитросплетение! Трудно поверить, что все это было спрятано в моей цепочке.

\speak{Ахилл} Так оно и есть. Удивительно, что вы смогли уложить правильно сформированную цепочку \--- верно говорят, что новичкам везет!

\speak{Черепаха} Но как же выглядела цепочка Великой Чепупахи? Мне кажется, в этом самая суть коана.

\speak{Ахилл} Сомневаюсь. Мы не должны «привязываться» к таким мелочам. Главное не детали, а дух коана как целого. А знаете, что мне только что пришло в голову? Я думаю, что вы, как это ни удивительно, только что наткнулись на давно утерянный коан, описывающий происхождение Искусства Дзен-цепочек!

\speak{Черепаха} О, это было бы слишком хорошо для того, чтобы иметь буддистскую природу!

\speak{Ахилл} Но это бы значило, что великий Мастер, единственный, кто достиг мистического состояния Архи-просветления, звался не Ментором, а Чепупахой. Вот уж поистине странное имя!

\speak{Черепаха} Я не согласна \--- по-моему, это очень красивое имя. Но я всё же хочу знать, как выглядела эта Чепупашья цепочка. Можете ли вы воссоздать её по описанию, данному в коане?

\speak{Ахилл} Я могу попытаться, хотя мне это будет очень трудно \--- ведь у меня нет лап, а в коане все описывается с точки зрения движения именно лап. Это очень необычно, но я постараюсь. Попытка \--- не пытка\ldots{}

\stage{\emph{(Он берет коан и кусочек цепочки и в течение нескольких минут, пыхтя от усердия, сгибает и складывает его самым невероятным образом, пока в его руках не оказывается готовый продукт.)}}

Вот, пожалуйста. Странно, но это выглядит очень знакомо.

\speak{Черепаха} И правда! Интересно, где я это видела?

\speak{Ахилл} Я знаю! Это же ВАША цепочка, разве не так?

\speak{Черепаха} Наверняка нет.

\speak{Ахилл} Ну конечно: это ваша первая цепочка, которую вы мне дали до того, как завязали вторую неточку.

\speak{Черепаха} Действительно, она самая. Надо же\ldots{} Интересно, что из этого следует?

\speak{Ахилл} Все это очень странно, чтобы не сказать большего.

\speak{Черепаха} Вы думаете, мой коан \--- подлинный?

\speak{Ахилл} Подождите-ка минутку\ldots{}

\speak{Черепаха} А эта цепочка \--- есть ли в ней природа Будды?

\speak{Ахилл} Ваша цепочка кажется мне подозрительной\ldots{}

\speak{Черепаха (с предовольным видом, не обращая на Ахилла никакого внимания)} А как насчёт Чепупашьей цепочки? Есть ли в ней природа Будды? У меня столько вопросов!

\speak{Ахилл} Я бы поостерегся задавать столько вопросов, г-жа Ч\@. Что-то здесь творится, и я совсем не уверен, что это мне нравится.

\speak{Черепаха} Грустно слышать; но я не понимаю, что вас тревожит?

\speak{Ахилл} Лучше всего это объясняет цитата из другого древнего Мастера дзен-буддизма по имени Киоген. Киоген сказал:

\begin{koan}
    Дзен подобен человеку, удерживающемуся зубами за ветку растущего над пропастью дерева. Руки и ноги его, не имея опоры, болтаются в воздухе. Под деревом стоит другой человек и спрашивает его: «Почему Бодхидхарма пришел из Индии в Китай?». Если человек на дереве не ответит, он изменит дзену, а если он ответит, то упадет и погибнет. Что ему делать?
\end{koan}

\speak{Черепаха} {\raggedright Ясно как день: ему надо оставить дзен и заняться молекулярной биологией.\par}

\end{dialogue}

\endgroup
\end{document}
