\documentclass[../main.tex]{subfiles}
\begin{document}

\DialogueChapter{Контрафактус}

\centerblock{
    \emph{Как-то субботним вечером Краб пригласил к себе нескольких друзей, чтобы посмотреть футбол по телевизору. Ахилл уже пришел, но Черепаха и её приятель Ленивец запаздывают.}
}

\begin{dialogue}

\speak{Ахилл} Не они ли это едут на странном одноколесном аппарате?

\stage{\emph{(Ленивец и Черепаха подъезжают, спрыгивают на землю и входят в дом)}}

\speak{Краб} Друзья мои, я так рад, что вы наконец здесь. Позвольте представить вам моего старого доброго товарища, Ленивца. Это Ахилл. С Черепахой, я думаю, вы все знакомы.

\speak{Ленивец} Я никогда раньше не встречал Бициклопа. Приятно с вами познакомиться, Ахилл. Я слышал много хорошего о Бициклопном роде.

\speak{Ахилл} Очень приятно. Могу ли я спросить вас, что это за элегантное средство передвижения, на котором вы прибыли?

\speak{Черепаха} Вы имеете в виду мой одноколесный тандем? Что же в нем элегантного? Просто машина, позволяющая двоим добраться от А до Б с одинаковой скоростью.

\speak{Ленивец} Он сделан той же компанией, которая производит мотоциклы «Зиг-зиг».

\speak{Ахилл} А, понятно. Для чего эта ручка?

\speak{Ленивец} Это переключатель скоростей.

\speak{Ахилл} Ага; и сколько же у этого аппарата скоростей?

\speak{Ленивец} Включая задний ход, одна. У большинства моделей скоростей меньше, но эта была сделана по спецзаказу.

\speak{Ахилл} Да, этот моно-тандем кажется отличной штукой\ldots{} Кстати, м-р Краб, хотел вам сказать, что вчера вечером я получил неописуемое удовольствие от игры вашего оркестра.

\speak{Краб} Благодарю вас, Ахилл. А вы там не были, м-р Ленивец?

\speak{Ленивец} Нет, к сожалению, я не смог пойти. Я участвовал в смешанном одиночном турнире по пинг-пингу. Это было захватывающе интересно, моя команда разделила первое место сама с собой.

\speak{Ахилл} Вы получили какой-нибудь приз?

\speak{Ленивец} Конечно \--- медный двухсторонний лист Мёбиуса, посеребренный с одной стороны и позолоченный с другой.

\speak{Краб} Поздравляю вас, м-р Ленивец.

\speak{Ленивец} Благодарю. Прошу вас, расскажите мне об этом концерте.

\speak{Краб} Это было замечательно, мы играли композиции близнецов Бах \---

\speak{Ленивец} Близнецы Бах? Знаменитые Иоганнс и Бастиан?

\speak{Краб} Они самые, схожие как две капли воды \--- словно один и тот же человек. Одна из пьес напомнила мне о вас, м-р Ленивец, \--- прелестный фортепианный концерт для двух левых рук. Его предпоследней (и единственной) частью была одноголосная фуга. Вы можете представить себе, насколько сложна подобная композиция. В завершение мы сыграли девятую дзенфонию Бетховена, после чего публика устроила нам овацию, хлопая одной рукой. Это было потрясающе!

\speak{Ленивец} Как жаль, что я такое пропустил. Как вы думаете, можно ли достать запись этого концерта? У меня есть отличный магнитофон \--- лучшая двухканальная моносистема, которую только можно достать за деньги.

\speak{Краб} Я уверен, что вы сможете найти где-нибудь эту запись. Друзья, игра вот-вот начнется!

\speak{Ахилл} Кто сегодня играет, м-р Краб?

\speak{Краб} Хозяева Поля против Местной Команды. Ах, нет \--- они играли на прошлой неделе. Сегодня Местная Команда играет против Гостей.

\speak{Ахилл} Я, как всегда, болею за Местную Команду.

\speak{Ленивец} Как банально\ldots{} Что до меня, то я всегда болею за ту команду, которая живет ближе всего к антиподам.

\speak{Ахилл} А, так вы из Антиподов? Я слышал, что это прелестное местечко, чтобы там жить, но мне бы не хотелось там побывать. Слишком уж далеко\ldots{}

\speak{Ленивец} Странно то, что в какую бы сторону вы ни ехали, вы к ним ничуть не приближаетесь!

\speak{Черепаха} Ну и местечко \--- как раз в моем вкусе!

\speak{Краб} Пора включать телевизор!

\stage{\emph{(Он подходит к огромному ящику с экраном, под которым находится панель управления, напоминающая по сложности приборную доску самолета. Краб нажимает на кнопку. На экране оживает цветное изображение футбольного поля.)}}

\speak{Комментатор} Добрый вечер, дорогие телезрители. Пришло время снова встретиться на футбольном поле Местной Команде с Командой Гостей; сейчас мы станем свидетелями еще одного дружеского, но непримиримого матча. Сегодня весь день накрапывал дождь, и поле мокрое; но, несмотря на погоду, матч обещает быть захватывающе интересным. Обратите внимание на великолепную четверку четвертьзащитников, играющих за Местную Команду: Буратинов, Бибигонов, Незнайкин и Дюймовочкин, краса и гордость клуба «Унинамо». А вот и защитник наших ворот, великолепный Пилипик. Свисток\ldots{} игра началась! Мяч у Бузюлюкина\ldots{} Местные переходят в наступление, Бузюлюкин передает мяч Голяшкину, тот пасует Чебурашкину\ldots{} Удар по воротам! Голкиперу Гостей на этот раз удается спасти свою команду.

\speak{Краб} Великолепная атака! Видели, как Чебурашкин ПОЧТИ забил гол \--- но Стопкошкину удалось каким-то чудом отбить мяч?

\speak{Ленивец} Не говорите глупостей, Краб. Ничего подобного не случилось. Чебурашкин не забил никакого гола; не надо смущать бедного Ахилла (и всех остальных) этими разговорчиками о том, что «почти» произошло. Факты есть факты, безо всяких там «почти», «если бы», «чуть было не», «и» и «но».

\speak{Комментатор} Передаем повтор: Чебурашкин получает пас Голяшкина, обводит защитника Гостей, бьет по воротам\ldots{} Немного левее, и сейчас счет был бы 1-0 в пользу Местной Команды!

\speak{Ленивец} «Был бы!» Чепуха!

\speak{Ахилл} Какая блестящая атака! Что бы мы делали без повторов?

\speak{Комментатор} Мяч опять у Галочкина; тот продвигается к воротам, пытается обойти Фисташкина, пасует Бибигонову\ldots{} Мяч вне игры. В наступление переходит команда Гостей; Фисташкин передает мяч назад, Семечкину, Семечкин находит Арахиса, стоящего еще ближе к собственным воротам, и Арахис пасует своему вратарю. Тройная передача назад!

\speak{Ленивец} Какая техника! Прекрасный матч, друзья, есть на что посмотреть!

\speak{Ахилл} Но я думал, что вы болеете за Команду Гостей, а они только что упустили отличный шанс.

\speak{Ленивец} Да? Не все ли равно, если команды играют интересно? Вот и повтор: посмотрим этот момент еще раз.

\stage{\emph{(\ldots так проходит первый период. В начале второго периода счет становится 1:0 в пользу Гостей; Местная Команда пытается сравнять счет. Возможность для этого появляется, когда Чебурашкин перехватывает мяч.)}}

\speak{Комментатор} Чебурашкин продвигается к воротам гостей, обводит Фисташкина\ldots{} Арахис пытается перехватить мяч, но Чебурашкин пасует Бибигонову, затем мяч переходит к Голяшкину\ldots{} Все усилия Гостей остановить атаку бесполезны. Мяч у Бузюлюкина; тот уже у самых ворот\ldots{} Наступил тот момент, которого так долго ждали «Унинамовцы»! Удар по воротам\ldots{} но что это? Бузюлюкин скользит на мокрой траве, теряет равновесие\ldots{} и мяч вне игры! Какой шанс утерян! Если бы Бузюлюкин не потерял равновесия, местной команде удалось бы сравнять счет! Давайте посмотрим гипотетический повтор.

\stage{\emph{(И на экране появляется то же расположение игроков, как минуту \mbox{назад}.)}}

Чебурашкин продвигается к воротам гостей, обводит Фисташкина\ldots{} Арахис пытается перехватить мяч, но Чебурашкин пасует Бибигонову, затем мяч переходит к Голяшкину\ldots{} Все усилия Гостей остановить атаку бесполезны. Мяч у Бузюлюкина; тот уже у самых ворот\ldots{} Наступил тот момент, которого так долго ждали «Унинамовцы»! Удар по воротам\ldots{} Но что это? Бузюлюкин скользит на мокрой траве, почти теряет равновесие\ldots{} но ему удается удержаться на ногах, и мяч летит прямо в ворота! Стопкошкин бросается в угол\ldots{} Слишком поздно! го-о-ол! Так, дорогие болельщики, проходила бы игра, если бы Бузюлюкин не потерял равновесия.

\speak{Ахилл} Минуточку: так был гол или не был?

\speak{Краб} Да нет \--- это был всего лишь гипотетический повтор. Они просто показали, как могла продолжаться игра.

\speak{Ленивец} В жизни не слыхал подобной чепухи! Того и гляди, они скоро додумаются до изобретения цементных шарфов и кружевных валенок!

\speak{Черепаха} Гипотетические повторы \--- штука довольно редкая, не правда ли?

\speak{Краб} Я бы не сказал, если вы смотрите Гипо-ТВ.

\speak{Ахилл} Гипо-ТВ? Это что, один из тех дорогущих аппаратов с гипертрофированным экраном?

\speak{Краб} Да нет, это такая модель телевизора, которая может переходить на гипотетический режим. Очень удобно, особенно когда смотришь футбол, хоккей и тому подобные вещи; я купил свой гипо-ТВ совсем недавно.

\speak{Ахилл} Почему здесь так много всяких кнопок и ручек?

\speak{Краб} Для настройки на нужную программу. В гипотетическом режиме есть множество программ, и эта панель дает возможность с легкостью выбирать между ними.

\speak{Ахилл} Покажите нам, пожалуйста, как она работает. Боюсь, что я не совсем понял, что это за штука \--- «передача в гипотетическом режиме», и с чем её едят.

\speak{Краб} О, это совсем нетрудно; вы можете сами во всем разобраться. Кстати, раз уж вы упомянули о еде\ldots{} Пожалуй, я пойду на кухню и поджарю несколько блинчиков \--- я знаю, что Ленивец к ним неравнодушен.

\speak{Ленивец (причмокивая)} Вот спасибо, Крабушка! Блинчики \--- моя любимая еда!

\speak{Краб} Как насчет остальных?

\speak{Черепаха} Я бы не отказалась от нескольких штук.

\speak{Ахилл} Я тоже. Но погодите \--- прежде, чем идти на кухню, скажите: чтобы использовать гипо-ТВ, нужно знать какой-нибудь специальный трюк?

\speak{Краб} Нет, все очень легко: продолжайте смотреть матч, и каждый раз, когда что-нибудь почти случается, или когда вам хочется, чтобы игра пошла иначе, просто начинайте крутить ручки и смотрите, что получится. Вреда от этого не будет, разве что вы поймаете какой-нибудь экзотический канал.

\stage{\emph{(И он исчезает на кухне.)}}

\speak{Ахилл} Интересно, что он имел в виду. Ну ладно, давайте дальше смотреть игру \--- я ею здорово увлекся.

\speak{Комментатор} Местные опять переходят в наступление. С мячом Буратинов; он посылает мяч к воротам противника\ldots{} Какой пас! Мяч летит через все поле, прямо к Бибигонову \---

\speak{Ахилл} Давай, Бибигонов! Покажи им, где раки зимуют!

\speak{Комментатор} \--- и приземляется в лужу \--- ПЛЮХ! Мяч отскакивает в другую сторону, минуя Бибигонова, и попадает к Фисташкину; Фисташкин передает мяч Семечкину, Семечкин бьет по воротам\ldots{} Пилипик пытается спасти положение, но мяч летит в верхний угол ворот. Гол! Счет становится 2:0 в пользу Гостей.

\speak{Ахилл} О, черт! Если бы только не дождь\ldots{} \emph{(в отчаянии ломает руки)}.

\speak{Ленивец} Снова эти дурацкие гипотетические ситуации! Почему вы все так любите уходить в абсурдные фантастические миры? На вашем месте, я бы не стал витать в облаках. Мой лозунг \--- «Никаких гипотетических глупостей!» И я не отказался бы от него, даже если бы мне предложили мне за это 100 \--- нет, лучше 112! \--- блинчиков!

\speak{Ахилл} Идея! Может быть, нужным образом повернув ручки, мы получим гипотетический повтор, в котором нет дождя и луж на поле и мяч не отскакивает к Фисташкину\ldots{} Ну-ка посмотрим. \emph{(Направляется к гипо-ТВ и смотрит на панель.)} Понятия не имею, для чего все это. \emph{(Наугад поворачивает несколько ручек.)}

\speak{Комментатор} Местные опять переходят в наступление. С мячом Буратинов; он посылает мяч к воротам противника\ldots{} Какой пас! Мяч летит через все поле, прямо к Бибигонову \---

\speak{Ахилл} Давай, Бибигонов! Покажи им, где раки зимуют!

\speak{Комментатор} \--- и приземляется в лужу \--- ПЛЮХ! Мяч отскакивает прямо под ноги Бибигонову! Удар по воротам\ldots{} Го-о-о-л! \emph{(Слышны восторженные вопли болельщиков Местной Команды.)} Вот так матч проходил бы, если бы вместо кожаного мяч был бы резиновым, как в баскетболе. Но в действительности Местная Команда теряет мяч, и Гости забивают второй гол. Что ж, такова жизнь\ldots{}

\speak{Ахилл} Что вы думаете об ЭТОМ, м-р Ленивец?

\stage{\emph{(И Ахилл самодовольно ухмыляется в сторону Ленивца \--- однако тот на него даже не смотрит; все его внимание поглощено Крабом, который выходит из кухни, неся огромное блюдо со ста двенадцатью\ldots{} нет, с сотней блинчиков и с блюдечками, полными сахара и варенья.)}}

\speak{Краб} Ну как, что вы думаете о моем гипо-ТВ?

\speak{Ленивец} Откровенно говоря, Краб, я в нем совершенно разочаровался. По-моему, он здорово барахлит \--- по меньшей мере в половине случаев показывает совершенную бессмыслицу. Если бы он принадлежал мне, я бы его сразу отдал кому-нибудь вроде вас, Краб, \--- но, разумеется, он мне не принадлежит.

\speak{Ахилл} Это очень странный аппарат. Я попытался посмотреть, как проходила бы игра при другой погоде \--- но эта штука, кажется, себе на уме! Вместо того, чтобы изменить погоду, она поменяла мяч с футбольного на баскетбольный. Ну скажите мне, как можно играть в футбол баскетбольным мячом? Чушь какая-то!

\speak{Краб} Какая скука! Я-то думал, вы найдете себе гипотетический канал поинтереснее. Хотите посмотреть, как выглядел бы последний матч, если бы вместо футбола это был баскетбол?

\speak{Черепаха} Отличная идея!

\stage{\emph{(Краб вертит ручки настройки.)}}

\speak{Комментатор} В наступление переходит великолепная шестерка Местной Команды. Мяч летит к Бибигонову \---

\speak{Ахилл} Шестерка!?

\speak{Комментатор} Именно так, друзья \--- шестерка. Когда вы превращаете футбол в баскетбол, приходится идти на компромисс! Итак, как я говорил, мяч летит к Бибигонову, который стоит неприкрытый вблизи от кольца Гостей.

\speak{Ахилл} Давай, Бибигонов! Покажи им, где крабы зимуют!

\speak{Комментатор} Но бросок был неточен, и мяч падает прямо перед Фисташкиным; Фисташкин ведет мяч, передает Арахису, тот минует Дюймовочкина\ldots{} еще два очка в пользу команды Гостей! Сегодня Местной Команде не везет\ldots{} Итак, друзья, так выглядел бы этот матч, если бы команды играли в баскетбол вместо футбола.

\speak{Ленивец} Ничего себе! Вы бы еще перенесли этот матч на луну!

\speak{Краб} Сказано \--- сделано! Слегка подкрутим эту ручку\ldots{} теперь вот эту\ldots{}

\stage{\emph{(На экране появляется испещренное кратерами поле, на котором стоят две команды в скафандрах. Внезапно они приходят в движение; игроки передвигаются длинными прыжками, иногда перелетая над головой друг у друга. Один из игроков бьет по воротам; мяч взлетает в воздух, так высоко, что его почти не видно, и плавно опускается прямо в руки к вратарю.)}}

\speak{Комментатор} Этот гипотетический повтор, друзья, показывает вам, как проходила бы игра на луне. А теперь \--- небольшая реклама, приготовленная для вас теми, кто производит пиво Плюх \--- мой любимый сорт!

\speak{Ленивец} Если бы мне не было лень, я бы собственноручно сдал этот дефектный телевизор обратно в магазин. Но увы, такова уж моя судьба \--- быть Ленивцем\ldots{} \emph{(Протягивает лапу к блюду с блинчиками и хватает сразу несколько штук.)}

\speak{Черепаха} Это замечательное изобретение, м-р Краб. Могу ли я предложить еще один гипотетический повтор?

\speak{Краб} Разумеется!

\speak{Черепаха} Как бы выглядел этот матч в четырехмерном пространстве?

\speak{Краб} О, это не так просто, г-жа Черепаха, но для вас я попытаюсь настроить телевизор\ldots{} подождите минутку.

\stage{\emph{(Он подползает к телевизору и начинает крутить ручки, на этот раз, по-видимому, выжимая из своего гипо-ТВ все, на что тот способен, нажимая на все мыслимые кнопки и не спуская глаз со шкалы настройки. Наконец, он отходит от аппарата с довольным видом.)}}

Думаю, что этого будет достаточно.

\speak{Комментатор} А теперь давайте посмотрим гипотетический повтор.

\stage{\emph{(На экране появляется изображение странной конфигурации изогнутых трубок. Она растет, затем уменьшается, и на секунду кажется, что она делает нечто вроде поворота. Потом она превращается в странный грибовидный объект \--- и затем снова в переплетение трубок. Пока с ней происходят эти метаморфозы, комментатор продолжает.)}}

\speak{Комментатор} Бибигонов передает гипермяч Бузюлюкину, тот приближается к штрафному объему. Удар по гиперворотам\ldots{} Гооол! Вот так, мои трехмерные друзья-болельщики, выглядел бы футбол в четырех измерениях.

\speak{Ахилл} Для чего, м-р Краб, вы вертите эти ручки на панели?

\speak{Краб} Чтобы выбрать нужный гипотетический канал. Видите ли, передача ведется одновременно по множеству гипотетических каналов, и я хочу выбрать именно тот канал, который передает предложенный вами гипо-повтор.

\speak{Ахилл} А как насчет других телевизоров \--- возможно ли там такое?

\speak{Краб} Нет, большинство телевизоров не улавливают гипотетических каналов. Для этого нужна специальная схема, которую очень трудно сделать.

\speak{Ленивец} Откуда вы знаете, что идет по определенному каналу? Смотрите в программе ?

\speak{Краб} Мне не нужно знать номеров каналов \--- я настраиваюсь на нужный канал, вводя цифровой код гипотетической ситуации, которую я хочу увидеть. Технически это называется «адресацией канала по его контрафактическим параметрам». По гипотетическим каналам можно увидеть любой воображаемый мир. Номера каналов, передающие «близкие» друг к другу миры, также близки.

\speak{Черепаха} Почему вы даже не подходили к ручкам, когда мы смотрели первый гипо-повтор?

\speak{Краб} Телевизор был настроен на канал, очень близкий к Реальности, только чуть-чуть сдвинутый в сторону. Так что время от времени там возникают гипотетические повторы, слегка отличающиеся от реальности. На Канал Реальности, знаете ли, почти невозможно настроиться точно \--- впрочем, это даже хорошо, поскольку там нет ничего интересного. Представляете себе, они повторяют в точности те же ситуации, которые возникают в игре \--- ну и скучища!

\speak{Ленивец} Что до меня, что я нахожу прескучной именно эту идею гипо-ТВ. Но, может быть, я мог бы изменить свое мнение, если бы вы показали мне хоть один ИНТЕРЕСНЫЙ гипо-повтор. Скажем, как проходила бы игра, если бы сложение не было коммутативным?

\speak{Краб} Ах, боже мой! Это слишком радикально меняет ситуацию \--- боюсь, что моей модели с таким заказом не справиться. Такое под силу только супергипо-ТВ \--- последнему слову гипо-телевидения; но, к несчастью, у меня его нет. Супергипо-ТВ способны выполнить ЛЮБУЮ просьбу.

\speak{Ленивец} Подумаешь\edots

\speak{Краб} Но я могу сделать что-то ПОДОБНОЕ \--- не хотите ли вы увидеть, как например, проходила бы игра, если бы 13 не было простым числом?

\speak{Ленивец} Нет уж, увольте! ЭТО-ТО совершенно бессмысленно. К тому же, на месте этой последней игры, я бы уже устал от того, как компания путаников, у которых каша в голове, перебрасывает меня с канала на другой. Давайте-ка лучше досмотрим настоящий матч!

\speak{Ахилл} Скажите, а где вы достали ваш гипо-ТВ, м-р Краб?

\speak{Краб} Верите ли, мы с Ленивцем недавно были на ярмарке, где разыгрывалась лотерея \--- и первым призом был этот телевизор. Обычно я в подобные игры не играю, но тут что-то на меня накатило, и я купил билетик.

\speak{Ахилл} А как насчет вас, м-р Ленивец?

\speak{Ленивец} Признаюсь, и я купил один билет \--- только затем, чтобы ублажить старика Краба\ldots{}

\speak{Краб} И когда выигрышный номер был объявлен, к моему удивлению оказалось, что первый приз достался мне!

\speak{Ахилл} Фантастика! Мне еще не приходилось своими глазами видеть человека, выигравшего что бы то ни было в лотерею.

\speak{Краб} Я и сам был поражен своей удаче.

\speak{Ленивец} Не забыли ли вы рассказать еще кое-что об этой лотерее, м-р Краб?

\speak{Краб} О, ничего существенного\ldots{} Дело в том, что номер моего билета был~129, а выигрышным билетом был объявлен номер~128 \--- разница всего на единицу.

\speak{Ленивец} Так что, как видите, на самом деле он ничего не выиграл.

\speak{Ахилл} Но он ПОЧТИ выиграл.

\speak{Краб} Я предпочитаю говорить, что я выиграл \--- ведь я был к этому так близок\ldots{} Если бы мой номер был на единицу меньше, выигрыш достался бы мне.

\speak{Ленивец} Но, к несчастью, м-р Краб, «почти» не считается, и в данном случае совершенно все равно, единица это была или сотня.

\speak{Черепаха} Или бесконечность. А какой номер достался ВАМ, м-р Ленивец?

\speak{Ленивец} У меня был номер 256 \--- после 128 это следующая степень~2. Если кто-нибудь и был близок к выигрышу, так это я! К сожалению, устроители лотереи, эти упрямцы, отказались выдать мне мой заслуженный приз. Какой-то шутник сказал, что приз должен был принадлежать ЕМУ, поскольку у него оказался номер~128. Я думаю, что МОЙ номер был намного ближе к выигрышному \--- но разве этих бюрократов переспоришь!

\speak{Ахилл} Подождите, вы меня совсем запутали! Если, на самом деле, м-р Краб, вы не выиграли гипо-телевизора, то как же мы могли провести перед ним целый вечер? Словно мы и сами оказались в каком-то гипотетическом мире, который был бы возможен, если бы обстоятельства оказались слегка иными\ldots{}

\speak{Комментатор} Так, друзья, прошел бы вечер в доме м-ра Краба, если бы он выиграл гипо-ТВ. Но поскольку этого не случилось, четверка приятелей просто провела приятный вечер, глядя, как Местная Команда была разбита в пух и прах со счетом 128\==0. Или же счет был 256\==0? Впрочем, какая разница, когда речь идет о пятимерном Плутонском хоккее на пару\ldots{}

\end{dialogue}

\end{document}
