\documentclass[../main.tex]{subfiles}
\begin{document}

\DialogueChapter{Канон Ленивца}

\centerblock{
    \emph{На этот раз мы находим Ахилла и Черепаху Тортиллу в гостях у их нового приятеля, Ленивца Сплюшки.}
}

\begin{dialogue}

\speak{Ахилл} Хотите послушать про забавное соревнование по бегу, которое мы однажды устроили с Черепахой?

\speak{Ленивец} О, да, прошу вас!

\speak{Ахилл} Это соревнование стало довольно известным \--- я слышал, что оно даже было описано неким Зеноном.

\speak{Ленивец} Это интересно, и я всегда настроен вас послушать.

\speak{Ахилл} Это, и правда, было интересно. Видите ли, г-жа Тортилла стартовала первой. У нее была огромная фора, и тем не менее \---

\speak{Ленивец} Вы её нагнали, не правда ли?

\speak{Ахилл} Разумеется \--- поскольку я такой быстроногий, я сокращал расстояние между нами с постоянной скоростью, и вскоре обогнал её.

\speak{Ленивец} Поскольку расстояние между вами становилось всё меньше и меньше, вам удалось это сделать.

\speak{Ахилл} Именно. О, смотрите \--- г-жа Черепаха принесла свою скрипку. Можно мне попробовать что-нибудь сыграть, г-жа Черепаха?

\speak{Черепаха} О, нет, прошу вас. Это будет неинтересно \--- она расстроена так, что невозможно слушать.

\speak{Ахилл} Ну ладно. Но у меня сегодня почему-то музыкальное настроение.

\speak{Ленивец} Можете поиграть на пианино, Ахилл.

\speak{Ахилл} Благодарю вас. Немного погодя я так и сделаю. Но сначала доскажу, что потом мы с Черепахой бегали наперегонки еще раз. К несчастью, в этом соревновании \---

\speak{Черепаха} Вы меня не нагнали, не правда ли? Расстояние между нами становилось всё больше и больше, так что вам не удалось этого сделать.

\speak{Ахилл} Это верно. Мне кажется, что и ЭТО соревнование было описано \--- неким Льюисом Кэрроллом. А теперь, м-р Сплюшка, я готов принять ваше любезное приглашение и сыграть что-нибудь на пианино. Но предупреждаю: я очень плохой пианист \--- даже не знаю, стоит ли мне пытаться.

\speak{Ленивец} Вы должны попытаться.

\stage{\emph{(Ахилл садится за пианино и начинает играть простенькую мелодию.)}}

\speak{Ахилл} Ой, как странно звучит! Это совершенно не то, что я хотел сыграть! Что-то здесь не в порядке.

\speak{Черепаха} Вы не можете играть на пианино, Ахилл. Вы не должны и пытаться.

\speak{Ахилл} Это похоже на отражение пианино в зеркале. Высокие ноты находятся слева, а низкие \--- справа. Каждая мелодия получается перевернутой с ног на голову. Интересно, кто это придумал такую дурацкую шутку?

\speak{Черепаха} Этим отличаются ленивцы. Они висят \---

\speak{Ахилл} Да, я знаю: на ветвях деревьев \--- головой вниз, разумеется. Это пианино-ленивец годится, чтобы играть на нем перевернутые мелодии, которые встречаются в канонах и фугах. Но научиться играть на пианино, свисая с дерева, нелегко \--- это требует немалого трудолюбия.

\speak{Ленивец} Этим ленивцы не слишком отличаются.

\speak{Ахилл} Да, мне кажется, что ленивцы не любят утруждать себя. Они делают всё вдвое медленнее, чем все остальные. И кроме того, вверх ногами. Какой своеобразный жизненный уклад! Кстати, о замедленных и перевернутых вещах \--- в «Музыкальном приношении» есть канон под названием «Canon per augmentationem contrario motu», что значит «увеличенный и перевернутый канон». В моем издании «Приношения» перед тремя нотными строчками стоят три буквы \--- «S»,~«А» и~«T». Непонятно, что они значат\ldots{} Так или иначе, Бах сделал всё это очень ловко. Как вы считаете, г-жа Тортилла?

\speak{Черепаха} Он превзошел самого себя. Что касается букв «S»,~«А» и~«T», то я догадываюсь, что они означают.

\speak{Ахилл} «Сопрано», «Альт» и «Тенор», скорее всего. Трехчастные пьесы часто пишутся для этой комбинации голосов. Как вы думаете, м-р Сплюшка?

\speak{Ленивец} Они означают \---

\speak{Ахилл} О, подождите минутку! Г-жа Черепаха, почему вы одеваетесь? Надеюсь, вы не хотите нас покинуть? Мы только что собирались приготовить блинчики\ldots{} Вы выглядите очень усталой. Как вы себя чувствуете?

\speak{Черепаха} Обессиленной. Пока! \emph{(Устало ползет к двери.)}

\speak{Ахилл} Бедняжка \--- она, действительно, выглядит измученной. Она пробегала всё утро \--- тренировалась для следующего соревнования со мной.

\speak{Ленивец} Видно, она превозмогала саму себя.

\speak{Ахилл} И совершенно зря. Может быть, она могла бы перегнать Сплюшку\ldots{} но меня? Никогда! Да, вы, кажется, хотели сказать, что означают буквы «S»,~«А»,~«T»?

\speak{Ленивец} Вам в жизни не догадаться!

\speak{Ахилл} Неужели они могут значит что-то еще? Вы меня заинтриговали. Я еще над этим поразмыслю. Скажите, а на каком молоке вы делаете тесто для блинчиков?

\speak{Ленивец} На обезжиренном.

\speak{Ахилл} Хорошо. Я ставлю сковородку на огонь.

\speak{Ленивец} Уже?

\speak{Ахилл} Ну ладно, тогда сначала размешаю тесто. Какие блинчики будут \--- пальчики оближешь! Жаль, что Черепахе не удастся их отведать.

% TODO: illustration 133
\emph{Рис. 133. «Канон Ленивца», из «Музыкального Приношения» И.\,С.~Баха. (Ноты напечатаны с помощью компьютерной программы СМУТ Дональда Бирда.)}

\end{dialogue}

\end{document}
