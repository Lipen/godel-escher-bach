\documentclass[../main.tex]{subfiles}
\enabletikzpreview%
\begin{document}%
% This diagram is being used in Chapter 5.
\begin{tikzpicture}[
    thick,
    font=\bfseries,
    ->-/.style={
        decoration={
            markings,
            mark=at position 0.6 with {\arrow{Stealth[scale=1.5]}}
        },
        postaction={decorate},
    },
]

\node[draw,ellipse,minimum height=1cm] (start) at (0,0) {\textit{начало}};
\node[draw,align=center] (adj) at (3,0) {ПРИЛАГА-\\ТЕЛЬНОЕ};
\node[draw,align=center] (noun) at (6,0) {СУЩЕСТВИ-\\ТЕЛЬНОЕ};
\node[draw,ellipse,minimum height=1cm] (end) at (9,0) {\textit{конец}};

\draw[->-] (start) -- (adj);
\draw[->-] (adj) -- (noun);
\draw[->-] (noun) -- (end);

\path (start) ++(1.5,1) coordinate (a);
\path (adj) ++(-0.8,1) coordinate (b);
\draw (start.30) to[out=45,in=180] (a);
\draw[->-] (a) -- (b);
\draw (b) to[out=0,in=135] (adj.90);

\path (start) ++(1.5,1.5) coordinate (c);
\path (noun) ++(-1.3,1.5) coordinate (d);
\draw (start.120) to[out=45,in=180] (c);
\draw[->-] (c) -- (d);
\draw (d) to[out=0,in=135] (noun.90);

\draw (adj.-40)++(0,\pgflinewidth/2) arc (90:-90:0.25) coordinate (e);
\draw (adj.-140)++(0,\pgflinewidth/2) arc (90:270:0.25) coordinate (f);
\draw[->-] (e) -- (f);

\end{tikzpicture}%
\end{document}
