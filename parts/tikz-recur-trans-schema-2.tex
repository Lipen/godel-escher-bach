\documentclass[../main.tex]{subfiles}
\enabletikzpreview%
\begin{document}%
% This diagram is being used in Chapter 5.
\begin{tikzpicture}[
    thick,
    font=\bfseries,
    ->-/.style={
        decoration={
            markings,
            mark=at position #1 with {\arrow{Stealth[scale=1.5]}}
        },
        postaction={decorate},
    },
    ->-/.default=0.6,
]

\node[draw,ellipse,minimum height=1cm] (start) {\textit{начало}};
\path (start.east) ++(0.8,0) node[draw,align=center,anchor=west,font=\small] (fancy-noun) {УКРАШЕННОЕ \\ СУЩЕСТВИ-\\ТЕЛЬНОЕ};
\path (fancy-noun.east) ++(1,-2) node[draw,align=center,anchor=west,minimum height=1cm] (prep) {ПРЕДЛОГ};
\path (prep.east) ++(1.5,0) node[draw,align=center,minimum height=1cm,anchor=west,font=\small] (super-fancy-noun) {СВЕРХУКРАШЕННОЕ \\ СУЩЕСТВИТЕЛЬНОЕ};
\path (fancy-noun.east) ++(9,0) node[draw,ellipse,minimum height=1cm,anchor=west] (end) {\textit{конец}};
\path (fancy-noun.east) ++(0,3) node[draw,align=center,anchor=west,minimum height=1cm,font=\small] (pronoun) {ОТНОСИ-\\ТЕЛЬНОЕ \\ МЕСТО-\\ИМЕНИЕ};
\path (pronoun.east) ++(1,-1) node[draw,align=center,minimum height=1cm,anchor=west,font=\small] (super-fancy-noun-bot) {СВЕРХУКРА-\\ШЕННОЕ \\ СУЩЕСТВИ-\\ТЕЛЬНОЕ};
\path (super-fancy-noun-bot.east) ++(1,0) node[draw,align=center,minimum height=1cm,anchor=west] (verb-bot) {ГЛАГОЛ};
\path (pronoun.east) ++(1,1) node[draw,align=center,minimum height=1cm,anchor=west] (verb-top) {ГЛАГОЛ};
\path (verb-top.east) ++(1,0) node[draw,align=center,minimum height=1cm,anchor=west,font=\small] (super-fancy-noun-top) {СВЕРХУКРА-\\ШЕННОЕ \\ СУЩЕСТВИ-\\ТЕЛЬНОЕ};
\path (end.90) ++(-0.5,1) coordinate (x);

\draw[->-] (start) -- (fancy-noun);
\draw[->-] (fancy-noun) -- (end);
\draw[->-] (prep) -- (super-fancy-noun);
\draw[->-] (super-fancy-noun-bot) -- (verb-bot);
\draw[->-] (verb-top) -- (super-fancy-noun-top);
\draw (x) to[out=-60,in=120] (end.90);

\draw[->-=0.4] (fancy-noun.-100) to[out=-60,in=180] (prep.180);
\draw[->-=0.4] (fancy-noun.100) to[out=60,in=180] (pronoun.180);
\draw[->-=0.8] (super-fancy-noun.0)--++(0.5,0) to[out=0,in=-120] (end.-90);
\draw[->-=0.6] (pronoun.30) to[out=0,in=180,looseness=1.3] (verb-top);
\draw[->-=0.6] (pronoun.-30) to[out=0,in=180,looseness=1.3] (super-fancy-noun-bot);
\draw[->-=0.6] (super-fancy-noun-top.0)--++(0.5,0) to[out=0,in=120] (x);
\draw[->-=0.4] (verb-bot.0)--++(1,0) to[out=0,in=120] (x);

\end{tikzpicture}%
\end{document}
