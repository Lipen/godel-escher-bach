\subsubsection{Соната для Ахилла соло}

\emph{Звонит телефон~--- Ахилл берет трубку.}

\emph{Ахилл} : Алло, Ахилл слушает.

\emph{Ахилл} : А, здравствуйте, г-жа Черепаха. Как дела?

\emph{Ахилл} : Кривошея и чихиллит? Что такое чихи\ldots~--- а, теперь понимаю. Будьте здоровы!\ldots{} Что и говорить, неприятная комбинация. Как это вы ухитрились такое подцепить?

\emph{Ахилл} : И долго вы ее так продержали?

\emph{Ахилл} : Еще на самом сквозняке~--- не удивительно, что вам в шею надуло!

\emph{Ахилл} : Что же вас заставило так долго там проторчать?

\emph{Ахилл} : Многие из них удивительные? Какие, например?

\emph{Ахилл:} Фантасмагорические чудища? Что вы имеете в виду?

\emph{Ахилл} : И вам не страшно было в такой компании?

\emph{Ахилл} : Гитара? Вот странно~--- откуда взялась гитара среди этих диковинных созданий. Кстати, вы играете на гитаре?

\emph{Ахилл} : Ах, для меня это одно и то же.

\emph{Ахилл} : Вы правы удивительно, как это я сам до сих пор не заметил, в чем разница между гитарой и скрипкой. Кстати о скрипках: не хотите ли вы~~заглянуть ко мне и послушать сонату для скрипки соло вашего любимого композитора, И. С. Баха? Я только что купил отличную запись.~Поразительно, как это Баху удалось, используя одну-единственную скрипку,~создать такую интересную вещь.

\emph{Ахилл} : Головная боль тоже? Бедняжка\ldots{} Пожалуй, вам лучше лечь в постель и постараться заснуть.

\emph{Ахилл} : Понятно. Овец считать уже пробовали? Где-то у меня была целая~картотека подобных трюков~--- говорят, они здорово помогают от бессоницы.

\emph{Ахилл} : Ах, да. Я отлично понимаю, что вы имеете в виду~--- я это тоже пробовал. Может быть, если уж эта задачка так застряла у вас в голове, вы~поделитесь ею со мной, чтоб и я мог попробовать свои силы?

\emph{Ахилл} : Слово, внутри которого встречаются подряд буквы «Р», «Т», «О», «Т», «Е»\ldots{} Г-м-м\ldots{} Как насчет «ретотра»?

\emph{Ахилл} : Ах, какой стыд\ldots{} Конечно вы правы~--- я опять все перепутал. К тому же в слове «реторта» эти буквы все равно идут задом наперед.

\emph{Ахилл} : Уже несколько часов? Хорошенькую вы мне задали задачку\ldots{} Где вы откопали такую дьявольскую головоломку?

\emph{Ахилл} : Вы имеете в виду, что он только делал вид, что размышляет над~эзотерическими буддистскими проблемами, когда на самом деле он пытался придумать сложные словесные головоломки?

\emph{Ахилл} : Ага! Улитка знала, чем он занимается. Как же вам удалось с ней~переговорить?

\emph{Ахилл} : Вы знаете, я как-то слышал похожую головоломку. Хотите, я вам ее задам? Или это еще хуже вас отвлечет?

\emph{Ахилл} : Согласен~--- хуже уже вряд ли будет. Так вот: какое слово начинается с «КА» и кончается на «КА»?

\emph{Ахилл} : Очень остроумно~--- но это нечестно. Я совершенно не это имел в виду!

\emph{Ахилл} : Согласен, это слово выполняет условие; но все равно это какое-то~дегенеративное решение.

\emph{Ахилл} : Абсолютно верно! Как вам удалось так быстро найти ответ?

\emph{Ахилл} : Это~--- еще один пример того, какой полезной может оказаться картотека трюков от бессоницы. Прекрасно! Но я все еще блуждаю в потемках с вашей задачкой о «PTOTE».

\emph{Ахилл} : Поздравляю~--- теперь вам, может быть, удастся заснуть. Скажите же мне решение!

\emph{Ахилл} : Вообще-то я не люблю подсказок, но на этот раз ладно, валяйте.

\emph{Ахилл} : Не понимаю. Что вы имеете в виду под «рисунком» и «фоном»?

\emph{Ахилл} : Разумеется, я знаком с «Мозаикой II». Я знаю ВСЕ работы Эшера. В конце концов, это мой любимый художник! Кстати, репродукция «Мозаики II» висит прямо у меня перед носом.

\emph{Ахилл} : Всех черных зверей? Конечно, вижу!

\emph{Ахилл} : Верно: их «негативное пространство»~--- то, что остается свободным~--- определяет белых зверей.

\emph{Ахилл} : А, так вот что вы называете «рисунком» и «фоном»! Но какое отношение это имеет к головоломке о «Р-Т-О-Т-Е»?

\emph{Ахилл} : Это для меня слишком сложно\ldots{} Теперь и у меня начинает болеть голова; пойду, пожалуй, поищу мою спасительную картотеку, может быть она мне поможет забыться сном.

\emph{Ахилл} : Вы хотите зайти сейчас? Но я думал\ldots{}

\emph{Ахилл} : Ну что ж, хорошо. А я пока постараюсь решить эту задачку с помощью вашей подсказки о рисунке и фоне и моей головоломки.

\emph{Ахилл} : С удовольствием сыграю их для вас.

\emph{Ахилл} : Вы изобрели о них теорию?

\emph{Ахилл:} В сопровождении какого инструмента?

\emph{Ахилл} : В таком случае, как странно, что он не записал также и партию~клавесина, и не опубликовал их в таком виде.

\emph{Ахилл} : А, понимаю~--- нам предоставляется выбор: слушать ее с~аккомпанементом или без оного. Но откуда мы знаем, как он должен звучать?

\emph{Ахилл} : Да, вы правы~--- наверное, лучше всего оставить эту работу воображению слушателя. Согласен~--- может быть, у Баха в мыслях вообще не было никакого аккомпанемента. Действительно, эти сонаты и так звучат~замечательно.

\emph{Ахилл} : Точно. Ну, до скорого.

\emph{Ахилл:} Пока, г-жа Ч.

\emph{Рис. 14. М К. Эшер. «Мозаика II» (литография, 1957).}

