\subsubsection{Маленький гармонический лабиринт}

\emph{Черепаха и Ахилл проводят день в Кони Айленде, огромном парке аттракционов. Купив себе по палочке «сахарной ваты», они решают прокатиться на колесе обозрения.}

Черепаха: Это мой любимый аттракцион. Кажется, что едешь так далеко --- а на самом деле никуда не попадаешь!

Ахилл: Понятно, почему это вам так нравится. Вы уже пристегнулись?

Черепаха: Да, все ремни на месте. Поехали! Ур-ра!

Ахилл: Я вижу, вы сегодня предовольны.

Черепаха: И не без основания: моя тетушка-гадалка предсказала мне на сегодня необыкновенную удачу. Так что я вся трепещу в предвкушении.

Ахилл: Неужели вы верите в предсказания судьбы?

Черепаха: Вообще-то нет\ldots{} но говорят, что они действуют, даже когда в них не веришь.

Ахилл: Ну, в таком случае, вам действительно повезло.

Черепаха: Ах, какой вид! Пляж, толпа, океан, город\ldots{}

Ахилл: И правда, великолепно. Взгляните-ка на вертолет --- вон там. Кажется, он летит в нашем направлении. На самом деле, он уже почти над нами.

Черепаха: Странно, оттуда свисает какая-то веревка\ldots{} и она совсем близко к нам --- можно ухватиться\ldots{}

Ахилл: Смотрите-ка: на конце веревки огромный крюк и на нем --- записка.

\emph{(Он протягивает руку и срывает записку. Колесо начинает опускаться.)}

Черепаха: Ну как, что там написано? Можете разобрать?

Ахилл: Да\ldots{} Здесь написано: «Приветик, друзья. Будете снова наверху --- хватайтесь за крюк, и получите Сюрприз!»

Черепаха: Записка грубовата\ldots{} но кто знает, к чему это может привести. Может, это начинается обещанное везенье. Давайте попробуем!

Ахилл: Давайте!

\emph{(Когда колесо снова начинает подниматься, они расстегивают свои ремни и на самой высокой точке хватаются за гигантский крюк. Внезапно веревка взлетает вверх, унося их к зависшему над их головами вертолету. Большая сильная рука втаскивает их внутрь.)}

Голос: Добро пожаловать на борт, лопухи!

Ахилл: К-кто\ldots{} кто вы такой?

Голос: Позвольте представиться: Гексахлорофен Ж. Удача, Знаменитый Похититель Детишек и Пожиратель Черепах --- к вашим услугам!

Черепаха: Ой!

Ахилл (шепотом Черепахе): Вот так «Удача»! Не совсем то, на что мы надеялись\ldots{} (Удаче): Гм-м-м\ldots{} если я могу позволить себе смелость спросить куда вы нас везете?

Удача: Хо-хо! На мою небесную кухню с полным электрическим оборудованием, где я собираюсь приготовить вот этот лакомый кусочек (бросая плотоядный взгляд на Черепаху) --- райский супчик получится, пальчики оближешь! И не сомневайтесь --- я проделываю все это исключительно в усладу моему чревоугодию! Хо-хо-хо!

Ахилл: На это я могу сказать лишь то, что смех у вас довольно злодейский.

Удача (злодейски смеясь): Хо-хо-хо! За эти слова, мои дорогой друг, ты мне дорого заплатишь! Хо-хо!

Ахилл: Ах, господи! Интересно, что он имеет в виду?

Удача: Все очень просто: у меня для вас обоих уготовлена Ужасная Судьба! Погодите --- вы у меня попляшете! Хо-хо- хо! Хо-хо-хо!

Ахилл: Ой, мамочка!\ldots{}

Удача: Вот мы и приехали. Высаживайтесь, друзья, прямо в мою электрическую небесную кухню. (Они заходят внутрь.) Располагайтесь и чувствуйте себя как дома, пока я буду решать вашу судьбу. Вот моя спальня. Вот мой кабинет. Присаживайтесь и подождите меня --- я ненадолго, только ножи наточу. Можете пока попробовать мои вина. Мое последнее приобретение --- «Витаскин»; там что-то еще на этикетке понаписано, да только я языка не понимаю, так что я называю эту штуку просто: «Вытаскин». Вон та бутылочка, на лосьон смахивает\ldots{} Я его еще сам не пробовал. Ну, я пошел. Хо-хо-хо! Черепаший супчик! Черепаший супчик! Мое любимое блюдо! (Уходит.)

Ахилл: Вытаскин! Давайте напьемся с горя!

Черепаха: Ахилл! Вы же уже выпили две кружки пива в парке! Да и как вы можете думать об этом в такой момент, именно когда нам необходима ясная голова?

Ахилл: А мне до лампочки\ldots{} (Поет.) Шуме-ел камы-ы-ыш\ldots{} о, миль пардон, я не должен петь подобных песен в присутствии дамы, да еще в такую ужасную минуту.

Черепаха: Боюсь, что наша песенка так и так спета.

Ахилл: Это еще бабушка надвое сказала. Давайте пока от нечего делать посмотрим, что за книги у нашего хозяина на полках. Ну и коллекция, только для посвященных: «Садовые головы, с которыми я был знаком», «Шахматы и верчение зонтиков --- без труда», «Концерт для чечеточника и оркестра»\ldots{} Гм-м-м.

Черепаха: Что это за открытая книжица лежит там на столе, рядом с додекаэдром и альбомом для рисования?

Ахилл: Эта? Она называется: «Занимательные приключения Ахилла и Черепахи или Вокруг света от кочки до кочки.»

Черепаха: Довольно занимательное название.

Ахилл: Действительно --- и приключение, на котором книга открыта, выглядит занимательно. Оно называется «Джинн и Настойка».

Черепаха: Гм-м-м\ldots{} Интересно, почему. Может, попробуем почитать? Я буду читать за Черепаху, а вы --- за Ахилла.

Ахилл: Согласен. Терять нам все равно нечего\ldots{}

\emph{(Они начинают читать «Джинна и настойку».)}

\emph{(Ахилл пригласил Черепаху в гости, посмотреть коллекцию гравюр его любимого художника, Эшера.)}

\emph{Черепаха} : Чудесные гравюры, Ахилл.

\emph{Ахилл} : Я так и знал, что вам понравится. Какая ваша любимая гравюра?

\emph{Черепаха} : Одна из моих любимых --- «Выпуклое и вогнутое», где совмещаются два внутренне непротиворечивых мира. В результате получается составной, абсолютно невозможный мир. Противоречивые миры всегда забавно посетить, но жить там мне бы не хотелось.

\emph{Ахилл} : «Забавно посетить?» Что вы имеете в виду? Как можно посетить противоречивые миры, если их вообще НЕ СУЩЕСТВУЕТ?

\emph{Черепаха} : Прошу прощения --- но разве мы только что не согласились, что на этой картине Эшера изображен противоречивый мир?

\emph{Ахилл} : Да, но это же двухмерный мир, фикция, картинка. Этот мир посетить не удастся.

\emph{Черепаха} : У меня есть свои способы\ldots{}

\emph{Ахилл} : Как же вам удается затолкать себя в плоский мир картины?

\emph{Черепаха} : Для этого надо выпить стаканчик ПРОТАЛКИВАЮЩЕГО ЗЕЛЬЯ.

\emph{Ахилл} : Что это за штука такая --- проталкивающее зелье?

\emph{Черепаха} : Это жидкость, обычно содержащаяся в маленьких керамических пузырьках; когда вы, глядя на картину, выпиваете немного, жидкость эта проталкивает вас прямо в мир картины. Люди, которые ничего не знают о свойствах проталкивающего зелья, часто бывают поражены тем, в какие ситуации они попадают.

\emph{Ахилл} : А как насчет противоядия? Когда человек таким образом оказывается протолкнутым в картину, он что, так и остается там на всю жизнь?

\emph{Черепаха} : Иногда это не такое уж большое несчастье\ldots{} Но, разумеется, имеется другое зелье --- на самом деле, это скорее что-то вроде бальзама\ldots{} или эликсира\ldots{}

Черепаха: Она, кажется, имеет в виду «настойку».

\emph{Ахилл} : Настойка?

\emph{Черепаха} : Точно, именно это я и имела в виду! ВЫТАЛКИВАЮЩАЯ НАСТОЙКА, так она и называется. Если вы держите ее в правой руке, когда глотаете проталкивающее зелье, то она тоже оказывается протолкнутой в картину вместе с вами. Как только вы возжаждете быть вытолкнутым обратно в реальный мир, отхлебните немного выталкивающей настойки и --- але-оп! --- вы в реальном мире, точно на том же месте, где вы были, когда отведали проталкивающего зелья.

\emph{Ахилл} : Все это звучит захватывающе интересно. А что получится, если принять выталкивающую настойку, не протолкнувшись предварительно в картину?

\emph{Черепаха} : Я точно не знаю, Ахилл, но я бы не стала играть с этими странными жидкостями. Когда-то у меня был друг Фома, который мне не поверил и решил сделать именно это --- и с тех пор никто о нем ничего не слыхал.

\emph{Ахилл} : Жаль. А можно ли взять с собой бутылочку проталкивающего зелья?

\emph{Черепаха} : О, конечно. Надо зажать ее в левой руке и она тоже оказывается протолкнутой в картину вместе с вами.

Ахилл: А если внутри этой картины окажется еще одна, и вы снова примете глоточек проталкивающего зелья?

Черепаха: Случится именно то, чего вы ожидаете: вы очутитесь внутри картины-в-картине.

Ахилл: И, наверное, тогда придется выталкиваться дважды, чтобы вытащить себя из вписанных друг в друга картин и вновь вернуться в реальную жизнь.

Черепаха: Совершенно верно. На каждое проталкивание приходится одно выталкивание, так как первое вводит вас в картину, а второе это действие отменяет.

Ахилл: Знаете, все это звучит подозрительно. Вы уверены, что вы говорите это не только с целью испытать пределы моей доверчивости?

Черепаха: Клянусь! Поглядите: вот тут, в кармане, у меня два пузырька. (Засовывает руку в жилетный карман и вытаскивает два довольно больших пузырька без этикетки; слышно, как в них булькает жидкость, в одном красная, в другом --- голубая.) Ежели желаете, можем попробовать!

Ахилл: Э-э-э\ldots{} ну ладно\ldots{} может быть\ldots{}

Черепаха: Ну и славно! Я так и думала, что вам захочется попробовать. Хотите протолкнуться в мир Эшеровского «Выпуклого и вогнутого?»

Ахилл: Ну, как вам сказать\ldots{}

Черепаха: Значит, решено. Не забыть захватить с собой бутылочку настойки, чтобы мы смогли вытолкнуться обратно. Возьмете на себя эту ответственность, Ахилл?

Ахилл: Знаете, я немного нервничаю, и, если вы не возражаете, я предпочел бы, чтобы вы, с вашим опытом, управляли бы этой операцией.

Черепаха: Отлично. Итак\ldots{}

\emph{(С этими словами Черепаха наливает две маленькие порции проталкивающего зелья, протягивает Ахиллу его стакан и зажимает в правой лапе пузырек с настойкой Оба подносят стаканы к губам.)}

Черепаха: Пей до дна!

\emph{(Они делают по глотку.)}

\emph{Ахилл} : Что за странный привкус!

\emph{Черепаха} : К нему постепенно привыкаешь.

\emph{Ахилл} : А у настойки такой же странный вкус?

\emph{Черепаха} : Что вы, никакого сравнения! После первого же глотка вы чувствуйте такое удовлетворение, будто вы всю жизнь только о ней и мечтали.

\emph{Ахилл} : Прямо не терпится попробовать!

\emph{Черепаха} : Ну, Ахилл, где мы находимся?

\emph{Ахилл (оглядываясь)} : Мы в маленькой гондоле, скользим вниз по каналу! Я хочу сойти на берег. Синьор гондольер, остановите здесь, пожалуйста!

\emph{Рис. 23. М. К. Эшер «Выпуклое и вогнутое» (литография, 1955)}

~\emph{(Гондольер не обращает на эту просьбу ни малейшего внимания)}

~\emph{\textbf{Черепаха}} : Он не понимает по-русски. Придется нам выпрыгивать на берег, пока гондола не вошла в этот ужасный «Туннель любви», прямо перед нами.

~\emph{(Ахилл, слегка побледнев, выпрыгивает из гондолы с быстротой молнии и вытаскивает свою более медлительную спутницу.)}

\emph{\textbf{Ахилл}} : Что-то мне в этом названии определенно не по вкусу. Я очень рад что нам удалось вовремя вылезти. Послушайте, а откуда вы так хорошо знаете эти места? Вы здесь уже бывали раньше?

\emph{\textbf{Черепаха}} : Много раз, но я всегда попадала сюда из других картин Эшера. Знаете ли, позади рам они все соединены. Войдя в одну из картин, можно оттуда попасть в любую другую.

~\emph{\textbf{Ахилл}} : Удивительно! Если бы я не видел всего этого своими глазами, я бы ни за что в это не поверил. (Они выходят наружу сквозь небольшую арку.) Ой, что это там за смешная парочка ящериц?

\emph{\textbf{Черепаха}} : Смешные? Никакие они не смешные --- я вся дрожу при одной мысли о них! Это же злобные стражи волшебной медной лампы. Вон она, висит на потолке. Одно прикосновение языка, и любой смертный превращается в огурчик для закуски!

\emph{\textbf{Ахилл}} : Соленый или маринованный?

\emph{\textbf{Черепаха}} : Маринованный.

\emph{\textbf{Ахилл}} : Какая горькая судьба! Все-таки, если лампа действительно волшебная, я, пожалуй, рискну\ldots{}

\emph{\textbf{Черепаха}} : Это чистое безумие, мой друг. Я бы на вашем месте не стала этого делать.

\emph{\textbf{Ахилл}} : Всего один разочек\ldots{}

\emph{(Крадется к лампе, стараясь не разбудить спящую поблизости ящерицу. Внезапно нога его попадает в странную выемку в форме ракушки --- Ахилл скользит и взлетает в воздух. Судорожно пытаясь за что-то уцепиться, он нащупывает лампу и хватается за нее одной рукой. Лампа раскачивается. Ахилл беспомощно болтается в воздухе, а взбешенные ящерицы шипят и высовывают языки, пытаясь до него достать.)}

\emph{\textbf{Ахилл}} : На по-о-о-мощь!

\emph{(Его крик привлекает внимание стоящей поблизости женщины --- та сбегает с лестницы и будит спящего внизу мальчишку. Оценив ситуацию, он ободряюще улыбается Ахиллу и жестами показывает ему, что все будет в порядке. На странном гортанном наречии мальчишка кричит что-то двум трубачам, глядящим из окон. Они тут же начинают играть. Чудные мелодии сплетаются друг с другом, в необычном ритмическом узоре. Сонный паренек кивает в сторону ящериц, и Ахилл видит, что музыка действует усыпляюще и на них. Вскоре они вновь замирают. Тогда услужливый мальчишка зовет двух товарищей, взбирающихся по лестницам. Они составляют из лестниц что-то вроде моста. Повинуясь их настойчивым приглашающим жестам, Ахилл хватается за перекладины --- но прежде он осторожно разгибает верхнее звено цепи, на которой висит лампа, и снимает ее. Потом он взбирается на лестничный мост и мальчики вытаскивают его на безопасное место. Благодарный воин поочередно обнимает каждого из них.)}

\emph{\textbf{Ахилл}} : Г-жа Черепаха, как мне их отблагодарить?

\emph{\textbf{Черепаха}} : Я слыхала, что эти смельчаки неравнодушны к кофе --- а там внизу, в городе, есть местечко, где подают несравненный кофе-экспресс. Пригласите-ка их на чашечку!

\emph{\textbf{Ахилл}} : Это то что надо!

\emph{(С помощью комической серии жестов, улыбок и слов, Ахиллу удается растолоковать паренькам, что он их приглашает. Компания спускается по крутой лестнице в город. Они подходят к небольшому уютному кафе, усаживаются за один из столиков на улице, и заказывают пять чашечек экспресса. Пока друзья попивают кофе, Ахилл внезапно вспоминает про свою волшебную лампу.)}

\emph{\textbf{Ахилл}} : Чуть не забыл, г-жа Черепаха, лампа-то здесь! А что же в ней такого магического?

\emph{\textbf{Черепаха}} : Да как обычно --- джинн.

\emph{\textbf{Ахилл}} : Что? Вы имеете в виду, что стоит ее потереть, появится джинн и исполнит все ваши желания?

\emph{\textbf{Черепаха}} : Именно. А вы чего ожидали? Манны небесной?

\emph{\textbf{Ахилл}} : Да это же просто фантастика! Любое желание, а? Я всегда мечтал о чем-нибудь подобном\ldots{}

\emph{(Ахилл начинает тихонько тереть большую букву Л, выгравированную на медном боку лампы. Внезапно из лампы вырывается клуб дыма, в котором пятеро друзей различают очертания огромной призрачной фигуры, похожей на башню.)}

\emph{\textbf{Ахилл}} : Джинн!

\emph{\textbf{Черепаха}} : Дух!

\emph{\textbf{Фигура}} : Можно звать просто Гением\ldots{} Приветствую вас, о высокочтимые друзья, и благодарю за спасение моей Лампы от злобной Ящеричной Парочки. (С этими словами Гений подбирает Лампу и сует ее в карман, спрятанный в складках его длинного призрачного одеяния, струящегося из Лампы.) В благодарность за ваш героический поступок, я хотел бы предложить вам, от лица моей Лампы, осуществить три ваших желания.

\emph{\textbf{Ахилл}} : Потрясающе! Как вы думаете, г-жа Ч.?

\emph{\textbf{Черепаха}} : Безусловно. Что ж, друг мой, говорите ваше первое желание.

\emph{\textbf{Ахилл}} : Ух ты!.. Чего же мне пожелать? А, знаю: это пришло мне в голову еще когда я в первый раз читал «Тысячу и одну ночь» --- эти немудреные сказочки, вставлены одна в другую наподобие матрешки. Я хочу иметь не три, а СТО желаний? Здорово, правда, г-жа Ч.? Никогда не~~понимал, почему эти балбесы в сказках не догадываются попросить то же самое?

\emph{\textbf{Черепаха}} : Может быть, сейчас вы поймете.

\emph{\textbf{Гений}} : Мне очень жаль, Ахилл, но я не исполняю мета-желаний.

\emph{\textbf{Ахилл}} : Мне бы хотелось знать, что такое мета-желание\ldots{}

\emph{\textbf{Гений}} : Но это уже мета-мета-желание, Ахилл, а их я тоже не могу исполнить.

\emph{\textbf{Ахилл}} : Что-о? Ничего не понимаю\ldots{}

\emph{\textbf{Черепаха}} : Почему бы вам не выразить вашу просьбу как-нибудь по-другому?

\emph{\textbf{Ахилл}} : Что вы имеете в виду? Почему по-другому?

\emph{\textbf{Черепаха}} : Дело в том, что вы начинаете со слов «Мне бы хотелось\ldots» Но, поскольку вы хотите получить информацию, почему бы вам просто не задать вопрос?

\emph{\textbf{Ахилл}} : Ну хорошо, хотя я не совсем понимаю\ldots{} Скажите, пожалуйста, мистер Гений, что такое мета-желание?

\emph{\textbf{Гений}} : Это всего-навсего желание о желаниях. У меня нет права исполнять мета-желания. В моей власти только самые обыкновенные желания; ящик пива, скатерть-самобранка, готовая на все красотка, миллион долларов\ldots{} Понимаете, что-нибудь простенькое. Но мета-желание --- не могу. БОГ не велит.

\emph{\textbf{Ахилл}} : БОГ? Кто такой БОГ? И почему он не велит вам исполнять мета-желания? Это кажется совсем легко по сравнению с желаниями, о которых вы только что упомянули.

\emph{\textbf{Гений}} : Как вам сказать\ldots{} На самом деле, это довольно сложно. Почему бы вам просто не загадать три желания? Или, для начала, хотя бы одно? Я, знаете ли, не могу сидеть тут у вас до скончания веков.

\emph{\textbf{Ахилл}} : Ах, какое разочарование\ldots{} А я-то так надеялся получить мои сто желаний.

\emph{\textbf{Гений}} : Боже мой, как неприятно разочаровывать людей. К тому же, мета-желания --- мой любимый вид желаний. Пожалуй, я могу постараться вам помочь. Это отнимет только одну минуточку\ldots{}

\emph{(Гений вынимает из легких складок своей, одежды почти такую же Лампу, какую он недавно положил в карман . На этот раз она не медная, а серебряная. На месте буквы «Л» на ней, помельче, выгравировано «МЛ.»)}

\emph{\textbf{Ахилл}} : А это что такое?

\emph{\textbf{Гений}} : Это моя Мета-Лампа.

\emph{(Он начинает тереть Мета-Лампу, из которой вырывается огромный клуб дыма. В дымных водоворотах вырисовывается гигантская призрачная~~фигура, нависшая над ними подобно башне. На этот раз джинн оказывается женщиной.)}

\textbf{Мета-Гений} : Я --- Мета-Гений. Вы звали меня, о, высокочтимый Гений? Каково ваше желание?

\emph{\textbf{Гений}} : Я хочу попросить вас, о Гений, и также БОГа, даровать мне исполнение специального желания: отмены ограничений на типы желаний на время одного Нетипового Желания. Можете ли вы это сделать?

\textbf{Мета-Гений} : Придется, разумеется, направить вашу просьбу по соответствующим каналам\ldots{} Это отнимет только полминутки.

\emph{(Вдвое быстрее чем Гений, она вынимает из легких складок своего платья почти такую же Лампу, какую тот недавно положил в карман. На этот раз она не серебряная, а золотая На месте букв «МЛ» на ней, помельче, выгравировано «ММЛ.»)}

\textbf{Ахилл (его голос теперь звучит на октаву выше)} : Что это такое?

\textbf{Мета-Гений} : Это моя Мета-Мета-Лампа\ldots{}

\emph{(Она начинает тереть Мета-Мета-Лампу и из нее вырывается огромный клуб дыма, в котором они различают смутные очертания фигуры, нависшей над ними, подобно башне.)}

\textbf{Мета-Мета-Гений} : Я Мета-Мета-Гений. Вы звали меня, о Мета-Гений? Чего вы желаете?

\textbf{Мета-Гений} : Я хочу попросить вас, о Гений, и также БОГа, даровать мне исполнение специального желания, отмены ограничений на типы желаний, на время одного Нетипового Желания. Можете ли вы это сделать?

\textbf{Мета-Мета-Гений} : Придется, разумеется, направить вашу просьбу по соответствующим каналам\ldots{} Это отнимет только четверть минутки.

\emph{(И, вдвое быстрее чем Мета-Гений, он достает из складок своего одеяния предмет, напоминающий золотую Мета-Мета-Лампу, с той разницей, что он сделан из\ldots{}}

~\emph{(\ldots втягивается обратно в Мета-Мета-Мета-Лампу, которую Мета-Мета-Гений прячет обратно в складки своего одеяния, вдвое медленнее, чем это делал Мета-Мета-Мета-Гений.)}

Ваше желание исполнено, о Мета-Гений.

\textbf{Мета-Гений} : Благодарю вас, о Гений, и БОГ. (И Мета-Мета-Гений, подобно всем высшим Гениям, исчезает в Мета-Мета-Лампе, которую Мета-Гений затем прячет в складках своего платья, вдвое медленнее, чем Мета-Мета-Гений.) Ваше желание исполнено, Гений.

\emph{\textbf{Гений}} : Благодарю вас, о Гений и БОГ. (И Мета-Гений, подобно всем высшим Гениям, исчезает в Мета-Лампе, которую Гений затем прячет в складках его одеяния, вдвое медленнее, чем Мета-Гений.)Ваше желание исполнено, Ахилл.

\emph{(Ровно минута прошла с тех пор, как он сказал: «Это отнимет только одну минуту».)}

\emph{\textbf{Ахилл}} : Благодарю вас, О Гений и БОГ.

\emph{\textbf{Гений}} : Рад вам сказать, Ахилл, что вам даровано право ровно на одно Нетиповое Желание. Это может быть просто желание, или мета-желание, или мета-мета-желание --- столько «мета», сколько вашей душеньке угодно --- даже бесконечно много, ежели желаете.

\emph{\textbf{Ахилл}} : Я вам бесконечно благодарен, Гений. Но вы задели мое любопытство. Прежде чем я скажу свое желание, не могли бы вы мне ответить, кто такой --- или что такое --- БОГ?

\emph{\textbf{Гений}} : Нет ничего проще. «БОГ» --- это сокращение. Оно расшифровывается так: «БОГ, Одолевающий Гения.»~~Слово «Гений» обозначает Гениев, Мета-Гениев, Мета-Мета-Гениев и т. д. Это Нетиповое слово.

\emph{\textbf{Ахилл}} : Но\ldots{} Но как БОГ может быть словом в своем собственном сокращении? Это совершенная бессмыслица!

\emph{\textbf{Гений}} : Разве вы ничего не слыхали о рекурсивных сокращениях? Я думал, это общеизвестно. Видите ли, БОГ означает «БОГ, Одолевающий Гения», что, в свою очередь, может быть расширено «БОГ, Одолевающий Гения, Одолевающий Гения», что также может быть расширено до «БОГ, Одолевающий Гения, Одолевающий Гения, Одолевающий Гения», что, в свою очередь, может быть расширено\ldots{} и так расширять его можно сколько угодно.

\emph{\textbf{Ахилл}} : Но я так никогда не кончу!

\emph{\textbf{Гений}} : Разумеется, нет. БОГа невозможно познать до конца.

\emph{\textbf{Ахилл}} : Гм-м-м\ldots{} Изрядная путаница. Что вы имели в виду, когда попросили Мета-Гения, а также БОГа, даровать исполнение специального желания?

\emph{\textbf{Гений}} : Я обращался не только к Мета-Гению, но и ко всем Гениям выше нее. С помощью рекурсивного сокращения это делается просто. Услыхав мою просьбу, Мета-Гений передала ее своему БОГу. Так просьба достигла Мета-Мета-Гения, который, в свою очередь, направил ее Мета-Мета-Мета-Гению\ldots{} Поднимаясь таким образом по инстанциям, просьба в конце концов достигает БОГа.

\emph{\textbf{Ахилл}} : Понятно. Значит, БОГ сидит наверху лестницы Гениев?

\emph{\textbf{Гений}} : Да нет же! Наверху ничего нет, так как никакого «верха» не существует Именно поэтому БОГ --- рекурсивное сокращение. БОГ --- не какой-то последний Супер-Гений; это «башня» всех Гениев, находящихся над данным Гением.

\emph{\textbf{Черепаха}} : Мне кажется, что в таком случае каждый Гений имеет свое представление о том, что такое БОГ, так как для каждого Гения БОГ --- это множество высших Гениев, и нет двух таких Гениев, у которых это множество было бы одинаковым.

\emph{\textbf{Гений}} : Вы совершенно правы --- и поскольку я самый «низкий» Гений из всех, мое представление о БОГе самое возвышенное. Бедные высшие Гении --- они воображают, что находятся ближе к БОГу. Какое кощунство!

\emph{\textbf{Ахилл}} : Ух ты! Слишком все это сложно. Поистине, чтобы изобрести БОГа, нужны Гении\ldots{}

\emph{\textbf{Черепаха}} : Вы действительно верите всем этом сказкам о БОГе, Ахилл?

\emph{\textbf{Ахилл}} : Ну конечно, верю. А вы что же, атеистка г-жа Черепаха? Или агностик?

\emph{\textbf{Черепаха}} : Не думаю. Может быть, я --- мета-агностик.

\emph{\textbf{Ахилл}} : Что-о-о? Ничего не понимаю.

\emph{\textbf{Черепаха}} : Понимаете, если бы я была мета-агностиком, я~бы сомневалась в том, агностик ли я --- но я не уверена, что я в этом сомневаюсь. Значит, я, наверное, мета-мета-агностик\ldots{} Ну, ладно. Скажите мне, Гений, а случается ли какому-нибудь Гению ошибиться и перепутать путешествующее вверх или вниз по цепи послание?

\emph{\textbf{Гений}} : Такое иногда случается; это самая распространенная причина того, что Нетиповые Желания не разрешаются. Видите ли, вероятность того, что путаница произойдет на каком-то ОПРЕДЕЛЕННОМ этапе, ничтожно мала --- но когда у вас имеется цепь из бесконечного числа этапов, становится практически неизбежным, что ГДЕ-НИБУДЬ выйдет ошибка. На самом деле, как это ни странно, ошибок бывает бесконечное множество, хотя они и встречаются весьма редко.

\emph{\textbf{Ахилл}} : Тогда это просто чудо, когда какое-нибудь Нетиповое Желание вообще бывает даровано.

\emph{\textbf{Гений}} : Не совсем так. Большинство ошибок остается без последствий, а некоторые ошибки взаимоуничтожаются. Но иногда, хотя и довольно редко, причиной неисполнения Нетипового Желания может быть ошибка какого-то одного несчастного Гения. Когда такое происходит, виновник прогоняется сквозь бесконечный строй, и БОГ наказывает его шлепками. Это большое развлечение для шлепающих и к тому же совсем не больно для виновника. Вас бы позабавило это зрелище.

\emph{\textbf{Ахилл}} : Было бы интересно посмотреть! Но это бывает только в том случае, когда не исполняется Нетиповое Желание?

\emph{\textbf{Гений}} : Верно.

\emph{\textbf{Ахилл}} : Гм-м-м\ldots{} Кажется, я знаю, чего мне пожелать.

\emph{\textbf{Черепаха}} : Да? Чего же?

\emph{\textbf{Ахилл}} : Я бы хотел, чтобы мое желание не исполнилось!

\emph{(В этот момент происходит такое странное событие --- да можно ли это вообще назвать «событием»? --- что его невозможно описать; а значит, мы и пытаться не будем.)}

Ахилл: Интересно, что означает этот загадочный комментарий?

Черепаха: Он относится к Нетиповому Желанию, исполнения которого попросил Ахилл.

Ахилл: Но он еще ничего не пожелал!

Черепаха: Напротив; он сказал: «Я хотел бы, чтобы мое желание не исполнилось,» и Гений принял ЭТИ СЛОВА за желание.

\emph{(В этот момент в коридоре раздаются шаги; они медленно приближаются.)}

Ахилл: Ой! Какой кошмар!

\emph{(Шаги останавливаются и затем начинают удаляться.)}

Черепаха: Уф-ф!\ldots{}

Ахилл: История продолжается, или это уже конец? Переверните-ка страницу и давайте проверим.

\emph{(Черепаха переворачивает страницу «Джинна и настойки», и они обнаруживают, что история продолжается.)}

\emph{\textbf{Ахилл}} : Эй! Что стряслось? Где мой Гений? Моя лампа? Моя чашка кофе-экспресса? Что случилось с нашими юными друзьями из Выпуклого и Вогнутого Миров? И что здесь делают все эти ящерицы?

\emph{\textbf{Черепаха}} : Боюсь, что наш контекст был восстановлен неправильно.

\emph{\textbf{Ахилл}} : Интересно, что означает этот загадочный комментарий?

\emph{\textbf{Черепаха}} : Я имею в виду Нетиповое Желание, исполнения которого вы попросили.

\emph{\textbf{Ахилл}} : Но я еще ничего не пожелал!

\emph{\textbf{Черепаха}} : Напротив --- вы сказали: «Я хотел бы, чтобы мое желание не исполнилось», и Гений принял ЭТИ СЛОВА за желание.

\emph{\textbf{Ахилл}} : Ой! Какой кошмар!

\emph{\textbf{Черепаха}} : Это называется ПАРАДОКС. Чтобы исполнить это Нетиповое Желание, надо отказать в его исполнении. В то же время отказать в его исполнении значило бы исполнить его!

\emph{\textbf{Ахилл}} : Так что же произошло? Земля остановилась? Пространство закуклилось?

\emph{\textbf{Черепаха}} : Нет --- просто система отказала.

\emph{\textbf{Ахилл}} : Что это значит?

\emph{\textbf{Черепаха}} : Это значит, что мы оба мгновенно очутились в Лимбедламии.

\textbf{\emph{Ахилл}} : Где?

\emph{\textbf{Черепаха}} : Лимбедламия --- страна прошедшей икоты и перегоревших лампочек. Это что-то вроде зала ожидания, где дремлют программы в ожидании компьютеров. Нельзя сказать, как долго мы пробыли в Лимбедламии --- может быть, несколько минут, часов или дней, а может быть, и несколько лет.

\emph{\textbf{Ахилл}} : Я не знаю, при чем здесь программы или компьютеры. Я знаю только то, что не успел загадать желания! Верните моего Гения обратно!

\emph{\textbf{Черепаха}} : Мне очень жаль, Ахилл, но вы упустили свой шанс. Из-за вас отказала Система. Благодарите Бога, что мы вообще куда-то попали. Все могло быть гораздо хуже. Не имею ни малейшего понятия, где мы очутились\ldots{}

\emph{\textbf{Ахилл}} : Я знаю, это другая картина Эшера. Она называется «Рептилии».

\emph{\textbf{Черепаха}} : Ага! Система попыталась запомнить как можно больше нашего контекста перед тем, как отказать; ей~~удалось сохранить в памяти то, что мы находились в картине Эшера с ящерицами. Весьма похвально!

\emph{Рис. 24. М. К. Эшер. «Рептилии» (литография, 1943).}

\emph{\textbf{Ахилл}} : И взгляните не наш ли это флакончик с Выталкивающей настойкой там на столе, рядом с ящеричным хороводом?

\emph{\textbf{Черепаха}} : Безусловно, это он, Ахилл. Должна сказать, что нам действительно везет. Система обошлась с нами по-божески, вернув нам эту драгоценную жидкость!

\emph{\textbf{Ахилл}} : Это верно. Теперь мы можем вытолкнуться из эшеровского мира и вернуться ко мне домой.

\emph{\textbf{Черепаха}} : Интересно, что это за книги там, рядом с настойкой? (Она берет книгу поменьше, открытую в середине.) Эта книжица выглядит довольно занимательно.

\emph{\textbf{Ахилл}} : Правда? Как она называется?

\emph{\textbf{Черепаха}} : «Занимательные приключения Черепахи и Ахилла или Вокруг света от точки до точки.» Интересно было было бы почитать немного.

\emph{\textbf{Ахилл}} : Вы можете читать, если хотите, а я не собираюсь рисковать, какая-нибудь ящерица может запросто толкнуть флакон и разлить настойку. Я выпью свою порцию немедленно! (Он бросается к столу и протягивает руку к пузырьку с настойкой; при этом он случайно толкает его. Пузырек падает со стола и катится.) Ой! Г-жа Ч, смотрите! Я нечаянно столкнул настойку на пол и она покатилась\ldots~ к лестнице! Быстрее, а то свалится вниз!

\emph{(Но Черепаха погружена в свою книгу.)}

\emph{\textbf{Черепаха (бормочет)}} : А? Эта история выглядит захватывающе.

\emph{\textbf{Ахилл}} : Г-жа Ч, скорей, на помощь! Помогите поймать пузырек!

\emph{\textbf{Черепаха}} : Что за шум?

\emph{\textbf{Ахилл}} : Пузырек с настойкой, я столкнул его со стола, и сейчас он катится, и\ldots{} (В этот момент пузырек достигает первой ступеньки и падает вниз ) Ох! Что теперь делать? Г-жа Черепаха, вас это не волнует? Мы теряем настойку! Она только что свалилась с лестницы. Единственная наша надежда --- перейти на другой этаж!

\emph{\textbf{Черепаха}} : Перейти на другой рассказ? С превеликим удовольствием! Желаете ко мне присоединиться?

\emph{(Она начинает читать вслух, Ахилл застывает в нерешительности, не зная, что предпринять. Наконец он решает остаться и начинает читать за Черепаху.)}

\textbf{Ахилл} : Как здесь темно, г-жа Ч Я ничего не вижу. Ой! Я натолкнулся на стену. Осторожнее!

\textbf{Черепаха} : У меня есть пара тросточек Вот, держи~те одну. Вы можете прощупывать дорогу, чтобы ни с чем не сталкиваться.

\textbf{Ахилл} : Отличная идея. (Он берет трость.) Вам не кажется, что дорога слегка изгибается влево?

\textbf{Черепаха} : Да, пожалуй.

\textbf{Ахилл} : Интересно, где мы находимся. И увидим ли мы когда-нибудь дневной свет опять. Как жаль, что я вас послушался и проглотил эту штуковину «Выпей меня».

\textbf{Черепаха} : Уверяю вас, она совершенно безвредна. Я делала это много раз и никогда еще об этом не пожалела. Лучше расслабьтесь и постарайтесь получить удовольствие от того, что вы так чудесно уменьшились.

\textbf{Ахилл} : Уменьшился? Что вы со мной сделали, г-жа Черепаха?

\textbf{Черепаха} : Пожалуйста, не обвиняйте меня. Вы проделали все по вашему собственному желанию.

\textbf{Ахилл} : Так вы меня уменьшили? А вдруг лабиринт, в котором мы находимся, такой крохотный, что кто-нибудь может на него наступить?

\textbf{Черепаха} : Лабиринт? Лабиринт? Может ли это быть? Неужели мы попали в знаменитый лабиринт ужасного Мажотавра?

\textbf{Ахилл} : Ой, мамочка! Что это такое?

\textbf{Черепаха} : Говорят --- хотя я лично в это никогда не верила --- что злобный Мажотавр создал миниатюрный лабиринт и сидит в углублении в центре, поджидая невинных жертв, затерявшихся в чудовищно запутанных переходах. Когда они, окончательно заблудившись, забредают в центр, он начинает над ними смеяться, да так громко, что засмеивает их до смерти!

\textbf{Ахилл} : О боже, не может быть!

\textbf{Черепаха} : Это только миф. Смелее, Ахилл!

\emph{(И храбрая парочка осторожно двигается вперед.)}

\textbf{Ахилл} : Потрогайте эти стены. Они напоминают сморщенные жестяные листы --- только все морщины разного размера.

\emph{(Чтобы подчеркнуть свои слова, он прикладывает конец трости к стене и идет вперед. Трость подпрыгивает на неровностях стены --- длинный изогнутый коридор, в котором они находятся, наполняется странными звуками.)}

\textbf{Черепаха (встревоженно)} : Что это такое?

\textbf{Ахилл} : Это я веду тросточкой по стене.

\textbf{Черепаха} : Ох --- я было подумала, что это рев кровожадного Мажотавра.

\textbf{Ахилл} : Я думал, вы сказали, что это все выдумки.

\textbf{Черепаха} : Конечно. Бояться совершенно нечего.

\emph{(Ахилл снова прикладывает трость к стене и идет вперед. При этом слышна музыка; звуки исходят из того места, где трость прикасается к стене.)}

\textbf{Черепаха} : Ох, Ахилл, у меня дурное предчувствие --- мне кажется, что этот Лабиринт не такой уж и миф.

\textbf{Ахилл} : Погодите-ка, что это заставило вас так внезапно передумать?

\textbf{Черепаха} : Слышите эту музыку? (Чтобы лучше слышать, Ахилл опускает трость, и мелодия прекращается.) Эй! Поставьте трость обратно! Я хочу послушать конец этой пьесы!

\emph{Рис. 25. Критский лабиринт (Итальянская гравюра; школа Финигерры) Из книги У. Г. Маттьюса «Лабиринты: их история и развитие» (W.H. Mattews, Mazes and Labyrinths. Their History and Development.)}

(Ахилл, сбитый с толку, повинуется и музыка возобновляется.) Благодарю. Теперь я догадалась, где мы находимся.

\textbf{Ахилл} : Правда? Где же?

\textbf{Черепаха} : Мы идем по звуковой дорожке пластинки, лежащей в конверте. Ваша трость, скребущая по морщинам на стене, действует как иголка, бегущая по звуковой дорожке, позволяя нам слушать музыку.

\textbf{Ахилл} : Ох, нет, нет\ldots{}

\textbf{Черепаха} : Что такое? Разве вы не радуетесь? Когда еще вы находились в таком интимном контакте с музыкой?

\textbf{Ахилл} : Как же я смогу выигрывать соревнования по бегу против людей в натуральную величину, если я теперь меньше блохи, г-жа Черепаха?

\textbf{Черепаха} : Ах, так вот что вас волнует? Право, Ахилл, стоит ли из-за этого беспокоиться\ldots{}

\textbf{Ахилл} : Вы говорите так, что у меня создается впечатление, что вы вообще никогда не волнуетесь.

\textbf{Черепаха} : Не знаю, не знаю\ldots{} Я уверена только в~~одном: о чем я не жалею, так это о том, что я уменьшилась. В особенности тогда, когда нам грозит страшная опасность от чудовищного Мажотавра.

\textbf{Ахилл} : О ужас!.. Вы хотите сказать, что\ldots{}

\textbf{Черепаха} : Боюсь, что да, Ахилл. Музыка выдала его с головой.

\textbf{Ахилл} : Каким же это образом?

\textbf{Черепаха} : Очень просто. Когда я услышала мелодию В-А-С-H в верхнем голосе, меня осенило: на звуковых дорожках, по которым мы идем, записано не что иное, как «Маленький гармонический лабиринт», одна из наименее известных органных пьес Баха. Она названа так из-за модуляций, таких частых, что от них начинает кружиться голова.

\textbf{Ахилл} : Ч-что --- что это такое и с чем это едят?

\textbf{Черепаха} : Как вы знаете, большинство музыкальных произведений написано в какой-нибудь тональности --- например, «до мажор», как эта пьеса.

\textbf{Ахилл} : Я уже слышал это название раньше. Не правда ли, это значит, что «до» --- та нота, на которой произведение должно заканчиваться?

\textbf{Черепаха} : Да, «до» --- это что-то вроде ключа от дома, куда вы хотите попасть. Ключ бывает и в музыке.

\textbf{Ахилл} : Значит, сначала мы удаляемся от этого «дома», чтобы потом туда возвратиться?

\textbf{Черепаха} : Правильно. В музыкальных произведениях часто используются мелодии, уводящие в сторону от ключевой тональности. Мало-помалу нарастает напряжение, и слушатель начинает все сильнее скучать по «дому» ---~ему хочется вновь услышать ключевую тональность.

\textbf{Ахилл} : Таким образом, в конце пьесы я всегда буду чувствовать такое удовлетворение, как будто я всю жизнь желал услышать именно эти звуки?

\textbf{Черепаха} : Точно. Композитор использует свои знания о гармонической прогрессии, чтобы таким образом управлять нашими чувствами и пробудить в нас желание услышать ключевую тональность\ldots{}

\textbf{Ахилл} : Понятно, но, кажется, вы собирались рассказать мне о модуляциях\ldots{}

\textbf{Черепаха} : Ах, да. Один из важных приемов, которые композитор может использовать где-то в середине пьесы, называется модуляцией; это означает, что он устанавливает временную~~«цель», отличную от конечного разрешения в ключевую тональность.

\textbf{Ахилл} : А-а-а\ldots{} кажется, я понимаю. Вы имеете в виду, что определенная серия аккордов изменяет гармоническое напряжение таким образом, что я начинаю желать разрешения в новой тональности?

\textbf{Черепаха} : Именно так. Это усложняет ситуацию, поскольку, наряду с этим новым желанием, подсознательно вы все время ощущаете, что ваша конечная цель --- ключевая тональность, в данном случае, «до мажор». И когда временная цель бывает достигнута, то\ldots{}

\textbf{Ахилл (внезапно начиная возбужденно жестикулировать)} : О, послушайте только: какие восхитительные поднимающиеся вверх аккорды! Какой прекрасный конец у «Маленького гармонического лабиринта»!

\textbf{Черепаха} : Нет, Ахилл, это не конец, это просто ---

\textbf{Ахилл} : Да нет, разумеется, это конец! Вот это да! Какой могучий финал! Какое облегчение! Вот разрешение так разрешение! Гениально! (Поет): Ля-ля-ля\ldots{} (И точно, в этот момент музыка прекращается; стен больше нет, и Черепаха с Ахиллом оказываются в открытом пространстве.) Вот видите, музыка действительно кончилась. Ну, что я вам говорил?

\textbf{Черепаха} : Что-то здесь не так. Эта запись позорит музыкальный мир.

\textbf{Ахилл} : Почему это?

\textbf{Черепаха} : Я только что вам объяснил: Бах промодулировал здесь от «до» в «ля», так что временной целью было услышать мелодию в ключе «ля». Это значит, что вы чувствуете сразу два желания: с одной стороны, вы ожидаете разрешения в «ля», а с другой стороны, вы все время помните, что конечная цель --- триумфальное возвращение в «до мажор».

\textbf{Ахилл} : Почему надо все время о чем-то помнить, когда слушаешь музыку? Разве музыка --- только упражнение для ума?

\emph{Черепаха} : Нет, конечно. Некоторые произведения весьма интеллектуальны, но большинство довольно просты. Обычно наше ухо или мозг делают все «расчеты» за нас, в то время как чувства решают, что именно нам хочется услышать. Нам не приходится думать об этом. Но в этой пьесе Бах проделывает разные трюки, в надежде сбить слушателя с толку --- и надо сказать, что в вашем случае, Ахилл, он вполне преуспел!

\textbf{Ахилл} : Вы хотите сказать, что я среагировал на разрешение во «второстепенной» тональности?

\textbf{Черепаха} : Правильно.

\textbf{Ахилл} : Все же я и сейчас уверен, что это был конец!

\textbf{Черепаха} : Именно этого эффекта Бах и добивался. Вы угодили прямиком в его ловушку. Это место написано так, что оно звучит как финал; но если вы внимательно следите за развитием гармонической прогрессии, вы увидите, что оно не в том ключе. Видимо, не только вы, но и та несчастная студия звукозаписи решила, что это конец, и записала только часть пьесы!

\textbf{Ахилл} : Какую недостойную шутку сыграл со мной старик Бах!

\textbf{Черепаха} : Как раз этого он и хотел --- заставить вас заблудиться в его «Лабиринте». Видите ли, злодей Мажотавр --- сообщник Баха. Если вы не остережетесь, он засмеет вас до смерти --- а может быть, и меня вместе с вами!

\textbf{Ахилл} : Надо срочно уносить ноги отсюда! Скорее! Если мы побежим обратно по звуковым дорожкам, то выберемся из пластинки прежде, чем страшный Мажотавр нас обнаружит!

\textbf{Черепаха} : Ну нет, мое ухо слишком чувствительно, чтобы вынести странные аккорды, получающиеся, когда время обращается вспять!

\textbf{Ахилл} : Ах, г-жа Ч, как же мы выберемся отсюда, если мы не можем вернуться по нашим следам?

\textbf{Черепаха} : Хороший вопрос\ldots{} (Почти отчаявшись, Ахилл начинает бегать взад-вперед в темноте. Внезапно раздается сдавленный крик и затем --- БА-БАХ! --- глухой звук падения.) Ахилл? С вами все в порядке?

\textbf{Ахилл} : Ничего особенного, только маленькая встряска: я свалился в какую-то ямину.

\textbf{Черепаха} : Вы угодили прямиком в логово Страшного Мажотавра! Постараюсь вас вытащить --- нам надо удирать побыстрее!

\textbf{Ахилл} : Осторожнее, г-жа Ч --- я совсем не хочу, чтобы и Вы тоже попали в западню\ldots{}

\textbf{Черепаха} : Да не суетитесь вы, Ахилл. Все будет в порядке\ldots{} (Внезапно раздается сдавленный крик и затем --- БА-БАХ! --- глухой звук~~падения.)

\textbf{Ахилл} : Г-жа Ч, вы тоже упали? Не ушиблись?

\textbf{Черепаха} : Кроме моей гордости, ничего не пострадало.

\textbf{Ахилл} : Вот теперь мы действительно попали в переплет!

\emph{(Внезапно, в опасной близости от них, друзья слышат оглушительный хохот.)}

\textbf{Черепаха} : Осторожно, Ахилл --- тут дело нешуточное!

\textbf{Мажотавр} : Ха-ха-ха! Хи-хи-хи! Хо-хо-хо!

\textbf{Ахилл} : Я слабею на глазах, г-жа Ч\ldots{}

\textbf{Черепаха} : Старайтесь не обращать внимания на его смех --- это ваша единственная надежда.

\textbf{Ахилл} : Я сделаю все, что в моих силах --- ах, если бы сейчас пропустить для храбрости рюмочку-другую\ldots{}

\textbf{Черепаха} : Мне кажется, я чувствую знакомый запах\ldots{} Не вытаскин ли это?

\textbf{Ахилл} : И правда\ldots{} откуда этот запах?

\textbf{Черепаха} : По-моему, это здесь\ldots{} О! Я нашла целую бутыль! Это он и есть!

\textbf{Ахилл} : Вытаскин! Давайте напьемся с горя!

Черепаха: Надеюсь, что это не протолкин --- они до того похожи, что их трудно различить.

\textbf{Ахилл} : Что вы сказали про Толкиена?

\textbf{Черепаха} : Я ничего подобного не говорила. У вас уже галлюцинации начинаются\ldots{}

\textbf{Ахилл} : Б-батюшки мои! Надеюсь, что нет\ldots{} Ну что же, поехали!

\emph{(И друзья начинают отхлебывать вытаскин (или протолкин?) --- и вдруг --- ХЛОП! Кажется, это-таки оказался вытаскин\ldots)}

\emph{\textbf{Черепаха}} : Забавная история, ничего не скажешь. Вам понравилось?

\emph{\textbf{Ахилл}} : Так, ничего себе\ldots{} Интересно, выбрались ли они в конце концов из ямы страшного Мажотавра? Бедняга Ахилл, он так хотел опять стать большим.

\emph{\textbf{Черепаха}} : Не беспокойтесь --- они выбрались, и Ахилл снова вырос до своих обычных размеров. Вытаскин оказался весьма кстати\ldots{}

\emph{\textbf{Ахилл}} : Не знаю, не знаю\ldots{} Единственное, в чем я сейчас АБСОЛЮТНО уверен, это в том, что нам не мешало бы найти нашу бутылочку с настойкой --- у меня уже давно горло пересохло. И ничто так не утоляет жажду, как выталкивающая настойка

\emph{\textbf{Черепаха}} : Она к тому же известна своим тонизирующим~~действием. Известны случаи, когда народ просто с ума по ней сходил. Например, когда в начале века продуктовая фабрика Шёнберга перестала производить джин с тоником и начала производство какао, вы не представляете себе, какой из-этого поднялся шум --- настоящая какаофония!

\emph{\textbf{Ахилл}} : Воображаю\ldots{} Но давайте же искать настойку! Погодите --- взгляните-ка на этих ящериц на столе! Не кажется ли вам, что в них есть что-то необычное?

\emph{\textbf{Черепаха}} : Не вижу ничего особенного. А что такое?

\emph{\textbf{Ахилл}} : Посмотрите: они вылезают из плоскости картины без помощи выталкивающей настойки! Как они это делают?

\emph{\textbf{Черепаха}} : Разве я вам не говорила? Вы можете вылезти из картины, двигаясь перпендикулярно ее плоскости. Ящерки научились лезть НАВЕРХ, когда они хотят выбраться из двухмерного мира альбома.

\emph{\textbf{Ахилл}} : Может быть, мы можем так же выбраться из этой картины Эшера наружу?

\emph{\textbf{Черепаха}} : Разумеется --- нужно только подняться уровнем выше. Хотите попытаться?

\emph{\textbf{Ахилл}} : Все что угодно, только бы попасть домой! Я уже сыт по горло этими занимательными приключениями.

\emph{\textbf{Черепаха}} : В таком случае, следуйте за мной наверх.

\emph{(И они поднимаются на один уровень.)}

\emph{Ахилл} : Хорошо быть снова у себя дома\ldots{} Но постойте, здесь что-то не то! Это вовсе не мой дом --- это ВАШ дом, г-жа Черепаха!

\emph{Черепаха} : Вы правы --- и я предовольна, так как перспектива тащиться от вас к себе домой мне совершенно не улыбалась. Я прямо-таки валюсь с лап от усталости.

\emph{Ахилл} : Что ж, я как раз не возражаю против небольшой прогулки; так что, мне кажется, все сложилось довольно удачно.

\emph{Черепаха} : Я думаю! Вот это удача так удача!

