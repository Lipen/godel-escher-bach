\subsubsection{Хроматическая фантазия и фига}

\emph{Вдоволь наплававшись в пруду, Черепаха вылезает и отряхивается; тут мимо идет Ахилл.}

\emph{Черепаха} : День добрый, Ахилл. Я о вас только что вспоминала, пока купалась.

\emph{Ахилл} : Ну не забавно ли? И вы у меня из головы не выходили, пока я бродил по лугам. Смотрите, я нашел для вас фигу. Правда, она еще зеленая\ldots{}

\emph{Черепаха} : Вы полагаете? Это напоминает мне об одной идейке\ldots{} Хотите послушать?

\emph{Ахилл} : С превеликим удовольствием. Только, пожалуйста, без этих злодейских логических ловушек, г-жа Ч.

\emph{Черепаха} : Злодейских ловушек? Хорошо же вы обо мне думаете! Какая же я злодейка? Я мирная душа, никому не мешаю, живу спокойной травоядной жизнью. Мои мысли текут себе среди странностей и завихрений мироздания (так как я его вижу). Я, скромная наблюдательница явлений, бреду себе потихоньку и бросаю на ветер всякие глупости, которые, боюсь, никого не впечатляют. Но не волнуйтесь, Ахилл, сегодня я собиралась поговорить всего-навсего о своем панцире --- он-то уж не имеет к логике ни малейшего отношения.

\emph{Ахилл} : Вы меня НА САМОМ ДЕЛЕ успокоили, г-жа Ч. И, честно говоря, мое любопытство задето. Охотно вас послушаю, даже если это и не очень впечатляюще.

\emph{Черепаха} : Ну что ж\ldots{} с чего мне начать? Гмм\ldots{} Присмотритесь-ка к моему панцирю --- вас ничего не удивляет?

\emph{Ахилл} : Как будто почище стал?

\emph{Черепаха} : Премного благодарна. Я только что оставила в пруду несколько слоев грязи, накопившихся на мне за последнее столетие. Теперь вы можете увидеть, какой у меня зеленый панцирь!

\emph{Ахилл} : Такой крепкий, зеленый панцирь --- и как ярко он блестит на солнце!

\emph{Черепаха} : Зеленый? Он вовсе не зеленый.

\emph{Ахилл} : Вы же сами только что сказали, что ваш панцирь зеленый!

\emph{Черепаха} : Я так и сказала.

\emph{Ахилл} : В таком случае, мы согласны: он зеленый.

\emph{Черепаха} : Нет, он не зеленый.

\emph{Ахилл} : О, я понимаю: вы намекаете на то, что то, что вы говорите, не обязательно истинно, что Черепахи играют с языком, что ваши утверждения не всегда совпадают с действительностью, что\ldots{}

\emph{Черепаха} : Ничего подобного у меня и в мыслях не было! Слово для Черепах --- святыня; Черепахи преклоняются перед точностью.

\emph{Ахилл} : Хорошо, тогда почему же вы говорите, что ваш панцирь зеленый, и что он не зеленый?

\emph{Черепаха} : Никогда я ничего такого не говорила --- а жаль!

\emph{Ахилл} : Вы хотели бы это сказать?

\emph{Черепаха} : Нисколько. Я сожалею о том, что я это сказала, и совершенно с этим не согласна.

\emph{Ахилл} : Но это противоречит тому, что вы только что сказали!

\emph{Черепаха} : Противоречит? Противоречит? Я никогда себе не противоречу. Это не в черепашьем характере.

\emph{Ахилл} : Ну, на этот раз я вас поймал, хитрюга этакая! Это же самое настоящее противоречие!

\emph{Черепаха} : Вероятно, вы правы.

\emph{Ахилл} : Опять! Теперь вы противоречите себе еще больше! Вы настолько запутались в противоречиях, что с вами невозможно спорить!

\emph{Черепаха} : Вовсе нет. Я спорю сама с собой постоянно, и у меня это прекрасно получается. Может быть, дело в вас самих. Позволю себе предположить, что противоречивы именно вы --- но, поскольку вы сами себя совершенно запутали, вы не в состоянии заметить собственной непоследовательности.

\emph{Ахилл} : Какое оскорбительное предположение! Я вам покажу, что противоречите себе именно вы, и что об этом не может быть двух мнений.

\emph{Черепаха} : Что ж, если это так, Ахилл, то это дело должно быть вам по плечу. Нет ничего легче, чем указать на противоречие. Валяйте, доказывайте, Ахилл!

\emph{Ахилл} : Гмм\ldots{} Даже не знаю, с чего начать\ldots{} А! Теперь вижу. Вы сказали сначала, что (1) ваш панцирь зеленый и тут же, что (2) ваш панцирь не зеленый. Что тут добавишь?

\emph{Черепаха} : Осталось только указать на противоречие. Будьте любезны, перестаньте, наконец, ходить вокруг да около.

\emph{Ахилл} : Но\ldots{} но\ldots{} но\ldots{} О, теперь я понимаю. (Видите ли, иногда я такой тугодум!) Наверное, мы с вами по-разному понимаем противоречие. В этом-то вся загвоздка. Позвольте мне объясниться: противоречие возникает, когда кто-то утверждает одну вещь и одновременно ее отрицает.

\emph{Черепаха} : Вот ловкий трюк! Хотела бы я увидеть, как подобное возможно. Наверное, лучше всего противоречия получались бы у чревовещателей, которые могут говорить одновременно двумя сторонами рта. Но я-то не чревовещатель\ldots{}

\emph{Ахилл} : На самом деле, я имел в виду только то, что кто-то утверждает одну вещь и ее же отрицает в одном и том же предложении. Это не должно быть буквально в один и тот же момент.

\emph{Черепаха} : Однако в вашем примере не ОДНО предложение, а два!

\emph{Ахилл} : Да --- два предложения, противоречащих друг другу.

\emph{Черепаха} : Ну и путаница у вас в голове, бедняга! Сначала вы мне говорите, что противоречие --- это что-то, что должно быть в одном и том же предложении. Тут же вы утверждаете, что вы нашли противоречие в паре моих предложений. Так и есть --- ваш мыслительный процесс настолько запутан, что вы сами не видите, как вы непоследовательны. Со стороны, однако, это ясно как день.

\emph{Ахилл} : Вы меня совсем сбили с толку вашими отвлекающими маневрами. Я уже перестал понимать, идет ли речь о каких-то чепуховых мелочах или же о чем-то важном и глубоком.

\emph{Черепаха} : Уверяю вас, Черепахи не занимаются мелочами. Следовательно, верно второе.

\emph{Ахилл} : Вы меня успокоили, благодарю вас. Теперь, поразмыслив, я вижу логический шаг, необходимый, чтобы уверить вас в том, что вы противоречили себе.

\emph{Черепаха} : Отлично! Надеюсь, что этот шаг столь же легок, сколь бесспорен.

\emph{Ахилл} : Так и есть --- даже вы с ним согласитесь. Идея в том, что если вы считаете истинным предложение 1 («Мой панцирь зеленый») и предложение 2 («Мой панцирь не зеленый»), то вы должны считать истинной комбинацию этих двух предложений. Не так ли?

\emph{Черепаха} : Безусловно. Это только естественно\ldots{} если, конечно, все согласны с тем, КАК эти предложения комбинировать.

\emph{Ахилл} : Разумеется --- и тут-то я вас поймаю! Я предлагаю такую комбинацию ---

\emph{Черепаха} : С комбинированием предложений надо быть осторожнее. Разрешите мне это продемонстрировать. Наверняка, Ахилл, вы согласитесь со следующим предложением, описывающим ваш странный род:

\emph{У людей пять пальцев.}

К тому же, истинность его весьма нетрудно проверить, не так ли?

\emph{Ахилл (неуверенно)} : Соглашусь ли я? То есть, э-э, гмм\ldots{} как это я могу не согласиться с таким скучным и плоским утверждением? Минуточку\ldots{} (Смотрит себе на пальцы и бормочет.) Раз, два, три, четыре\ldots{} (Вслух, Черепахе) Г-жа Черепаха, а мизинец тоже считается за палец?

\emph{Черепаха} : Я думаю, да.

\emph{Ахилл (снова бормочет)} : Ага! Получается пять. Кажется, правильно. Я проверил все необходимые и достаточные условия истинности, так что\ldots{} (Вслух, на этот раз гораздо более уверенно): ЛЮБОЙ знает, что тривиальное суждение «у людей пять пальцев» --- истинно! Что может быть более очевидно?

\emph{Черепаха} : Разумеется. А теперь потрудитесь проверить почти такое же очевидное утверждение, а именно:

\emph{В этом предложении пять слов.}

\emph{Ахилл (бормочет себе под нос)} : Гмм\ldots{} раз\ldots{} два\ldots{} три\ldots{} четыре\ldots{} пять! Да, действительно, я должен согласиться с истинностью и этого утверждения. В ЭТОМ предложении я не вижу никаких проблем.

\emph{Черепаха} : Превосходно! Теперь, когда мои теоретические предположения получили экспериментальное подтверждение в ваших строгих исследованиях, я чувствую себя значительно лучше. Сейчас же, поскольку мы согласны по всем статьям, нам остается только соединить эти два невинных предложения в одно подлиннее, с помощью вашего безопасного слова «и».

\emph{Ахилл} : Именно «безопасного», г-жа Ч. Вам не удастся обвести меня вокруг пальца! Что ж, начнем, пожалуй\ldots{}

\emph{Черепаха} : Прекрасно. Посмотрим\ldots{} у меня получается следующее предложение, которое, безусловно, должно оказаться истинным:

\emph{У людей пять пальцев и в этом предложении пять слов.}

\emph{Ахилл} : Постойте, г-жа Ч! Что-то здесь не то!

\emph{Черепаха (всем своим видом выражая невинное удивление)} : Что? Что вы имеете в виду?

\emph{Ахилл} : Вы соединяете эти предложения неправильно!

\emph{Черепаха} : Я только последовала вашему совету и использовала ваше любимое «и».

\emph{Ахилл} : Не знаю, не знаю\ldots{} То, что у вас получилось, НЕЛОГИЧНО! Где-то здесь должна быть ошибка\ldots{}

\emph{Черепаха} : Ну вот, вы снова заговорили о г-же Логике и ее великих принципах\ldots{} Будьте так любезны, увольте --- хотя бы на сегодня.

\emph{Ахилл} : Г-жа Черепаха, у меня уже черепушка трещит от всего этого. Признайтесь, что вы немного сжульничали\ldots{}

\emph{Черепаха} : Пожалуйста, не обвиняйте меня в собственных грехах, кто из нас хотел соединить два высказывания с помощью «и».Мне кажется, я только следовала вашим пожеланиям --- и какова же ваша благодарность? Ну и молодежь нынче пошла\ldots{}

\emph{Ахилл} : Ну вот, я же и виноват. Ведь я хотел, как лучше\ldots{}

\emph{Черепаха} : Добрыми намерениями, мой юный друг, вымощена дорога в преисподнюю\ldots{}

\emph{Ахилл} : Я чувствую себя ужасно\ldots{}

\emph{Черепаха} : Я отлично понимаю, куда вы клонили: вы хотели заставить меня принять за истинную фразу «Мой панцирь зеленый и мой панцирь не~зеленый». О, создатель!\ldots{} Какая страшная ложь, и как она противна черепашьему духу!

\emph{Ахилл} : Умоляю вас простить меня, дурака\ldots{} Честное слово, у меня и в мыслях~не было вас обидеть.

\emph{Черепаха} : Ничего, мой друг --- мы, черепахи, привыкли к людской бестактности. Я ценю вашу компанию, Ахилл, пусть ваши мысли и не так кристально ясны, как у созданий нашей хладнокровной породы.

\emph{Ахилл (вздыхая)} : Надеюсь, что для меня еще не все потеряно --- хотя я, наверняка, сделаю еще немало ложных шагов на пути к истине\ldots{}

\emph{Черепаха} : Мужайтесь, Ахилл. Может быть, наша сегодняшняя беседа вам поможет\ldots{} Кстати, не забудьте отдать мне ту фигу, что вы мне принесли. Хоть она и зеленая, все равно пригодится!

\emph{Ахилл} : Вот, возьмите.

\emph{Черепаха} : Что ж, до скорого, мой друг.

\emph{Ахилл} : До скорого.

