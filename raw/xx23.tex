\subsubsection{Крабий канон}

\emph{В один прекрасный день, Ахилл и Черепаха, прогуливаясь по парку, наталкиваются друг на друга.}

\emph{Черепаха} : Приветствую, г-н А.!

\emph{Ахилл} : И я вас тоже.

\emph{Черепаха} : Всегда рада вас видеть.

\emph{Ахилл} : Вы читаете мои мысли.

\emph{Черепаха} : В такой денек приятно пройтись; пожалуй, я пойду домой пешочком.

\emph{Ахилл} : Неужели? Гулять, знаете ли, весьма полезно для здоровья.

\emph{Черепаха} : Кстати, в последнее время вы выглядите как огурчик.

\emph{Ахилл} : О, благодарю вас.

\emph{Черепаха} : Не стоит. Не желаете ли угоститься моими сигарами?

\emph{Ахилл} : Да вы, как я погляжу, филистер. По моему мнению, голландский вклад в эту область --- значительно худшего вкуса, и я хочу попытаться вас в этом убедить.

\emph{Черепаха} : Наши мнения по этому вопросу расходятся. Кстати, говоря о вкусах: несколько дней назад я была на выставке, где, наконец, увидела «Крабий канон» вашего любимого художника, М. К. Эшера. Какая красота! Как ловко он переворачивает тему задом наперед! Но боюсь, что для меня Бах всегда останется выше Эшера.

\emph{Ахилл} : Не знаю, не знаю\ldots{} Я уверен только в том, что меня не волнуют споры о вкусах. De gustibus non est disputandum.

\emph{Черепаха} : Поговорим лучше о другом. Знаете ли вы, что я уже давно пытаюсь собрать полную коллекцию редких записей Баха --- хоть это и отнимает много времени, но я считаю, что лучшего хобби не найти.

\emph{Ахилл} : Ну и волокита! Не знаю, как кому-то могут нравиться такие вещи\ldots{}

\emph{(Вдруг, откуда ни возьмись, появляется Краб. Он стремительно подбегает к друзьям, указывая на огромный синяк под глазом.)}

\emph{Краб} : Приветик! Бонжурчик! Я сегодня как огурчик, только вот синяк --- кошмар, не правда ли? Мне его наставил этот поляк, ужасный, скажу вам, пошляк. Хо! Да еще в такой чудесный денек! Я себе по парку гулял, никого не задирал; вдруг слышу --- музыка небесная, полька расчудесная. Гляжу, а на скамье сидит девица, да такая, что нам с вами не пара; а в руках у нее --- гитара. Я и сам, знаете ли, из музыкальной семьи: мой кузен рак --- мужик не дурак! --- всегда зимовал ничуть не ближе, чем в Париже. Он был придворным музыкантом короля --- услаждал его величество художественным свистом, когда тот сидел с придворными за вистом. Любовь к музыке у нас, ракообразных, в крови\ldots{} Понимаете теперь, почему я не удержался, на скамейку взобрался, и говорю на ушко девице: «Щипать струны вы, гляжу, мастерица! Позвольте мне, как музыканту, сделать вам комплимент --- а также предложить свой аккомпанемент. Чтоб польке дать полнее звук, сыграем-ка в двенадцать рук!» Она как вскочит, да как завопит, что есть мочи! Тут откуда ни возьмись, явился этот здоровяк, этот поляк\ldots{} Бах! Трах! Прямо в глаз попал --- вот откуда этот фингал! Не думайте, что я трус --- атаковать я не боюсь. Но по давней семейной традиции, крабы --- мастера защитной диспозиции\ldots{} Ведь мы, когда идем вперед, движемся назад. Это у нас в генах --- переворачивать все задом наперед. Кстати, это мне напоминает\ldots{} Я всегда спрашивал себя: «Что было раньше, Краб или Ген?» То есть, я хочу сказать: «Что было позже, Ген или Краб»? Я всегда переворачиваю все задом наперед, знаете ли --- это у нас в генах. Ведь мы, когда идем вперед, движемся назад\ldots{} Ох, и заболтался же я, друзья! Да еще в такой чудный денек, хо! Поползу себе, пожалуй. Приветик!

\emph{(И он исчезает так же внезапно, как и появился.)}

\emph{Рис. 43. Кусочек одного из Крабьих Генов. Если спирали ДНК развернуть и положить рядом, то получится следующая картина: TTTTTTTCGAAAAAAA ... AAAAAAAGTTTTTTTT... Обратите внимание на то, что спирали одинаковы - разница только в том, что одна из них идет в обратном порядке. Эта черта определяет также музыкальную форму под названием ракоход, или «крабий канон.» Очень похожи на это и палиндромы --- предложения, которые при прочтении задом наперед дают точно тот же результат. В молекулярной биологии подобные сегменты ДНК называются «палиндромами» --- ко самом деле, более точным названием было бы «крабий канон». Этот сегмент ДНК не только «крабо-каноничен» --- в его основной структуре также закодирована структура Диалога. Присмотритесь повнимательней!}

\emph{Черепаха} : Ну и волокита! Не знаю, как кому-то могут нравиться такие вещи\ldots{}

\emph{Ахилл} : Поговорим лучше о другом. Знаете ли вы, что я уже давно пытаюсь собрать полную коллекцию редких гравюр Эшера --- хоть это и отнимает много времени, но я считаю, что лучшего хобби не найти.

\emph{Черепаха} : Не знаю, не знаю\ldots{} Я уверена только в том, что меня не волнуют споры о вкусах. Disputandum non est de gustibus.

\emph{Ахилл} : Наши мнения по этому вопросу расходятся. Кстати, говоря о вкусах: несколько дней назад я был на концерте, где наконец, услышал «Крабий канон» вашего любимого композитора, И. С. Баха. Какая красота! Как ловко он переворачивает тему задом наперед! Но боюсь, что для меня Эшер всегда останется выше Баха.

\emph{Черепаха} : Да вы, как я погляжу, филистер. По моему мнению, голландский вклад в эту область --- значительно худшего вкуса, и я хочу попытаться вас в этом убедить.

\emph{Ахилл} : Не стоит. Не желаете ли угоститься моими сигарами?

\emph{Черепаха} : О, благодарю вас.

\emph{Ахилл} : Кстати, в последнее время вы выглядите как огурчик.

\emph{Черепаха} : Неужели? Гулять, знаете ли, весьма полезно для здоровья.

\emph{Ахилл} : В такой денек приятно пройтись; пожалуй, я пойду домой пешочком.

\emph{Черепаха} : Вы читаете мои мысли.

\emph{Ахилл} : Всегда рад вас видеть.

\emph{Черепаха} : И я вас тоже.

\emph{Ахилл} : Приветствую, г-жа Ч.

\emph{РИС. 44. «Крабий канон»~из «Музыкального приношения» И. С. Баха.}

