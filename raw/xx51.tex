\subsection{Библиография}

Книги, отмеченные двумя звездочками, были основными источниками для моей книги. Одна звездочка означает, что данная книга или статья чем-то удивительна и необычна и заслуживает внимания. Я почти не ссылаюсь на техническую литературу. Вместо этого я привожу «мета-ссылки» ссылки на книги, которые в свою очередь ссылаются на техническую литературу.

Allen, John «The Anatomy of LISP» New York McGraw Hill, 1978. Исчерпывающее описание ЛИСПа компьютерного языка, доминирующего в области исследований по искусственному интеллекту в течение последних двадцати лет. Краткое и ясное изложение.

** Anderson, Alan Ross ed «Minds and Machines» Englewood Cliffs, N J Prentice Hall 1964. Сборник стимулирующих статей за и против искусственного интеллекта. Содержит знаменитую статью Тюринга «Вычислительные машины и интеллект», а также раздражающую статью Лукаса «Интеллект, машины и Гедель».

Babbage, Charles «Passages from the Life of a Philosopher» London Longman, Green, 1864. Перепечатано в 1968 году, Dawsons of Pall Mall, Лондон. Беспорядочный набор мыслей и событий из жизни этого недооцененного гения. Читатель найдет там даже пьесу, герой которой --- Турнстайль, бывший философ, а ныне политик, увлекающийся игрой на шарманке. Книга показалась мне довольно забавной.

Baker, Adolph «Modern Physics and Anti-physics» Reading, Mass Addison-Wesley, 1970. Книга о современной физике, делающая упор на квантовой механике и относительности. Необычная черта книги --- диалоги между Поэтом (противником науки) и Физиком. Эти диалоги --- пример странных проблем, возникающих, когда один собеседник использует логическое мышление для защиты логического мышления, а другой оборачивает логику против самой себя.

Ball W. W. Rouse «Calculating Prodigies», in James R. Newman, ed «The World of Mathematics, Vol 1 New York Simon \& Schuster, 1956. Рассказы о людях соперничавших в своих способностях к вычислениям с машинами.

Barker Stephen F. «Philosophy of Mathematics» Englewood Cliffs, N. J. Prentice Hall 1969. Краткое обсуждение эвклидовой и неэвклидовой геометрии, а также теоремы Геделя и её результатов Без математических формализмов.

* Beckmann, Petr «A History of Pi» New York St Martin's Press, 1976. На самом деле, это история всего мира в фокусе которой --- число пи. Увлекательное чтение для тех кто интересуется историей математики.

* Bell, Eric Temple «Men of Mathematics» New York Simon \& Schuster, 1965. Возможно самый романтический писатель, когда-либо писавший об истории математики. Каждая биография читается, как маленький роман. Даже далекие от математики люди, прочитав эту книгу, могут почувствовать мощь, красоту и значение математики.

Benacerraf Paul «God the Devil and Godel» «Monist» 51 1967) 9. Одна из важнейших попыток опровержения Лукаса. Сведения о метафизике и механицизме в свете трудов Геделя.

Benacerraf, Paul, and Hilary Putnam. «Philosophy of Mathematics --- Selected Readings.» Englewood Cliffs, N.J.: Prentice Hall, 1964. Сборник статей ведущих математиков о реальности чисел и множеств, природе математической истины и т. д.

* Bergerson, Howard. «Palindromes and Anagrams.» New York: Dover Publications, 1973. Превосходное собрание самых странных и невероятных примеров игры слов в английском языке. Палиндромные стихотворения, пьесы, рассказы и так далее.

Bobrow, D. G., and Allan Collins, eds. «Representations and understanding; studies in cognitive science. New York: Academic Press, 1975. Эксперты по искусственному разуму спорят о природе «фреймов», декларативном и процедурном представлениях информации и так далее. В каком-то смысле, эта книга отмечает начало новой эры в искусственном интеллекте --- эры представления.

* Boden, Margaret. «Artificial Intelligence and Natural Man». New York: Basic Books, 1977. Лучшая книга по искусственному интеллекту, которую я когда-либо читал, включая технические и философские аспекты. Я считаю, что это --- классический труд. Книга продолжает английскую традицию ясного мышления и выражения идей, касающихся разума, свободной воли и т. п. Содержит обширную техническую библиографию.

-\/-\/-\/-.«Purposive Explanations in Psychology». Cambridge, Mass.: Harvard University Press. Боден утверждает, что её труд по искусственному интеллекту --- лишь расширенное примечание к этой книге.

* Boeke, Kees. «Cosmic View: The Universe in 40 Jumps». New York: John Day, 1957. Последнее слово в обсуждении уровней описания. Рано или поздно, эту книгу должен прочесть любой. Годится для детей.

** М. М. Бонгард. «Проблема узнавания», М., «Наука», 1967. Автор раздумывает над определением категорий в неопределенном пространстве. Книга содержит великолепное собрание 100 «задач Бонгарда» (как я их называю) --- головоломки для искателя закономерностей, человеческого или механического. Они стимулируют мысль любого человека, интересующегося разумом.

Boolos, George, S., and Richard Jeffrey. «Computability and Logic». New York: Cambridge University Press, 1974. Продолжение книги Джеффри «Формальная логика». Содержит большое количество результатов, которые нелегко найти в других источниках. Легко читается, несмотря на строгость изложения.

Carroll, John В., Peter Davies, and Barry Rickman. «The American Heritage Word Frequency Book». Boston: Houghton Mifflin, and New York: American Heritage Publishing Co., 1971. Список слов в порядке частотности в современном письменном американском варианте английского. Знакомство с этой книгой открывает интереснейшие факты о мыслительном процессе.

Cerf, Vinton. «Parry Encounters the Doctor». «Datamation», июль 1973, стр. 62- 64. Первая встреча двух искусственных разумов --- какой шок!

Chadwick, John. «The Decipherment of Linear B». New York: Cambridge University Press, 1958. Книга о классической расшифровке критской письменности, проделанной одним-единственным человеком --- Майклом Вентрисом.

Chaitin, Gregory J. «Randomness and Mathematical Proof». «Scientific American», май 1975. Статья об алгоритмическом определении случайности и близком родстве последней с простотой. Эти два понятия соотнесены с теоремой Гёделя, которая приобретает новое значение.

Cohen Paul С. «Set Theory and the Continuum Hypothesis» Menlo Park, Calif W A Benjamin, 1966. Значительный вклад в современную математику --- доказательство того, что некоторые суждения неразрешимы в обычной формальной системе теории множеств --- поясняется здесь неспециалистам самим открывателем. Необходимые сведения по математической логике представлены четко, коротко и ясно.

Cooke, Deryck «The Language of Music» New York Oxford University Press, 1959. Единственная известная мне книга, которая пытается установить связь между элементами музыки и элементами человеческих эмоций. Первый шаг по длинной и трудной дороге, ведущей к пониманию музыки и человеческого разума.

* David Hans Theodore «J. S. Bach's Musical Offering History, Interpretation, and Analysis» New York Dover Publications, 1972. Хорошо написанная книга --- богатый источник информации о Баховском шедевре.

** David, Hans Theodore, and Arthur Mendel «The Bach Reader» New York W. W. Norton 1966. Великолепное аннотированное сборник материалов о жизни Баха. Содержит иллюстрации, фотографии страниц рукописей, высказывания из современников, случаи из жизни композитора и т. д. и т. п.

Davis, Martin «The Undecidable» Hewlett, N. Y. Raven Press, 1965. Антология важнейших работ по метаматематике с 1931 года (дополняющая таким образом антологию Ван Хейенорта). Включает перевод статьи Геделя 1931 года, конспект лекций, прочитанных Геделем о его результате, и статьи Черча, Клини Россера, Поста и Тюринга.

Davis, Martin, and Reuben Hersh «Hilbert's Tenth Problem» «Scientific American», ноябрь 1973, стр. 84. О том как 22-летний русский доказал неразрешимость знаменитой проблемы теории чисел.

** DeLong, Howard «A Profile of Mathematical Logic» Reading, Mass Addison-Wesley, 1970. Строгий труд по математической логике с объяснением теоремы Геделя и обсуждением многих философских вопросов. Включает отличную аннотированную библиографию. Эта книга оказала на меня большое влияние.

Doblhofer, Ernst «Voices in Stone» New York Macmillan, Collier Books, 1961. Хорошая книга о расшифровке древних письменностей.

* Dreyfus, Hubert «What Computers Can't Do A Critique of Artificial Reason» New York Harper \& Row 1972. Различные доводы против искусственного интеллекта, представленные неспециалистом. Интересно попытаться их опровергнуть. Общество работников ИИ и Дрейфус находятся в отношениях полнейшего антагонизма. Люди, подобные Дрейфусу, необходимы, хотя порой и раздражают.

Edwards Harold M «Fermat s Last Theoiem» «Scientific American», октябрь 1978, стр 104-122. Обсуждение одного из самых крепких орешков математики с его рождения до последних результатов. Отлично иллюстрировано.

* Ernst, Bruno «The Magic Mirror of M С Escher» New York Random House, 1976. Эшер как человек и как художник глазами его старого друга. Необходимо прочитать каждому поклоннику Эшера.

** Escher, Maunts С et al «The World of M С Escher» New York Harry N Abrams 1972. Наиболее полное собрание репродукций работ Эшера. Эшер подходит к понятию рекурсии так близко, как только возможно, и в некоторых своих рисунках удивительно хорошо передает дух теоремы Геделя.

Feigenbaum, Edward, and Julian Feldman, eds. «Computers and Thought». New York: McGraw Hill, 1963. Хотя и немного устаревшая, эта книга --- все еще важное собрание идей об искусственном интеллекте. Включает статьи о геометрической программе Гелернтера, шашечной программе Самуэля, узнавании структур, понимании языка, философии и т. д.

Finsler, Paul. «Formal Proofs and Undecidability». Перепечатано в антологии Ван Хейхенорта «От Фреджа до Гёделя» (см. ниже). Предшествует работе Гёделя. Намекает на существование неразрешимых математических суждений, хотя и не доказывает этого с точностью.

Fitzpatrick, P. J. «To Godel via Babel». «Mind» 75 A966): 332-350. Изобретательное изложение доказательства Гёделя. Для различения важных уровней автор использует три разных языка: английский, французский и латинский!

von Foerster, Heinz, and James W. Beauchamp, eds. «Music by Computers». New York, John Wiley, 1969. Кроме подборки статей о разных типах компьютерной музыки, интересно также приложение: четыре пластинки, чтобы читатель мог услышать и оценить описанное. Среди записей --- запрограммированная Максом Матьюсом смесь маршей «Джонни марширует домой» и «Британские гренадеры».

Frenkel, Abraham, Yehoshua Bar-Hillel, and Azriel Levy. «Foundations of Set Theory», 2nd ed. Atlantic Highlands, N. J.: Humanities Press, 1973. Малотехничное обсуждение теории множеств, логики, ограничительных теорем и неразрешимых суждений. Включает подробное обсуждение интуиционизма.

Frey, Peter W. «Chess Skill in Man and Machine.» New York: Springer Verlag, 1977, Отличный обзор современных идей о компьютерных шахматах: почему программы работают, почему они не работают... История и будущее компьютерных шахмат.

Friedman, Daniel P. «The Little Lisper». Palo Alto, Calif.: Science Research Associates, 1974. Вполне удобоваримое введение в рекурсивное мышление ЛИСПа. Читается за один присест!

Gablik, Suzi. «Magritte». Boston, Mass.: New York Graphic Society, 1976. Отличная книга о Магритте и его работах; содержит хорошую подборку репродукций.

* Gardner, Martin. «Fads and Fallacies». New York: Dover Publications, 1952. Возможно, что эта книга до сих пор является лучшим опровержением оккультизма. Хотя, скорее всего, книга не была задумана как труд по истории философии, в ней содержится немало сведений из этой области. Снова и снова читатель сталкивается с вопросом: «Что такое очевидность?» Гарднер показывает, что для открытия истины необходима как наука, так и искусство.

Gebstadter, Egbert В. «Copper, Silver, Gold: an Indestructible Metallic Alloy». Perth: Acidic Books, 1979. Ужасная мешанина, непонятная и запутанная --- и при этом удивительно похожая на данную книгу. Профессор Гебштадтер приводит отличные примеры косвенной автореференции. Особенно интересна полностью аннотированная библиография, включающая ссылку на изоморфную, но воображаемую книгу.

** Godel, Kurt. «On formally undecidable propositions.» New York Basic Books, 1962. Перевод статьи 1931 года и её обсуждение.

-\/-\/-«Uber Formal Unentscheidbare Satze der «Principia Mathematica» und Verwandter Systeme, I». «Monatshefte fur Mathematik und Physik», 38 A931), 173-198. Статья Гёделя 1931 года.

* Goffman, Erving. «Frame Analysis». New York: Harper \& Row, Colophon Books, 1974. Подробные сведения об определении «систем» в человеческой коммуникации и о том, как граница между «системой» и «реальностью» воспринимается, используется и нарушается в живописи, театре, репортаже и реклаламе.

Goldstein, Ira, and Seymour Papert «Artificial Intelligence, Language, and the Study of Knowledge» «Cognitive Science»~ (январь 1977) 84-123. Обзор прошлого и будущего искусственного разума Авторы подразделяют историю ИИ на три периода классический, романтический и современный.

Good I J «Human and Machine Logic» «British Journal for the Philosophy of Science» 18 A967) 144. Одна из интереснейших попыток опровержения Лукаса, рассматривает вопрос, возможно ли механизировать саму повторную операцию диагонализации.

-\/-\/-«Godel's Theorem is a Red Herring» «British Journal for the Philosophy of Science» 19 A969) 357. Гуд утверждает, что доводы Лукаса не имеют никакого отношения к теореме Геделя, и что Лукас должен был бы назвать свою статью «Minds, Machines and Transhnite Counting» («Разум, машины и трансфинитные вычисления») Спор Гуда и Лукаса очень интересен.

Goodman, Nelson «Fact, Fiction, and Forecast» 3rd ed Indianapolis Bobbs-Mernll, 1973. Книга посвящена обсуждению гнпотетических ситуаций и индуктивной логики, включает знаменитые парадоксальные неологизмы Гудмана «blееn» и «grue» (смесь слов «green» и «blue» --- зелёный и голубой) Книга особенно интересна с точки зрения ИИ, поскольку Гудман останавливается на вопросе человеческого восприятия.

* Goodstein, R L «Development of Mathematical Logic» New York Springer Verlag, 1971. Краткий обзор математической логики, включает материалы, которые трудно найти в другом месте Хорошая книга, полезная для справок.

Gordon, Cyrus «Forgotten Scripts» New York Basic Books, 1968. Краткий и хорошо написанный обзор истории расшифровки древних письменностей иероглифов, клинописи и других.

Griffin, Donald The Question of Animal Awareness» New York Rockfeller University Press, 1976. Небольшая книга о пчелах, человекообразных обезьянах и других животных, о том обладают ли они «сознанием» и в особенности о том, правомочно ли вообще использовать слово «сознание», объясняя поведение животных.

deGroot, Adnaan «Thought and Choice in Chess» The Hague Mouton, 1965. Глубокое исследование в области когнитивной психологии, описывает классически простые и элегантные эксперименты.

Gunderson, Keith «Mentality and Machines» New York Doubleday, Anchor Books, 1971. Ярый противник ИИ объясняет свою позицию. Местами довольно смешно.

** Hanawalt, Philip С, and Robert H Haynes, eds «The Chemical Basis of Life» San Francisco W H Freeman, 1973. Отличный сборник статей из «Сайентифк Американ» Дает хорошее представление о том, чем занимается молекулярная биология.

* Hardy, G Н, and E M Wright «An Introduction to the Theory of Numbers», 4th ed New York Oxford University Press, 1960. Классический труд по теории чисел Набит информацией об этих загадочных существах --- целых числах.

Harmon, Leon «The Recognition of Faces» «Scientific American», ноябрь 1973, стр. 70. Исследование того, как мы представляем лица в памяти и какая информация нам необходима, чтобы узиать знакомое лицо. Одна из самых интересных задач узнавания структур.

van Heijenoort, Jean. «From Frege to Godel: A Source Book in Mathematical Logic». Cambridge, Mass.: Harvard University Press, 1977. Сборник важнейших статей по математической логике, приведших к поразительному открытию Гёделя (последняя статья книги).

Henri, Adrian. «Total Art: Environments, Happenings, and Performances». New York: Praeger, 1974. Показывает, как в современном искусстве значение выродилось настолько, что само его отсутствие приобретает глубокое значение (что бы это ни означало).

* Ноаге, С. A. R., and D. С. S. Allison. «Incomputability». «Computing Surveys» 4, по.З (Сентябрь 1972). Хорошо изложенное объяснение того, почему проблема остановки неразрешима. Доказывает следующую фундаментальную теорему: любой компьютерный язык, в котором есть условное наклонение и определения через рекурсивную функцию, достаточно мощный, чтобы запрограммировать собственного интерпретатора, не может быть использован для того, чтобы запрограммировать собственную функцию остановки.

Hofstadter, Douglas R. «Energy levels and wave (unctions of Bloch electrons in rational and irrational magnetic fields». «Physical Review B», 14, no. 6 15 сентября 1976). Докторская диссертация автора, представленная в форме статьи. Детально показан рекурсивный график G, представленый иа рис. 34.

Hook, Sidney, ed. «Dimensions of Mind». New York: Macmillan, Collier Books, 1961. Сборник статей о проблемах разума и мозга, а также разума и компьютера. Некоторые статьи довольно категоричны.

* Horney, Karen. «Self-Analysis». New York: W. W. Norton, 1942. Интереснейшее описание того, как запутываются уровни самовосприятия, когда человек пытается понять самого себя в этом сложном мире. Человечный и глубокий труд.

Hubbard, John I. «The Biological Basis of Mental Activity». Reading, Mass.: Addison-Wesley, 1975. Еще одна книга о мозге. её достоинство в том, что она содержит длинный список вопросов для размышления и ссылки на статьи, отвечающие на эти вопросы.

* Jackson, Philip С. «Introduction to Artificial Intelligence». New York: Petrocelli Charter, 1975. Книга, с энтузиазмом описывающая идеи ИИ, на многие из которых автор только намекает; именно поэтому её интересно даже перелистать. Другая причина, по которой книга достойна рекомендации, --- её обширная библиография.

Jacobs, Robert L. «Understanding Harmony». New York: Oxford University Press, 1958. Прямолинейная книга о гармонии, заставляющая читателя задаться вопросом о том, почему условная гармония европейской цивилизации настолько привлекает наш мозг.

Jaki, Stanley L. «Brain, Mind, and Computers». South Bend, Ind.: Gateway Editions, 1969. Полемическая книга, каждая страница которой дышит ненавистью к попыткам понять разум при помощи компьютеров. Тем не менее, некоторые из идей интересны.

* Jauch, J. M. «Are Quanta Real?» Bloomington, Ind.: Indiana University Press, 1973. Прелестная книжица диалогов, три героя которых заимствованы из Галилея и пересажены на современную почву. Обсуждает не только вопросы квантовой механики, но также темы узнавания структур, простоты, мозговых процессов и философии науки. Чтение этой книги доставляет истинное удовольствие и стимулирует ум.

* Jeffrey, Richard. «Formal Logic: Its Scope and Limits». New York: McGraw Hill, 1967. Легко читаемый учебник, последняя глава которого посвящена теоремам Гёделя и Чёрча. В этой книге читатель найдет подход, отличный от большинства учебников по логике; это делает её достойной внимания.

* Jensen, Hans. «Sign, Symbol, and Script». New York: G. P. Putnam's, 1969. Возможно, наилучшая книга о символических письменностях мира, как современных, так и древних. В книге много красоты и тайны --- например, нерасшифрованная письменность острова Пасхи.

Какпйг, Laszld. «An Argument against the Plausibility of Church's Thesis». В сб. A. Heiting, ed. «Constructivity in Mathematics: Proceedings of the Colloquium held at Amsterdam», 1957, North-Holland, 1959 . Интересная статья, написанная, возможно, самым ярым скептиком в отношении Тезиса Чёрча-Тюрннга.

* Kim, Scott E. «The Impossible Skew Quadrilateral: A Four-Dimensional Optical Illusion». В сб. David Brisson, ed. «Proceedings of the 1978 A.A.A.S. Symposium on Hypergraphics: Visualizing Complex Relationships in Art and Science». Boulder, Colo.: Westview Press, 1978. To, что на первый взгляд кажется невероятно сложной идеей, постепенно становится ясным, как день, благодаря виртуозному изложению и серии прекрасно сделанных диаграмм. Форма статьи так же необычна и интригующа, как и её содержание: она трехчастна одновременно на нескольких уровнях. Эта статья писалась одновременно с моей книгой, и они взаимно стимулировали друг друга.

Kleene, Stephen С. «Introduction to Mathematical Logic». New York: John Wiley, 1967. Полный и вдумчивый текст, написанный экспертом в этой области. Заслуживает всяческого внимания. Перечитывая эту книгу, в каждом абзаце я нахожу для себя что-то новое.

-\/-\/- «Introduction to Metamathematics». Princeton: D. Van Nostrand, 1952. Классический труд по математической логике; учебник, приведенный выше, представляет из себя сокращенную версию. Сейчас этот строгий и полный труд немного устарел.

Kneebone, G. J. «Mathematical Logic and the Foundations of Mathematics». New York: Van Nostrand Reinhold, 1963. Серьезная книга --- философское обсуждение таких вопросов как интуиционизм, «реальность» натуральных чисел, и так далее.

Koestler, Arthur. «The Act of Creation». New York: Dell, 1966. Интересная теория о том, как, соединяя идеи, можно получить нечто новое. Книгу можно читать с любого места.

Koestler, Arthur, and J. R. Smythies, eds. «Beyond Reductionism». Boston: Beacon Press, 1969. Материалы конференции, участники которой считали, что биологические системы нельзя объяснить с редукционистской точки зрения и что жизнь --- это нечто, «возникающее внезапно». Одна из тех интересных книг, которые кажутся в чем-то неверными, но в которых очень трудно найти конкретные ошибки.

Kubose, Gyomay. «Zen Koans». Chicago: Regnery, 1978. Одно из лучших известных мне собраний коанов. Книга, необходимая для библиотеки дзен-буддиста.

Kuffler, Stephen W. and John G. Nicholls. «From Neuron to Brain». Sunderland, Mass.: Sinauer Associates, 1976. Несмотря на свое название, эта книга в основном рассматривает микроскопические процессы мозга и почти не уделяет внимания тому, как из путаницы нейронов возникают человеческие мысли. Особенно подробно прокомментирована работе Хубеля и Визеля о зрительных системах.

Lacey, Hugh, and Geoffrey Joseph. «What the Gbdel Formula Says». «Mind» 77 A968): 77. Полезное обсуждение значения результатов Гёделя, основанное на четком разделении трех уровней: неинтерпретированные формальные системы, интерпретированные формальные системы и метаматематика. Книга стоит изучения.

Lakatos, Imre. «Proofs and Refutations». New York: Cambridge University Press, 1976. Очень интересная книга, в форме диалогов обсуждающая формирование идей в математике. Полезна не только для математиков, но и для людей, интересующихся мыслительными процессами.

** Lehninger, Albert «Biochemistry». New York: Worth Publishers, 1976. Несмотря на высокотехнический уровень, книга довольно легко читается. В ней можно найти множество примеров переплетения белков и генов. Материал хорошо подан и очень интересен.

** Lucas J. R. «Minds, Machines, and Gadel». «Philosophy» 36 A961): 112. Перепечатано сб. Андерсон «Minds and Machines», а также в Sayre and Crosson «The Modeling of the Mind». Противоречивая и вызывающая статья; автор утверждает, что он нашел доказательство того, что человеческий мозг в принципе не может быть смоделирован при помощи компьютерной программы. Его интересные доводы целиком основаны на теореме неполноты Гёделя. Стиль этой статьи кажется мне необыкновенно раздражающим --- и именно поэтому забавным для чтения.

-\/-\/- «Satan Stultified: A Rejoinder to Paul Benacerraf». Monist 52 A968)- 145. Полемика с идеями Пола Бенасеррафа, написанная в забавно ученом стиле. Борьба Лукаса с Бенасеррафом, как и борьба Лукаса с Гудом, представляет богатую пищу для ума.

-\/-\/- «Human and Machine Logic: A Rejoinder». British Journal for the Philosophy Science 19 A967): 155. Попытка опровержения предпринятой Гудом попытки опровержения первоначальной статьи Лукаса.

** MacGillavry, Caroline H. «Symmetry Aspects of the Periodic Drawings of M. C. Escher». Utrecht: A. Oosthoek's Uitgevermaatschappij, 1965. Мозаичные рисунки Эшера с научными комментариями кристаллографа. Источник некоторых моих иллюстраций --- например, «Муравьиной фуги» и «Крабьего канона». Переиздано в 1976 году в Нью-Йорке под названием «Fantasy and Symmetry».

MacKay, Donald M. «Information, Mechanism, and Meaning». Cambridge, Mass.: M.I.T. Press, 1970. Книга о различных измерениях информации, применимых к разным ситуациям; теоретические вопросы человеческого восприятия и понимания; объяснение того, как сознание может возникнуть на механической основе.

* Mandelbrot, Benoft. «Fractals: Form, Chance, and Dimension». San Francisco: W. H. Freeman, 1977. Редкая книга --- собрание иллюстраций к сложным современным идеям математики. Речь идет о рекурсивно определенных кривых и фигурах, чья размерность не выражается целым числом. Удивительным образом, Мандельброт показывает их отношение практически ко всем отраслям науки.

McCarthy, John. «Ascribing Mental Qualities to Machines». В сб. Martin Ringle, ed. «Philosophical Perspectives in Artificial Intelligence». New York: Humanities Press, 1979. Глубокая статья, исследующая, при каких обстоятельствах можно сказать, что у машины есть убеждения, желания, намерения, сознание или свобода воли. Интересно сравнить эту статью с книгой Гриффина.

Meschkowski, Herbert. «Non-Euclidean Geometry». New York: Academic Press, 1964. Небольшая книга с хорошими историческими сведениями.

Meyer, Jean. «»Essai d'application de certain modeles cybernetiques к la coordination chez les insectes sociaux». «Insectes Sociaux» XIII, no. 2 A966): 127. Статья, проводящая некоторые параллели между нейронной организацией мозга и организацией муравьиной колонии.

Meyer, Leonard В. «Emotion and Meaning in Music». Chicago: University of Chicago Press, 1956. Автор пытается приложить идеи гештальт психологии и теории восприятия к объяснению музыкальной структуры. Одна из самых необычных книг о музыке и разуме.

-\/-\/- «Music, the Arts, And Ideas». Chicago: University of Chicago Press, 1967. Вдумчивый анализ мыслительных процессов, участвующих в слушании музыки, а также иерархических структур музыки. Автор сравнивает современные тенденции в музыке с идеями дзен-буддизма.

Miller, G. A., and P. N. Johnson-Laird. «Language and Perception». Cambridge: Mass. Harvard University Press, Belknap Press, 1976. Интереснейшее собрание лингвистических фактов и теорий, тесно связанных с гипотезой Уорфа о том, что язык определяет мировоззрение. Типичный пример --- обсуждение странного «тещиного языка» у народности Двирбал в Австралии: специального языка, используемого только для разговоров с тещей.

** Minsky, Marvin L. «Matter, Mind, and Models». В сб. Marvin L. Minsky, ed. «Semantic Information Processing» Cambridge, Mass.: M.I.T. Press, 1968. Несколько страниц этой короткой статьи вмещают целую философию сознания и искусственного разума. Выдающаяся статья, написанная одним из глубочайших мыслителей в этой области.

Minsky, Marvin L, and Seymour Papert. «Artificial Intelligence Progress Report». Cambridge, Mass/ M.I.T. Artificial Intelligence Laboratory, AI Memo 252, 1972. Обзор работы в области искусственного интеллекта, проведенной в МИТ до 1972 года, в соотношении с эпистемиологией и психологией. Может служить отличным введением в курс по исскусственному разуму.

** Monod, Jacques. «Chance and Necessity». New York: Random House, Vintage Books, 1971. Исключительно плодовитый автор, пишущий в занимательной манере об интересных вопросах: как живое строится из неживого и каким образом эволюция зависит от второго закона термодинамики, хотя и кажется, что она его нарушает. На мой взгляд, замечательная книга.

* Morrison, Philip and Emily, eds. «Charles Babbage and His Calculating Engines». New York: Dover Publications, 1961. Ценный источник информации о жизни Баббаджа. Приводит значительную часть автобиографии изобретателя, вместе с несколькими статьями о его машинах и «механической нотации».

Myhill, John. «Some Philosophical Implications of Mathematical Logic». «Review of Metaphysics» 6 A952): 165. Необычный взгляд на то, каким образом теоремы Гёделя и Черча связаны с эпистемиологией и психологией. Заканчивается обсуждением понятий красоты и творческих способностей.

Nagel, Ernest. «The Structure of Science» New York: Harcourt, Brace, and World, 1961. Классический труд по философии науки; четкое обсуждение различий между редукционизмом и холизмом, теологическими и нетеологическими объяснениями и т. д.

** Nagel, Ernest and James R. Newman. «Godel's Proof». New York: New York University Press, 1958. Это занимательное изложение во многом послужило источником вдохновения для моей книги.

* Nievergelt, Jurg, J. С Farrar, and E. M. Reingold. «Computer Approaches to Mathematical Problems». Englewood Cliffs, N. J.: Prentice Hall, 1974. Довольно необычное собрание различных типов задач, которые могут решаться или уже решались с помощью компьютеров --- например, задача «Зn+1», о которой я упомянул в «Арии с различными вариациями», и другие задачи из области теории чисел.

Pattee, Howard H., ed. «Hierarchy Theory: The Challenge of Complex Systems» New York: George Braziller, 1973. Включает хорошую статью Херберта Саймона, обсуждающую примерно те же идеи, как и моя глава «Уровни описания».

Peter Rozsa. «Recursive Functions». New York- Academic Press, 1967. Подробное обсуждение примитивно рекурсивных функций, общерекурсивных функций, частично рекурсивных функций, диагонального метода и других технических вопросов.

Quine, Willard Van Orman. «The Ways of Paradox, and Other Essays» New York: Random House, 1966. Собрание размышлений Квайна на многие темы Первое эссе описывает разнообразные парадоксы и их решения В нем Квайн объясняет операцию, которая в моей книге названа «квайнированием».

Ranganathan, S. R. «Ramanujan, The Man and the Mathematician». London1 Asia Publishing House, 1967. Склоняющаяся к мистике биография индийского гения, написанная его поклонником. Странная, но интересная книга.

Reichardt, Jasia. «Cybernetics, Arts, and Ideas». Boston' New York Graphic Society, 1971. Причудливое собрание идей о компьютерах, искусстве, музыке и литературе. Некоторые идеи там совершенно сумасшедшие --- но не все. Интересны статьи J. R. Pierce «A Chance for Art» и Margaret Masterman «Computerized Haiku». Renyi, Alfred. «Dialogues on Mathematics». San Francisco- Holden-Day, 1967. Три простых, но стимулирующих диалога между классическими историческими персонажами, пытающимися понять природу математики. Для широкой публики.

** Reps Paul. «Zen Flesh, Zen Bones». New York: Doubleday, Anchor Books. Отлично передает дух дзена, его антирациональную, антисловесную, антиредукционисткую и в основном холистскую природу.

Rogers, Hartley. «Theory of Recursive Functions and Effective Computability». New York: McGraw Hill, 1967. Техническая книга, читая которую, можно многому научиться. Содержит обсуждение интригующих задач по теории множеств и теории рекурсивных функций.

Rokeach, Milton. «The Three Christs of Ypsilanti». New York Vintage Books, 1964 . Изучение шизофрении и тех странных типов «последовательности», которые возникают в сознании больного. Описывает удивительный конфликт между тремя пациентами психиатрической больницы, каждый из которых считал себя Богом, и то, как они смогли жить бок о бок в течение многих месяцев.

** Rose, Steven. «The Conscious Brain», исправленное изд. New York: Vintage Books, 1976. Замечательная книга --- возможно, лучшее введение в изучение мозга. Содержит полное обсуждение физической структуры мозга, а также философское обсуждение природы разума, редукционизма vs. холизма, свободы воли vs. детерминизма и так далее. Автор представляет широкую, разумную и человечную точку зрения; неверны только его идеи по поводу ИИ.

Rosenblueth, Arturo. «Mind and Brain: A Philosophy of Science». Cambridge, Mass.- M.I.T. Press, 1970. Хорошая книга, написанная специалистом-невро-специалистом-неврологом. Затрагивает большинство проблем, касающихся мозга и разума.

* Sagan, Carl, ed. «Communication with Extraterrestrial Intelligence». Cambridge, Mass.: M.I.T. Press, 1973. Доклады интереснейшей конференции, в которых группа знаменитостей и других ученых спорят об этом гипотетическом вопросе.

Salmon, Wesley, ed. «Zeno's Paradoxes». New York: Bobbs-Merrill, 1970. Сборник статей о старинных парадоксах Зенона в свете современной теории множеств, квантовой механики и так далее Интересно и иногда забавно; наводит на размышления.

Sanger, F., et al. «Nucleotide sequence of bacteriophage 0X174 DNA», «Nature» 265 (февраль 24, 1977). Интереснейшее описание впервые найденного полного генетического материала живого организма. Неожиданная двусмысленность: коды двух белков накладываются друг на друга! Почти невероятно.

Sayre, Kenneth M., and Frederick J. Crosson. «The Modeling of Mind: Computers and Intelligence». New York: Simon and Schuster, Clarion Books, 1963. Собрание философских комментариев ученых из различных областей науки об идее искусственного интеллекта (Anatol Rapoport, Ludwig Wittgenstein, Donald Mackay, Michael Scriven, Gilbert Ryle и другие).

*Schank, Roger, and Kenneth Colby. «Computer Models of Thought and Language». San Francisco: W. H. Freeman, 1973. Сборник статей, представляющих разный подход к вопросам стимуляции мыслительных процессов, таких как понимание языка, системы убеждений, перевода и так далее. Важная книга по ИИ; многие статьи легки для чтения даже для неспециалистов.

Schrodinger, Erwin. «What is Life? \& Mind and Matter». New York: Cambridge University Press, 1967 . Знаменитая книга знаменитого физика (одного из основателей квантовой механики). Исследует физический фундамент жизни и мозга и обсуждает сознание в довольно метафизических терминах. Первая часть книги, What is Life?», в 1940-х годах оказала сильное влияние на поиски переносчиков генетической информации.

Shepard, Roger N. «Circularity in Judgments of Relative Pitch». «Journal of the Acoustical Society of America» 36, номер 12 (декабрь 1964), стр. 2346-2353. Источник удивительной слуховой иллюзии «тональной системы Шепарда».

Simon, Herbert A. «The Science of the Artificial». Cambridge, Mass.: M.I.T. Press, 1969. Интересная книга о понимании сложных систем. В последней главе, «Architecture of Complexity», затрагиваются проблемы редукционизма и холизма.

Smart, J. J. С. «Godel's Theorem, Church's Theorem, and Mechanism». «Synthese» 13 A961): 105. Хорошо написанная статья, предваряющая статью Лукаса 1961 года и содержащая аргументы против идей Лукаса.

** Smullyan, Raymond. «Theory of Formal Systems». Princeton, N. J.: Princeton University Press, 1961 Серьезный труд, начинающийся с превосходного обсуждения формальных систем; содержит элегантное доказательство упрощенной версии теоремы Гёделя. Заслуживает внимания хотя бы только из-за первой главы.

*-\/-\/- «What is the Name of This Book?» Englewood Cliffs, N. J. Prentice Hall, 1978. Скорее всего, эта книга должна понравиться многим из моих читателей. Она вышла из печати, когда моя книга уже была целиком написана, за исключением некоей записи в библиографии.

Sommerhoff, Gerd. «The Logic of the Living Brain». New York: John Wiley, 1974. Автор пытается, используя знания о микроструктурах мозга, создать теорию работы мозга как целого.

Sperry, Roger. «Mind, Brain, and Humanist Values». В сб. John R. Platt, ed. «New Views on the Nature of Man». Chicago: University of Chicago Press, 1965. Ведущий нейрофизиолог живым языком объясняет, как, по его мнению, мозговая деятельность сочетается с сознанием.

* Steiner, George. «After Babel: Aspects of Language and Translation». New York: Oxford University Press, 1975. Эта книга, написанная ученым-лингвистом, посвящена сложным проблемам перевода и понимания языка людьми. Хотя ИИ практически не упоминается, ясно, что автор убежден в том, что стараться запрограммировать компьютер на понимание романов или стихов, --- безнадежное занятие. Хорошо написанная книга, наводящая на размышления и местами раздражающая.

Stenesh, J. «Dictionary of Biochemistry». New York: John Wiley, Wiley Interscience, 1975. Эта книга послужила для меня полезным дополнением к технической литературе по молекулярной биологии.

** Stent, Gunther. «Explicit amd Implicit Semantic Content of the Genetic Information». В сб. «The Centrality of Science and Absolute Values», том 1. Proceedings of the 4th International Conference on the Unity of the Sciences, New York, 1975. Удивительно то, что эта статья находится среди материалов конференции, организованной ныне дезакредитированным преподобным отцом Сун Мьюнг Мун. Тем не менее, статья отличная. В ней обсуждается вопрос о том, можно ли сказать, в некоем практическом смысле, что в генотипе содержится «вся» информация о его фенотипе. Иными словами, тема статьи --- местонахождение значения в генотипе.

-\/-\/- «Molecular Genetics: A Historical Narrative». San Francisco: W. H. Freeman, 1971. Стент представляет широкую, гуманистическую точку зрения и ставит идеи в историческую перспективу. Необычный текст по молекулярной биологии.

Suppes, Patrick. «Introduction to Logic». New York: Van Nostrand Reinhold, 1957. Стандартный текст, ясно излагающий исчисление высказываний и исчисление предикатов. Моя глава об исчислении высказываний базируется, в основном, на этой книге.

Sussman, Gerald Jay. «A Computer Model of Skill Acquisition». New York: American Elsevier», 1975. Теория программ, объясняющая задачи программирования компьютера. Детально обсуждается, как можно разбить задание на подзадачи и как эти подзадачи взаимодействуют.

** Tanenbaum, Andrew S. «Structured Computer Organization». Englewood Cliffs, N. J.: Prentice Hall, 1976. Великолепно прямолинейное, замечательно написанное объяснение многих уровней современных компьютерных систем Включает главы о языках микропрограммирования, машинных языках, языках ассемблера, операционных системах и так далее. Хорошая, частично аннотированная библиография.

Tarski, Alfred. «Logic, Semantics, Metamathematics». Статьи 1923-1938 годов Перевод J. H. Woodger. New York Oxford University Press, 1956. Излагает идеи о истине и представленном ею отношении между языком и миром. Эти идеи все еще влияют на проблемы представления знаний в ИИ.

Taube, Mortimer. «Computers and Common Sense». New York: McGraw Hill, 1961. Возможно, первый протест против искусственного разума, книга раздражает.

Tietze, Heinrich. «Famous Problems of Mathematics». Baltimore: Graylock Press, 1965. Книга о знаменитых задачах, написанная в оригинальном и эрудированном стиле. Хорошие иллюстрации и исторические материалы.

Trakhtenbrot, V. «Algorithms and Computing Machines». Heath. Обсуждение теоретических вопросов, касающихся компьютеров, --- в особенности, неразрешимых проблем, таких как проблема остановки и проблема словесной эквивалентности. Одно из достоинств книги --- её краткость.

Turing, Sara. «Alan M. Turing». Cambridge, U.K.: W. Heffer \& Sons, 1959. Биография великого пионера компьютерного дела, с любовью составленная его матерью.

* Ulam, Stanislav. «Adventures of a Mathematician». New York: Charles Scribner's, 1976. Автобиография 65-летнего человека, написанная так, словно автору 20 лет; он безумно влюблен в математику. Книга полна сведений о том, кто кого считал лучшим, кто кому завидовал и так далее. Не только интересное, но и серьезное чтение.

Watson, J. D. «Molecular Biology of the Gene», 3rd ed. Menlo Park, Calif.: W. A. Benjamin, 1976. Хорошая книга, но, по моему мнению, далеко не так четко организованная, как труд Ленингера. Тем не менее, почти на каждой странице есть что-то интересное.

Webb, Judson. «Metamathematics and the Philosophy of Mind». «Philosophy of Science» 35 A968): 156. Подробные и строгие доводы против Лукаса. Автор утверждает: «Моя позиция в данной статье такова: проблема мозга-машины Гёделя не сможет получить связного объяснения до тех пор, пока не будет разрешена проблема конструктивности в основаниях математики.»

Weiss, Paul. «One Plus One Does Not Equal Two». В сб. G. С. Quarton, Т. Melnechuk, and F. 0. Schmitt, ed. «The Neuroscience: A Study Program». New York: Rockefeller University Press, 1967. Статья, пытающаяся примирить холизм и редукционизм, но, на мой вкус, слишком холистски направленная.

* Weizenbaum, Joseph. «Computer Power and Human Reason». San Francisco: W. H. Freeman, 1976. Вызывающая книга одного из первых специалистов по ИИ, который пришел к выводу, что многие исследования в области вычислительной техники --- особенно, в области ИИ --- опасны. Хотя я и могу согласиться с некоторыми из критических идей автора, мне все же кажется, что он заходит слишком далеко. Когда он называет работников в области искусственного интеллекта «искусственной интеллигенцией» в первый раз, это забавно, но после десятого раза надоедает. Каждый, интересующийся компьютерами, должен прочесть эту книгу.

Wheeler, William Morton. «The Ant-Colony as an Organism». «Journal of Morphology» 22, 2 A911): 307-325. Один из ведущих специалистов-энтомологов своего времени объясняет, почему муравьиная колония, так же, как и её отдельные части, заслуживает называться «организмом».

Whitely, С. Н. «Minds, Machines, and Godel: A Reply to Mr Lucas». «Philosophy» 37 A967); 61. Простые, но убедительные контраргументы против Лукаса.

Wilder, Raymond. «An Introduction to the Foundations of Mathematics». New York: John Wiley, 1952. Хороший обзор, представляющий в перспективе важные идеи прошлого века.

* Wilson, Edward О. «The Insect Societies». Cambridge, Mass.: Harvard University Press, Belknap Press, 1971. Авторитетный труд об общественном поведении насекомых. Несмотря на обилие деталей, книга доступна для широкой публики и обсуждает множество интереснейших идей. Превосходные иллюстрации и обширная (но, к сожалению, не аннотированная) библиография.

Winograd, Terry. «Five Lectures on Artificial Intelligence» AI Memo 246 Stanford, Calif.: Stanford University Artificial Intelligence Laboratory, 1974. Один из ведущих современных работников по искусственному интеллекту предлагает описание основных проблем ИИ и новых идей для их решения.

*-\/-\/- «Language as a Cognitive Process». Reading, Mass.: Addison-Wesley. Интереснейшая книга, как никакая другая до сих пор, описывающая язык во всей его сложности.

*-\/-\/- «Understanding Natural Language». New York: Academic Press, 1972. Подробное описание программы, удивительно сообразительной в ограниченном контексте. Книга показывает, что язык неотделим от общего понимания мира и предлагает направление работы по созданию программ, понимающих язык, как люди. Важная контрибуция, стимулирующая множество идей.

-\/-\/- «On some contested suppositions of generative linguistics about the scientific study of language», «Cognition» 4:6. Забавное опровержение яростных атак некоторых лингвистов-доктринеров на искусственный интеллект.

* Winston, Patrick. «Artificial Intelligence». Reading, Mass.. Addison-Wesley, 1977. Общий обзор многих сторон ИИ, предложенный страстным исследователем этой области, молодым, но уже известным. Первая часть независима от программ, вторая часть зависит от ЛИСПа и включает краткое объяснение этого языка. Содержит множество ссылок на современную литературу по ИИ.

*-\/-\/- , ed. «The Psychology of Computer Vision». New York McGraw Hill, 1975. Неудачное название, но отличная книга Содержит статьи о том, как запрограммировать компьютер на зрительное узнавание предметов, сцен и т. д. Затронуты все уровни проблемы, от узнавания отрезков прямой до общей организации знаний. В частности, в книге опубликована статья самого Уинстона о его программе, способной развивать абстрактные понятия на основе конкретных примеров, и статья Минского о зарождающемся понятии фреймов.

* Wooldridge, Dean. «Mechanical Man --- The Physical Basis of Intelligent Life» New York: McGraw Hill. 1968. Глубокое и ясно изложенное обсуждение отношения мыслительных процессов к физическим процессам мозга По новому исследует сложные философские понятия, поясняя их на конкретных примерах.
